\chapter{二阶线性偏微分方程的分类与化简}

\begin{table}[htbp]
	\centering
	\caption{本章研究对象}
	\begin{tabular}{|c|c|c|c|}
		\hline
		双曲型 & \(u_{\xi\xi}-u_{\eta\eta}=\ldots\) & 波动方程为代表 & \(\frac{\partial^2u}{\partial t^2}-a^2\frac{\partial^2u}{\partial x^2}=f(x,t)\) \\
		\hline
		抛物型 & \(u_{\eta\eta}=Au_\xi+\ldots\) & 热传导方程为代表 & \(\frac{\partial u}{\partial t}-a^2\frac{\partial^2u}{\partial x^2}=f(x,t)\) \\
		\hline
		椭圆型 & \(u_{\xi\xi}+u_{\eta\eta}=\ldots\) & 位势方程为代表 & \(\frac{\partial^2u}{\partial x^2}+\frac{\partial^2u}{\partial y^2}=f(x,y)\) \\
		\hline
	\end{tabular}
\end{table}

2个自变量的一般形:
\[
\underbrace{a_{11}u_{xx}+2a_{12}u_{xy}+a_{22}u_{yy}}_{\text{二阶主部项}}+\underbrace{b_1u_x+b_2u_y+cu}_{\text{低阶项}}=f
\]

其中\(a_{11},a_{12},a_{22},b_1,b_2,c,f\)都是区域\(\Omega\)上的实函数。例如\(u_{xx}+5u_{xy}-4u_{yy}=1\),其中\(a_{11}=1,a_{22}=-4,a_{12}=5/2\)

\(n\)个自变量时一般形式:\(\sum\limits_{i,j=1}^{n}\frac{a_{ij}\left(\partial^2u\right)}{\partial x_i\partial x_j}+\sum\limits_{i=1}^{n}{b_i\frac{\partial u}{\partial x_i}}+cu+f=0,a_{ij}=a_{ji}\)其中\(a_{ij},b_i,c,f\)是自变量\(x_1,x_2,\ldots,x_n\)的函数

方程化简:重点引入非奇异变化,将方程化为三类方程的标准型。

\section{两个自变量的方程的分类与化简}

\subsection{分类情况}

记\(\Delta=a_{12}^2-a_{11}a_{22}\),则有:
\[\begin{cases}
\Delta>0\text{双曲型}u_{xx}-a^2u_{yy}=f\\
\Delta=0\text{抛物型}u_{x}-a^2u_{yy}=f\\
\Delta<0\text{椭圆型}u_{xx}+a^2u_{yy}=f
\end{cases}\]

常系数方程分类是全局的,而变系数方程的分类是依赖\(x,y\)的,如特里波利方程(混合类型):\(yu_{xx}+u_{yy}=0\rightarrow\Delta=-y\)其类型取决于\(y\)的符号(\(x\)轴上方椭圆型,\(x\)轴抛物型,\(x\)轴下双曲型)

\subsection{方程化简}
寻求一个非奇异变换,使得原方程\(a_{11}u_{xx}+2a_{12}u_{xy}+a_{22}u_{yy}+b_1u_x+b_2u_y+cu=f\)化简为相应的标准形式(希望二阶主部项系数两个为零)。其中\(a_{11},a_{12},a_{22},b_1,b_2,c,f\)都是区域\(\Omega\)上的实函数

\subsubsection{推导}
\begin{enumerate}
	\item 作非奇异变量代换\(\begin{cases}\xi=\xi(x,y)\\\eta=\eta(x,y)\end{cases}\)	(将原坐标→新坐标用于简化,我们的目的是找到合适的这个代换)
	\item 转换为\(A_{11}u_{\xi\xi}+2A_{12}u_{\xi\eta}+A_{22}u_{\eta\eta}+B_1u_\xi+B_2u_\eta+Cu=F\)
	\item 适当取变换,使得\(A_{ij}\)部分为0(两个为零),达到化简之目的
\end{enumerate}

\subsubsection{代换}

非奇异变化:雅克比(Jacobi)行列式在点\((x_0,y_0)\)不等于零,则在点\((x_0,y_0)\)附近变换是可逆的。 两个曲面要求相交,保证变化前后同号。
\[
J=\frac{\partial(\xi,\eta)}{\partial(x,y)}=\begin{vmatrix}\xi_x&\xi_y\\\eta_x&\eta_y\\\end{vmatrix}\neq0
\]
					
具体变换:\(u(x,y)\rightarrow u(\xi(x,y),\eta(x,y))\rightarrow u(\xi,\eta)\)保证新旧坐标互相变换。将方程转型为:\(A_{11}u_{\xi\xi}+2A_{12}u_{\xi\eta}+A_{22}u_{\eta\eta}+B_1u_\xi+B_2u_\eta+Cu=F\),其中的新系数为:
\begin{gather*}
A_{11}=a_{11}\xi_x^2+2a_{12}\xi_x\xi_y+a_{22}\xi_y^2\\
A_{12}=a_{11}\xi_x\eta_x+a_{12}\left(\xi_x\eta_y+\xi_y\eta_x\right)+a_{22}\xi_y\eta_y\\
A_{22}=a_{11}\eta_x^2+2a_{12}\eta_x\eta_y+a_{22}\eta_y^2\\
B_1=a_{11}\xi_{xx}+2a_{12}\xi_{xy}+a_{22}\xi_{yy}+b_1\xi_x+b_2\xi_y\\
B_2=a_{11}\eta_{xx}+2a_{12}\eta_{xy}+a_{22}\eta_{yy}+b_1\eta_x+b_2\eta_y\\
C_1=c\quad F=f  
\end{gather*}

系数由链式法则得到:\(u_x=u_\xi\xi_x+u_\eta\eta_x\quad u_y=u_\xi\xi_y+u_\eta\eta_y\)二阶导数同理

\subsubsection{性质}
\begin{enumerate}
	\item \(\left[\begin{matrix}A_{11}&A_{12}\\A_{12}&A_{22}\\\end{matrix}\right]=\left[\begin{matrix}\xi_x&\xi_y\\\eta_x&\eta_y\\\end{matrix}\right]\left[\begin{matrix}a_{11}&a_{12}\\a_{12}&a_{22}\\\end{matrix}\right]\left[\begin{matrix}\xi_x&\xi_y\\\eta_x&\eta_y\\\end{matrix}\right]^\top\)两边取行列式:\(A_{12}^2-A_{11}A_{22}=\left(a_{12}^2-a_{11}a_{22}\right)J^2=J^2\Delta\)非奇异变化能够保证转换后类型不变。
	\item 发现\(A_{11}\)与\(A_{22}\)有相同的形式,可以尝试让这两个为零,即\(A_{11}=a_{11}\xi_x^2+2a_{12}\xi_x\xi_y+a_{22}\xi_y^2=0,A_{22}=a_{11}\eta_x^2+2a_{12}\eta_x\eta_y+a_{22}\eta_y^2=0\)\\
	解方程\(a_{11}\phi_x^2+2a_{12}\phi_x\phi_y+a_{22}\phi_y^2=0\)得到两个无关解\(\phi_1(x,y),\phi_2(x,y)\),那么就取 \(\begin{cases}\xi=\phi_1(x,y)\\\eta=\phi_2(x,y)\end{cases}\)可以得到\(A_{11}=A_{22}=0\)如果能求解到\(\phi_1(x,y),\phi_2(x,y)\),就可以获得这个变换了
\end{enumerate}
%%%%%%%%%%%%%%%%%%%%%%%%%%%%%%%5
\subsubsection{求解过程}

\(a_{11}\phi_x^2+2a_{12}\phi_x\phi_y+a_{22}\phi_y^2=0\)这是一个完全非线性方程,其解应当为\(\phi(x,y)\)的形式,表现为一个曲面。如果有两个解,就是两个曲面,如果无关,一定相交。

方程中假设\(\phi_x^2+\phi_y^2\neq0\),设\(y\neq0\)(要求非平凡解(常数解)),那方程等价于\(a_{11}\left(\frac{\phi_x}{\phi_y}\right)^2+2a_{12}\frac{\phi_x}{\phi_y}+a_{22}=0\)

\begin{figure}
	\centering
	\caption{两族曲线}\label{pic:2}
	\begin{tikzpicture}
		\draw[<->] (4,0)node[right]{\(x\)}--(0,0)node[below,left]{\(O\)}--(0,4)node[above]{\(y\)};
		\draw (1.5,0.5) arc (-25:75:2 and 1.5);
		\draw (1.5,3.5) arc (25:-75:2 and 1.5);
		\draw[dashed] (2.5,0.5) arc (-25:75:2 and 1.5);
		\draw[dashed] (2.5,3.5) arc (25:-75:2 and 1.5);
		\draw[dotted] (3.5,0.5) arc (-25:75:2 and 1.5);
		\draw[dotted] (3.5,3.5) arc (25:-75:2 and 1.5);
	\end{tikzpicture}
\end{figure}

想象在空间\(Oxyz\)内有两个无关解平面,用\(z=c(\phi(x,y)=c)\),让\(c\)不断变动,则投影如图\ref{pic:2}所示,是两族曲线。在曲线两边做微分\(0=\mathrm{d}\phi=\phi_x\,\mathrm{d}x+\phi_y\,\mathrm{d}y\),用\(-\frac{\mathrm{d}y}{\mathrm{d}x}\)替代\(\frac{\phi_x}{\phi_y}\),则非线性偏微分方程转换为了常微分方程。

可得特征方程:\(a_{11}\left(\frac{\mathrm{d}y}{\mathrm{d}x}\right)^2-2a_{12}\frac{\mathrm{d}y}{\mathrm{d}x}+a_{22}=0\)和特征线方程:\(\begin{cases}\frac{\mathrm{d}y}{\mathrm{d}x}=\frac{a_{12}+\sqrt\Delta}{a_{11}}\\\frac{\mathrm{d}y}{\mathrm{d}x}=\frac{a_{12}-\sqrt\Delta}{a_{11}}\end{cases}\)。根据\(\Delta\)的符号给出常微分方程相应的解(特征线),利用特征线方程可以进一步解的特征面(例如,\(\frac{\mathrm{d}y}{\mathrm{d}x}=x\rightarrow y=\frac{1}{2}x^2+c\rightarrow y-\frac{1}{2}x^2=z\))

化简方程过程总结:原方程→特征方程→特征曲线→变量非奇异变换→未知函数及其导数的导数变换→结合原方程→简化后的标准型。

\subsection{方程化简的三类情况}

\subsubsection{双曲型\(\Delta>0\)}

方程有两族不同的实解曲线\(\phi_1(x,y)=c_1,\phi_2(x,y)=c_2\)又有\(\begin{cases}\xi=\phi_1(x,y)\\\eta=\phi_2(x,y)\end{cases}\)

双曲型方程第一标准形式:\(2A_{12}u_{\xi\eta}+A_1u_\xi+B_1u_\eta+C_1u=F\)

\paragraph{例题}
\begin{enumerate}
	\item \(u_{xx}-4u_{yy}+u_x\)特征方程及求解特征坐标\\
	\(\begin{cases}\frac{\mathrm{d}y}{\mathrm{d}x}=\frac{a_{12}+\sqrt\Delta}{a_{11}}=2\\\frac{\mathrm{d}y}{\mathrm{d}x}=\frac{a_{12}-\sqrt\Delta}{a_{11}}=-2\end{cases}\rightarrow\begin{cases}y=-2x+c_1\\y=2x+c_2\end{cases}\rightarrow\begin{cases}\xi=y+2x=c_1\\\eta=y-2x=c_2\end{cases}\)(得到新的特征坐标)
	\item 简化方程\(y^2u_{xx}-x^2u_{yy}=0(x>0y>0)\)\\
	同理\(u_{\xi\eta}=\frac{\eta u_xi-\xi u_\eta}{2(\xi^2-\eta^2)}\)
	\item 证明方程\(3u_{xx}+7u_{xy}+2u_{yy}=0\)对所有的\(x,y\)是双曲型的,并求出新的特征坐标\\
	双曲形:\(\Delta=\left(\frac{7}{2}\right)^2-3\times 2=\frac{25}{4}>0\)。特征坐标\(\xi=y-2x,\eta=y-\frac{1}{3}x\)。
	\item 对方程\(u_{xx}+4u_{xy}=0\)求出新的特征坐标,并简化方程,再求解简化后的方程
\end{enumerate}

\subsubsection{抛物型\(\Delta=0\)}
\[
\frac{\mathrm{d}y}{\mathrm{d}x}=\frac{a_{12}+\sqrt\Delta}{a_{11}}\xrightarrow{\left(\Delta=0\right)}\frac{\mathrm{d}y}{\mathrm{d}x}=\frac{a_{12}}{a_{11}}\rightarrow\phi_1(x,y)=C\left(A_{11}=0orA_{22}=0\right) 
\]

此事只能解到一个方程,令\(\xi=\phi_1(x,y)\),取\(\eta=y \vee \eta=x\)(根据实际情况选取)。抛物型方程的标准形式为:
\[
u_{\eta\eta}=A_4u_\xi+B_4u_\eta+C_4u+F_4
\]

\paragraph{例题}化方程\(u_{xx}+2u_{xy}+u_{yy}=0\)
\[
\frac{\mathrm{d}y}{\mathrm{d}x}=1\rightarrow y=x+c\rightarrow\begin{cases}\xi=y-x\\\eta=y\end{cases}\rightarrow u_{\eta\eta}=0
\]

\subsubsection{椭圆型\(\Delta<0\)}
\[
\frac{\mathrm{d}y}{\mathrm{d}x}=\frac{a_{12}\pm\sqrt\Delta}{a_{11}}=\frac{a_{12}\pm i\sqrt{-\Delta}}{a_{11}}\rightarrow\frac{\mathrm{d}y}{\mathrm{d}x}=p(x,y)\pm iq(x,y)\rightarrow\begin{cases}\phi_1(x,y)=\alpha+\mathrm{i}\beta\\\phi_2(x,y)=\alpha-\mathrm{i}\beta\end{cases}
\]

取变换\(\begin{cases}\xi=\alpha(x,y)\text{实部}\\\eta=\beta(x,y)\text{虚部}\end{cases}\)

例如:\(\begin{cases}\frac{\mathrm{d}y}{\mathrm{d}x}=-\sqrt{-4x^2}=-2\mathrm{i}x\\\frac{\mathrm{d}y}{\mathrm{d}x}=\sqrt{-4x^2}=2\mathrm{i}x\end{cases}\rightarrow\begin{cases}\xi=y+\mathrm{i}x^2\\\eta=y-\mathrm{i}x^2\end{cases}\xrightarrow{\text{需求实变量}}\begin{cases}\xi=y\\\eta=x^2\end{cases}\)挑选一个特征方程

椭圆方程的标准形式:\(A_{11}u_{\xi\xi}+A_{22}u_{\eta\eta}=A_5u_\xi+B_5u_\eta+C_5u+F_5,A_{11}=A_{22}\)

\paragraph{例题}判断方程类型并化简\(y^2u_{xx}+x^2u_{yy}=0\)