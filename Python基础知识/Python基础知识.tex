\documentclass[oneside]{ctexbook}
\usepackage{listings}
\usepackage{graphicx}
\usepackage{amsmath}
\usepackage{amssymb}
\usepackage{enumitem}
\usepackage{booktabs}
\usepackage{array}
\usepackage{ulem}
\usepackage{pifont}
\usepackage{tikz}
\usepackage{geometry}
\geometry{a4paper,hmargin=2.5cm,vmargin=2cm}

\newcommand{\bookname}{Python基础知识}
\newcommand{\uira}{https://github.com/ZhangtongCN}
\newcommand{\uirb}{https://github.com/Clignniis}
\usepackage[toc]{multitoc}%双栏目录
%颜色设置
\usepackage{xcolor}
\definecolor{nuist}{RGB}{0, 103, 156}%南信大蓝
\definecolor{sky}{RGB}{101, 170, 221}%天空蓝
\definecolor{tech}{RGB}{38, 96, 173}%科技蓝
\definecolor{gold}{RGB}{201, 160, 99}%高贵金
%引用
\usepackage[colorlinks]{hyperref}
%设置页眉页脚

%章节设置
\ctexset{
    chapter={pagestyle=fancy},%使得章节页页眉页脚格式一致
}
\geometry{a4paper,top=2.5cm,bottom=2.5cm}
\setcounter{tocdepth}{1}
\linespread{1.3}\selectfont

%页眉页脚
\usepackage{fancyhdr}
\pagestyle{fancy}
\renewcommand{\sectionmark}[1]{\markright{\thesection\ #1}}
\fancyhf{}
\fancyhead[L]{\href{\uira}{\textcolor{tech}{\bookname}}}
\fancyhead[R]{\href{\uira}{\textcolor{tech}{笔记整理与汇总}}}
\fancyfoot[C]{\href{\uirb}{\textcolor{tech}{\thepage}}}
\renewcommand{\headrulewidth}{0pt}
\setlength{\headheight}{27pt}


\lstset{
    basicstyle=\ttfamily,
    keywordstyle=\color{blue},
    commentstyle=\color{gray},
    stringstyle=\color{red},
    numbers=left,
    numberstyle=\tiny,
    frame=single,
    breaklines=true,
    tabsize=4,
    showstringspaces=false,
    escapeinside={@*}{*@}
}

\begin{document}

\frontmatter

\begin{titlepage}
    \begin{tikzpicture}[remember picture, overlay]
        \node at ([shift={(0,1)}]current page.center){
            \begin{tikzpicture}[scale=0.7]
            \path[fill=sky] (0:0)--++(90:2)--++(30:6)--++(150:6)--++(210:6)--++(330:6)--++(90:1)--++(150:4)--++(30:4)--++(330:4)--++(210:4)--++(270:3)--++(150:1)--++(210:3)--++(270:1)--++(30:5)--++(330:3)--++(270:1)--++(210:4)--++(150:4)--++(210:1)--++(330:5)--++(30:5)--cycle;
            \path[fill=cyan!50] (0:0)--++(270:1)--++(210:3)--++(150:1)--++(90:5)--++(150:1)--++(270:6)--++(30:6)--++(90:1)--++(30:4)--++(270:6)--++(210:5)--++(270:1)--++(30:6)--++(90:8)--++(210:6)--cycle;
            \path[fill=cyan!50] (90:6)--++(90:1)--++(210:4)--++(330:1)--++(30:3)--++(90:1);
            \path[fill=nuist] (0:0)--++(270:1)--++(330:3)--++(30:1)--++(90:5)--++(30:1)--++(270:6)--++(150:6)--++(90:1)--++(150:4)--++(270:6)--++(330:5)--++(270:1)--++(150:6)--++(90:8)--++(330:6)--cycle;
            \path[fill=nuist] (90:6)--++(90:1)--++(330:4)--++(210:1)--++(150:3)--++(90:1);
            \end{tikzpicture}
        };
        \node at ([shift={(0,8)}]current page.center){\href{\uirb}{\fontsize{72}{0}\selectfont\fontspec{Times New Roman}[scale=1.0]\heiti\textcolor{gold}{\bookname}}};
        \node at ([shift={(0,-5)}]current page.center){\Huge\heiti\href{\uira}{\textcolor{nuist}{Tong Zhang}}\quad\href{\uirb}{\textcolor{nuist}{Cls}}};
        \node[anchor= south] at ([shift={(0,6)}]current page.south){\href{\uirb}{\fontsize{48}{0}\heiti \textcolor{tech}{笔记整理与汇总}}};
    \end{tikzpicture}
\end{titlepage}

\chapter*{版权声明}

本《\bookname{} 笔记整理与汇总精编版》一册电子版遵循有限的知识共享许可协议。本书授权包含署名-非商业性使用-相同方式共享(CC BY-NC-SA)。即您被允许在授权范围内对该电子书进行转载、节选、二次创作,但不得用于任何商业目的,且使用时须署原作名,且必须采用与本创作相同的协议(CC BY-NC-SA)进行授权。限于编者水平,本书难免有疏漏错误,敬请读者批评指正。

\begin{tikzpicture}[remember picture, overlay]
    \node [opacity=1] at ([shift={(-4,4)}]current page.south east){
        \begin{tikzpicture}[scale=0.5]
            \foreach \a in {1,0.8,...,0.2} {%
                \draw (-\a,1) circle [radius=\a];
                \draw (-2+\a,-1) circle [radius=\a];
                \draw (2-\a,-1) circle [radius=\a];
            }
        \end{tikzpicture}
    };
\end{tikzpicture}

\tableofcontents

\mainmatter

\chapter{数据基本类型}
\section{数字类型}
\begin{itemize}
    \item \textcolor{blue}{\texttt{Type()}} 查询对象类型
    \item \textbf{int} 整形,任意大小的整数
    \item \textbf{Float} 浮点数,小数。
    \begin{quote}
        可写为 \texttt{4.3e-3} 的形式。
    \end{quote}
    \item \textbf{Bool} 布尔值,有 \texttt{True} 和 \texttt{False}
    \item \textbf{Complex} 复数,可以用 \texttt{complex(a,b)} 表示。
    \begin{quote}
        使用 \texttt{n.real} 获取实部,\texttt{n.imag} 获取虚部
    \end{quote}
    \item \textbf{None} 空值,是一个特殊的值。
\end{itemize}

\section{运算符}
\subsection{算术运算符}
\begin{itemize}
    \item \textcolor{red}{取模\%} 返回除法的余数
    \begin{quote}
        例如 \texttt{a=10; b=21; b\%a=1}
    \end{quote}
    \item \textcolor{red}{取整除//} 向下取接近除数的整数
    \begin{quote}
        例如 \texttt{9//2=4; -9//2=-5}
    \end{quote}
\end{itemize}

\subsection{赋值运算符}
\begin{itemize}
    \item \texttt{+=} \texttt{c+=a} 即 \texttt{c=c+a}
    \item \texttt{-=} \texttt{c-=a} 即 \texttt{c=c-a}
    \item \texttt{\%=} \texttt{c\%=a} 即 \texttt{c=c\%a}
\end{itemize}

\subsection{表达式-类型转换}
\begin{itemize}
    \item \texttt{Int()} 整形,舍弃小数部分
    \item \texttt{Float()} 浮点
    \item \texttt{Complex()} 复数
    \item \texttt{Hex()} 十六进制,或者输入 \texttt{0O}
    \item \texttt{Oct()} 八进制,或者输入 \texttt{0X}
    \item \texttt{Bin()} 二进制,或者输入 \texttt{0b}
    \item \textcolor{red}{\texttt{Chr()}} 转换成整数对应的ASCII字符
    \item \textcolor{red}{\texttt{Ord()}} 转换成ASCII对应的整数
\end{itemize}

\section{常见内置函数}
\begin{itemize}
    \item \textcolor{red}{\texttt{eval()}} 自动类型转换
    \item \texttt{Abs()} 绝对值
    \item \texttt{Max(a,b,c)} 最大值
    \item \texttt{Min(a,b,c,d)} 最小值
    \item \texttt{Pow(x,y)} \texttt{x**y} 运算
    \item \textcolor{red}{\texttt{Round(x, n)}} 返回浮点数x的四舍五入值,如给出 $n$ 的值,代表舍入到小数点后的位数。
    \item \textcolor{red}{\texttt{Divmod(x,y)}} 商余,同时输出商和余数,输出形式为 \texttt{(3,1)}
    \begin{quote}
        例如 \texttt{divmod(10,3)} $\rightarrow$ \texttt{(3,1)}
    \end{quote}
    \item \texttt{len(s)} 返回对象长度或项目个数
    \item \texttt{str()} 任意类型x对应的字符串形式
\end{itemize}

\chapter{字符串与列表}
\section{字符串}
\begin{itemize}
    \item \textcolor{red}{字符串的索引} \texttt{s="abcdefg"} 使用 \texttt{s[1]} 输出 \texttt{b}
    \item \textcolor{red}{多行字符串} 起止是3个单引号 \texttt{'''abcdef'''}
    \item \textcolor{red}{字符串运算符 切片运算符}
    \begin{itemize}
        \item \texttt{s[0:5]} 得到包含 \texttt{s[0]} 到 \texttt{s[4]} 的全部内容
        \item \texttt{[:N]} 从头开始
        \item \texttt{[:3]} 到第二个为止
        \item \texttt{[:]} 整个字符串
        \item \texttt{[M:N:K]} 根据\textbf{步长}对字符串切片
        \item \texttt{[1:8:2]} 一隔一取字符
        \item \texttt{[::-1]} \textbf{反向}(-1表示倒着取)
        \item \texttt{[0:-1]} 从0到最后一个前一个
    \end{itemize}
\end{itemize}

\section{字符串基本操作}
\subsection{基本函数}
\begin{itemize}
    \item \texttt{len(s)} 返回对象长度或项目个数(字符,列表,元组)
    \item \texttt{str(x)} 任意类型x对应的字符串形式,转换为字符串
    \item \texttt{oct(x)} 以八进制小写形式输出
    \item \texttt{hex(x)} 输出为十六进制
    \item \texttt{chr(u)} u为\textbf{整数},返回对应字符的Unicode编码
    \item \texttt{ord(x)} 输入\textbf{字符x},返回其对应的Unicode编码
\end{itemize}

\subsection{大小写转换}
\begin{itemize}
    \item \textcolor{red}{\texttt{upper()}} 转换为\textbf{大写}
    \item \textcolor{red}{\texttt{lower()}} 转换为\textbf{小写}
    \item \texttt{swapcase()} \textbf{翻转}大小写
    \item \texttt{capitalize()} 把字符串的\textbf{第一个字符}大写
    \item \texttt{isupper()} 若\textbf{都是大写},则返回 \texttt{True}
    \item \texttt{islower()} 若都是小写,则返回 \texttt{True}
    \item \texttt{title()} 所有单词\textbf{以大写开始},其余字母均为小写
    \item \texttt{istitle()} 若x标题化,返回 \texttt{True},否则返 \texttt{False}
\end{itemize}

\subsection{字符类型判断}
\begin{itemize}
    \item \texttt{isalpha()} \textbf{只包含字母},且非空,返回 \texttt{True}
    \item \texttt{isalnum()} \textbf{只含字母和数字},且非空,返回 \texttt{True}
    \item \texttt{isdigit()} \textbf{只包含数字}字符,且非空,返回 \texttt{True}
    \item \texttt{isspace()} \textbf{只包含空格、制表符和换行},且非空
\end{itemize}

\subsection{字符串基本方法}
\begin{itemize}
    \item \textcolor{red}{\texttt{str.split(X)}} 返回\textbf{按照X对str进行切割}列表
    \begin{quote}
        \texttt{"A,B,C".split(",")} $\rightarrow$ \texttt{['A','B','C']}
    \end{quote}
    \item \textcolor{red}{\texttt{str.splitlines()}} 按照\textbf{换行符}来切分字符串
    \item \textcolor{red}{\texttt{str.count(X)}} 输出字串X在str中\textbf{出现的次数}
    \item \textcolor{red}{\texttt{str.replace(老,新)}} \textbf{所有old字串被替换为new},替换每一个。此处被视为字串对待
    \begin{quote}
        \texttt{"aa bb cc dd".replace(" ",",")}\\
        \texttt{"aa,bb,cc,dd"}
    \end{quote}
    \item \textcolor{red}{\texttt{str.strip(chars)}} 从字符串\textbf{左右两边同时去掉}chars列出的字符。当检测到不在chars中的字符时即停,忽略中间的chars。此处为chars含有的每一个字符。
    \begin{quote}
        \texttt{"aabbccddaabbccdd".strip("adc")}\\
        \texttt{'bbccddaabb'}
    \end{quote}
    \item \texttt{str.lstrip()} 从左侧去掉列出的字符
    \item \texttt{str.rstrip()} 从右侧去掉列出的字符
    \item \textcolor{red}{\texttt{str.join(iter)}} 将iter变量除最后一个元素外\textbf{每个元素后增加一个str}
    \begin{quote}
        \texttt{",".join("12345")} \texttt{'1,2,3,4,5'}
    \end{quote}
\end{itemize}

\subsection{字符串查找}
\begin{itemize}
    \item \textcolor{red}{\texttt{string.find(str,beg=0,end=len(string))}}\\
    检查str是否包含在\textbf{string}中,如果\textbf{beg}和\textbf{end}指定范围,则检查是否包含在指定范围内,如果是则返回开始的索引值,否则返回-1
    \item \texttt{string.rfind(str)} 从右往左找
\end{itemize}

\section{列表及其基本操作}
\textbf{使用方括号} \texttt{列表名=[1,2,3......]} 其中甚至可以放任何支持迭代的容器或对象,例如:
\begin{quote}
    \texttt{week1=['a','b'] week2=[x for x in week1]}
\end{quote}

\subsection{添加元素}
\begin{itemize}
    \item \textcolor{red}{\texttt{append(元素)}} \textbf{追加} 将一个新的元素添加到列表的尾部,\textbf{直接修改,没有返回值}
    \item \textcolor{red}{\texttt{insert(位置,元素)}} \textbf{插入} 用法:\texttt{insert(0,"a")}在0位置添加一个元素,其余后移一位
    \item \textcolor{red}{\texttt{extend()}} \textbf{扩展} 列表,相当于连接符+\\
    例如 \texttt{week1=[a]; week2=[b];}\\
    \texttt{week1.extend(week2)} 则修改 \texttt{week1}
\end{itemize}

\subsection{删除元素}
\begin{itemize}
    \item \textcolor{red}{\texttt{Remove()}} \textbf{移除} 其根据一个\textbf{元素的内容}进行删除。若元素不在列表里,则会异常报错。
    \begin{quote}
        为避免错误可提前进行判断,如\\
        \texttt{if(item in week): week.remove(item)}\\
        \texttt{week.remove('星期六')} 删除星期六\\
        \textbf{从左往右删除,仅删除第一个}
    \end{quote}
    \item \textcolor{red}{\texttt{Pop()}} \textbf{弹出} 根据\textbf{下标}删除单个或多个函数。
    \begin{quote}
        pop方法有返回值,会返回被删除的数据,\textbf{同时更改原始数据}(这是两个操作)\\
        例如 \texttt{item=week.pop(2)} 其删除元素后,能够返回被删除的元素\\
        括号里可不带值,删除最后一个元素
    \end{quote}
\end{itemize}

\subsection{列表查询}
\begin{itemize}
    \item \texttt{count.(x)} 统计x出现次数
    \item \texttt{index.(x,a,b)} 返回从开始位a开始到结束位b结束的x的具体位置。有异常错误,需确保存在性
\end{itemize}

\subsection{列表复制}
\begin{itemize}
    \item \textbf{错误示例} \texttt{week1=[] week2=week1}
    \begin{quote}
        若运行上述语句,若修改week1,则week2\textbf{同步变化}。数据在内存中仅有一份
    \end{quote}
    \item \textbf{切片法} \texttt{week2=week1[:]} 实现部分数据复制
    \item \textcolor{red}{\texttt{a.copy()}} 方便的复制方法。\texttt{week2=week1.copy()}
    \begin{quote}
        返回\textbf{浅拷贝}:\texttt{[a,[ ],c]} 这类列表无法通过 \texttt{[:]}、\texttt{copy} 进行复制,减少内存占用
    \end{quote}
    \item \textbf{深拷贝}:需要调用包:\texttt{deepcopy}
\end{itemize}

\subsection{列表排序}
\begin{itemize}
    \item \textcolor{red}{\texttt{sort()}} \textbf{方法} 例如 \texttt{list1.sort()} 可直接使用。内存地址不变,其\textbf{修改原始列表,不返回排序结果}。
    \begin{quote}
        所以不要写成 \texttt{print(list.sort())}。排序默认\textbf{升序},使用 \texttt{sort(reverse=True)} 降序排序,或使用 \texttt{[::-1]}
    \end{quote}
    \item \textcolor{red}{\texttt{sorted()}} \textbf{函数} 其不改变原有列表,\textbf{返回新的排序后的列表}\\
    例如 \texttt{list2=sorted(list1)}
    \item \texttt{reverse()} \textbf{逆序} 不排序,和 \texttt{[::-1]} 效果一致
    \begin{quote}
        数据量大的时候不要用,直接调用 \texttt{reverse}
    \end{quote}
    \item \textbf{索引方法} 形如 \texttt{lst[1][0][2]} 同样需要注意越界问题
\end{itemize}

\chapter{程序的控制结构}
\section{分支结构}
\subsection{双分支结构}
\begin{itemize}
    \item \textbf{紧凑格式} \texttt{<表达式1>if<条件>else<表达式2>}\\
    \texttt{print('猜\{\}了'.format('对'if a==9 else '错'))}
\end{itemize}

\subsection{条件组合}
\begin{itemize}
    \item \textbf{优化原则} 
    \begin{enumerate}
        \item \textbf{与或非运算}时系统会有优化,如果and前已经算出False了,那么结果就是False,后面的表达式计算机不会进行计算。
        \item 如果or前已经算出True,那么结果就是True,后面的表达式不计算
    \end{enumerate}
    \item \textbf{注意}
    \begin{enumerate}
        \item 0,None为假,\textbf{空对象为假}
        \begin{quote}
            空对象包括 \texttt{""}, \texttt{()}, \texttt{[]}, \texttt{\{\}}
        \end{quote}
        \item \textbf{浮点数相减判断}两者差值小于一个足够小的数 \texttt{if abs(a-b)<=1e-8:}
        \item 使用 \texttt{pass} 可以跳过当前分支
    \end{enumerate}
\end{itemize}

\section{异常处理}
\textbf{高级使用方式}
\begin{lstlisting}[language=Python]
try:
    <1> 正常执行的内容
except <异常类型>:
    <2>
else:
    <3> 没有异常时执行
finally:
    <4> 这一部分一定会被执行
\end{lstlisting}
常用于文件读写,并发任务时的资源访问等场景。\\
例如,完成读写文件后,要求关闭该文件。\\
若无错误,则执行1→3→4,若报错,则执行1→2→4

\section{循环结构}
\begin{itemize}
    \item \textbf{遍历循环} \texttt{For <循环变量> in <遍历结构>:}\\
    \texttt{<语句块>}
    \begin{quote}
        从结构里面按顺序取一个值赋给循环变量,语句的循环次数是固定的。
    \end{quote}
    \item \textbf{计数循环 循环特定次}\\
    \texttt{for i in range(N):}\\
    \texttt{<语句块>}
    \begin{quote}
        \textbf{共循环N次,i第一次为0,最后为N-1}
    \end{quote}
    \item \texttt{range(M,N,K)} 从M到N-1,步长为K
    \item \texttt{range(M)} 从0到M-1,一共M个数
    \item \texttt{range(M,N)} 从M到N-1
    \item \textbf{无限循环} \texttt{while <条件>: <语句块>}\\
    反复执行语句块,直到条件不满足
    \item \textbf{循环控制保留字}
    \begin{itemize}
        \item \texttt{break} 跳出并结束这个循环
        \item \texttt{continue} 结束当前循环,继续执行后续次数循环
    \end{itemize}
    \item \textbf{双重循环}
    \begin{lstlisting}[language=Python]
s="PYTHON"
while s!='':
    for c in s:
        print(c,end='')
    s=s[:-1]   #每次丢掉最后一个符号
    \end{lstlisting}
    得到 \texttt{PYTHONPYTHOPYTHPYTPYP}
\end{itemize}

\chapter{元组 集合 字典}
\section{元组 tuple ( )}
\textbf{特点} 视作为只能读取数据,\textbf{不能修改数据的列表}
\begin{quote}
    元组在处理大批量数据时,内存占用相比列表要小,且查找速度更快。\\
    元组的括号可以省略 例如 \texttt{tup="1",2,True}\\
    \texttt{x1,x2,x3=tup} 自动分配tup中的内容给x1,x2,x3,但需要注意数量左右一一对应
\end{quote}
\begin{itemize}
    \item \textbf{空元组} \texttt{tup=()}
    \begin{quote}
        值得注意的是,若元组中仅有一个元素,需要在这个元素后\textbf{增加一个逗号}\\
        例如 \texttt{tup1=(50)} 和 \texttt{tup2=(50,)} 其类型不同
    \end{quote}
    \item \textbf{访问元组} 按照下标或切片操作
    \item \textbf{修改元组} \textbf{元组只读}。无法通过下标赋值,没有replace等方法。\\
    但其可以连接,例如 \texttt{tup1+ tup2} 得到一个新的元组;\texttt{tup1*3} 得到新的元组
    \item \textbf{删除元组} 无法实现删除某个元素,可以用 \texttt{del} 删除整个元组
    \item \textbf{相关函数} \texttt{len()} \texttt{max()} \texttt{min()}
    \item \texttt{tuple()} 将某个东西强行转换为元组
    \item \textbf{封装和解封} 自动加圆括号,自动去圆括号。\\
    \texttt{tup="1",2,True}
\end{itemize}

\section{集合 set( )}
\subsection{基本内容}
\begin{itemize}
    \item \textbf{创建集合} \textbf{仅使用 \texttt{set()}}\\
    \texttt{num=set(["a","b","c"])}\\
    \texttt{num=set("123456")}\\
    \texttt{>>> \{'1','4','6','2','3','5'\}}
    \item \textbf{特征} 
    \begin{enumerate}
        \item 自动去除所有的重复元素,每个集合\textbf{元素唯一}。因此可以\textbf{用其去除重复值}\\
        \texttt{lst=list(set(lst))} 注意,顺序可能改变
        \item 集合内的元素是\textbf{无序}的,集合中元素的顺序可能与输入的顺序不同。因此\underline{不应该}使用下标索引集合。
        \item 集合中可以包含类型不同的元素
        \item \textbf{非空集合的判断}使用len,长度为零
    \end{enumerate}
    \item \textbf{运算符} 设 \texttt{a=set("abcd") b=set("bcd")}
    \begin{itemize}
        \item \texttt{a-b} 差:a去掉ab的交集
        \item \textcolor{red}{\texttt{a|b}} 交集
        \item \textcolor{red}{\texttt{a\&b}} 并集
        \item \texttt{a \textasciicircum b} \textbf{对称差集} 并集-交集(\textbf{不同时包含}的元素)
        \item \texttt{in / not in} 元素包含判断
        \item \texttt{==} 判断两个集合是否全等
        \item \texttt{!=} 判断两个集合不等
        \item \texttt{> <} 检查包含关系(父级判断函数),此处为真子集
        \item \texttt{>=} 子集
    \end{itemize}
\end{itemize}

\subsection{基本函数}
\begin{itemize}
    \item \texttt{len min max sum} 等
    \item \textcolor{red}{\texttt{n.add(x)}} \textbf{添加函数},向集合n中添加x。若元素已存在,则不进行操作。
    \item \textcolor{red}{\texttt{n.update(x)}} 添加元素,参数可以是列表、元组、字典、集合等组合数学类型\\
    其不能直接添加单个值
    \item \textcolor{red}{\texttt{n.discard(x)}} \textbf{移除元素},其不会报错。\textbf{其没有返回值}
    \item \textcolor{red}{\texttt{n.pop()}} 随机地删除一个集合元素。
    \begin{quote}
        (实质上是删除第一个元素)
    \end{quote}
    \item \texttt{n.clear()} 清除集合中全部元素
\end{itemize}

\section{字典 \{ \}}
字典由一系列\textbf{键-值}组成,每个键与一个值相关联,使用键来访问与之相关联的值。关联值可以是数字、字符串、列表甚至字典。\\
不要使用 \texttt{dict} 作为变量名。

\subsection{字典创建}
\textbf{方法 字典名}=\{\textbf{键1:值1}, \textbf{键2:值2},............\}
\begin{quote}
    \texttt{fruit\_num=\{"apple":1,"orange":2,"banan":3\}}
\end{quote}
\textbf{注意} 
\begin{enumerate}
    \item 同一字典中,键和值的类型\textbf{均可以不同}。
    \begin{quote}
        \texttt{any\_type=\{"a":[1,2,3],2:8,True:\{"hello"\}\}}
    \end{quote}
    \item 键和值均可以重复,但键重复,则仅最后一个有效。关键字必须唯一,且可哈希,可散列(例如列表)
    \item \textbf{键必须不可变}。如 \texttt{["abc"]:2}、\texttt{\{1:2\}:5}、\texttt{(2,\{\}):6} 均错误,其包含有列表、字典等可变类型
\end{enumerate}

\subsection{元素读取}
\begin{itemize}
    \item \textbf{方法1 值=字典名[key]}
    \begin{quote}
        使用 \texttt{fruit\_num["apple"]} 获取值1
    \end{quote}
    若键不存在,\textbf{则报错}。
    \item \textbf{方法2} \textcolor{red}{\texttt{值=字典名.get(key,default)}}
    \begin{quote}
        \textbf{default}为\textbf{key}不存在时返回的默认值\\
        例如 \texttt{fruit\_num.get("pear",-1) >>> -1}
    \end{quote}
\end{itemize}

\subsection{元素新增修改}
\begin{itemize}
    \item \textbf{方法1} \textcolor{red}{\texttt{字典名[key]=新值}}
    \begin{quote}
        若\textbf{key}不存在,则直接添加新纪录。\\
        \textbf{等同于新增记录}。例如 \texttt{fruit\_num["pear"]=4}
    \end{quote}
    \item \textbf{方法2} \texttt{x1.update(x2)} 合并字典,如果关键字重复,则更新为x2中的新数据
\end{itemize}

\subsection{元素删除}
\begin{itemize}
    \item \textcolor{red}{\texttt{字典名.pop(key)}} 根据关键字删除,同时删除键值。其\textbf{返回}被删除的键的对应值。
    \item \texttt{x.popitem()} 无需给\textbf{key},删除固定的最后一个。不确定被删元素,慎用。
\end{itemize}

\subsection{列表遍历}
\begin{itemize}
    \item \textcolor{red}{\texttt{n.items()}} 所有的元素,\textbf{同时给出键值}。可以被转换为列表类型
    \begin{quote}
        \texttt{dict\_items([('chinese', 80), ('python', 100), ('math', 59)])}
    \end{quote}
    \item \textcolor{red}{\texttt{n.keys()}} 所有的\textbf{关键字},仅给出\textbf{keys}\\
    \texttt{dict\_keys(['chinese', 'python', 'math'])}
    \item \textcolor{red}{\texttt{n.values()}} 所有的\textbf{数值},仅给出\textbf{values}
    \begin{quote}
        \texttt{dict\_values([80, 100, 59])}
    \end{quote}
    \item \textcolor{red}{\texttt{for key,item in n.items():}} 用于遍历字典\textbf{n}里面的\textbf{key}和\textbf{item}
\end{itemize}

\subsection{字典拷贝}
\begin{itemize}
    \item \texttt{copy()} 浅拷贝,例如 \texttt{b=a.copy()}\\
    如果字典里面的值带有列表,会存在问题
    \item \texttt{a=b} 直接赋值,即浅拷贝,引用对象(\textbf{同步修改})
\end{itemize}

\subsection{字典的特性}
\begin{enumerate}
    \item 列表动态数组,可变重设长度。元组静态数组不可变
    \item 元组缓存于Python运行环境,每次使用无需访问内核分配内存
    \item 字典的查找和插入\textbf{速度极快},不会随着\textbf{key}的增加而变慢,但内存占用很大;列表查找插入时间随着元素的增加而增加,但占用空间小,内存少
    \item 字典的\textbf{key}是不可变对象。
\end{enumerate}

\subsection{列表装配拆分}
假设有两个列表:
\begin{lstlisting}
>>> name=['zhang','zhao','wang']
>>> grades=[70,80,45]
\end{lstlisting}
装配起来使用\textbf{zip函数}:
\begin{lstlisting}
>>> list(zip(name,grades))
[('zhang', 70), ('zhao', 80), ('wang', 45)]
\end{lstlisting}
假设拿到一个组合的列表:
\begin{lstlisting}
new=[('zhang', 70), ('zhao', 80), ('wang', 45)]
\end{lstlisting}
仍然使用\textbf{zip}拆除,\textcolor{red}{变量名前加*}:
\begin{lstlisting}
>>> list(zip(*new))
[('zhang', 'zhao', 'wang'), (70, 80, 45)]
>>> a,b=zip(*new) 同时使用两个变量接收返回值
>>> a→  ('zhang', 'zhao', 'wang')
>>> b→  (70, 80, 45)
\end{lstlisting}
注:zip只能使用一次,用完即释放原有内存。\\
\textbf{枚举} 自动给元素序号,并以元组形式返回,当我们用下标时,可以以此查询:
\begin{lstlisting}
>>> for i,v in enumerate(a):    print(i,v)
>>> 0 zhang    1 zhao     2 wang
\end{lstlisting}

\chapter{函数}
\section{函数 Function}
\textbf{定义} 函数是一段功能模块,将常用的功能编写成函数利于代码复用。\\
\textbf{程序设计思想:}
\begin{enumerate}
    \item 自顶向下逐步求精,顶层设计善于抽象(先不考虑细节,考虑总体功能分区实现),再进一步拆分,分而治之。
    \item 模块化,结构化→函数;面向对象→类(变量+函数)
\end{enumerate}

\subsection{定义函数}
\begin{lstlisting}[language=Python]
def 函数名([传入参数变量列表]):  # []表示可选项
    函数体
    [return [返回值表达式]]
\end{lstlisting}
同时表示函数结束,若return无返回值,则返回空。\\
\textbf{实参} 调用时给的\textbf{实际参数}\\
\textbf{形参} 指的是定义时写的\textbf{形式参数}\\
\textbf{函数调用 函数名(参数)}
\begin{quote}
    函数的定义必须出现在函数调用之前
\end{quote}

\subsection{函数参数}
\begin{enumerate}
    \item \textbf{位置参数}\\
    \textbf{规则} 调用时与声明时一致,位置不能变更。形参和实参个数必须完全一致(不可省略),形参实参类型必须完全一致,否则均会报错\\
    \textbf{示例} \texttt{def sum(begin,end,model):}\\
    \texttt{sum(1,10,0)}
    \item \textbf{关键字参数}\\
    \textbf{规则} 使用关键字参数传参,以 \texttt{name=value} 形式传递\\
    \textbf{示例} \texttt{def sum(begin,end,model):}\\
    \texttt{sum(begin=1,end=10,model=0)}\\
    \textbf{注意} 可以将关键字与位置参数混用,关键字参数必须在位置参数\textbf{之后}
    \begin{quote}
        \texttt{sum(20,end=40,model=0)} 可行\\
        \texttt{sum(end=40,start=0,20)} 会报错
    \end{quote}
    \item \textbf{默认参数}\\
    \textbf{规则} 编写函数时可以给每个形参提供默认值。\\
    \textbf{示例} \texttt{def sum(begin,end,model=0):}\\
    \texttt{sum(1,10)} 可省略最后一个\\
    \textbf{注意} 形参书写中必须先列出没有默认值的参数
    \begin{quote}
        \texttt{def sum(begin=35,end,model=0):}\\
        \texttt{sum(50)} 直接报错
    \end{quote}
    \item \textbf{不定长参数}\\
    \textbf{规则} 一个函数能够处理比声明时更多的参数,这些参数叫做不定长参数。
    \begin{lstlisting}[language=Python]
def 函数名([参数列表, ]*不定长参数,**不定长参数2):
    \end{lstlisting}
    \textbf{示例}
    \begin{lstlisting}[language=Python]
def print_info(arg1,*arg2):
    "打印任何传入的参数"
    print(arg1)
    for var in arg2:
        print(var)
    return
    \end{lstlisting}
    调用:\texttt{print\_info(10,20,30)}
\end{enumerate}

\subsection{函数返回}
\begin{itemize}
    \item \textbf{函数的返回值} Python在形式上允许返回多个值
    \item \textbf{返回一个值} \texttt{return A}
    \item \textbf{返回空值} \texttt{return} 等价于 \texttt{return None}
    \item \textbf{返回多个值} \texttt{return A,B}
    \item \textbf{返回大量数据} 可利用列表、字典、元组等返回数据\\
    例如: \texttt{result.append(s)}\\
    \texttt{return result}
\end{itemize}

\subsection{其他}
\begin{itemize}
    \item \textbf{嵌套调用} 在一个函数里面调用其他函数
    \begin{quote}
        通常程序定义一个包含程序主要功能的main的函数,使程序更规范易读。
    \end{quote}
    \textbf{示例}
    \begin{lstlisting}[language=Python]
def sum(begin,end,model):
    sum=0
    for i in range(begin,end):
        if i%2==model:
            sum=sum+i
    return sum

def main():    
    s=sum(1,10,0)
    print(s)

main()  # 函数调用语句
    \end{lstlisting}
    \item \textbf{调用栈} 原理:开辟内存栈空间,main在底层,sum1在顶部,后入先出。
    \item \textbf{递归调用} 自己调用自己
    \begin{quote}
        要求:先讨论分支、讨论清晰\textbf{结束递归的边界条件},再找到需要递归的条件
    \end{quote}
    \textbf{示例} 阶乘实现:
    \begin{lstlisting}[language=Python]
def fact(n):
    if n==1 or n==0:  #边界条件
        return 1
    return n*fact(n-1)
    \end{lstlisting}
\end{itemize}

\section{变量作用域}
\subsection{作用域类型}
\begin{itemize}
    \item \textbf{局部作用域(Local)} 包含在def关键字定义的语句块中,不同函数内部的变量名与函数外的变量名为不同。函数为一个独立的内存空间。\\
    称为\textbf{局部变量}。
    \item \textbf{嵌套作用域(Enclosing)} 函数中嵌套函数,上级影响下级
    \item \textbf{全局作用域(Global)} 作用仅限于单个模块文件内
    \item \textbf{内置作用域(Built-in)} 系统内固定模块里定义的变量,如math里的pai
    \item \textbf{原则} 局部变量不能扩散到全局
    \begin{quote}
        全局变量能够渗透到全部
    \end{quote}
    \item \textbf{示例}
    \begin{lstlisting}[language=Python]
var=100  #全局作用域,覆盖全部范围
def fun1():
    var=200 #外部嵌套函数作用域
    print("var in fun1:",var)
    def fun2():  #嵌套定义
        var=300  #局部作用域
        # 只要在函数内赋值,就是局部
        print("var in fun2:",var)
        print("max in fun2",max)  
        # max在内建函数作用域中,只读不能改变
    fun2()
fun1()
print("var in global",var)
    \end{lstlisting}
    输出:
    \begin{lstlisting}
>>> var in fun1: 200
>>> var in fun2: 300
>>> max in fun2 <built-in function max>
>>> var in global 100
    \end{lstlisting}
\end{itemize}

\subsection{作用域优先级}
\textbf{局部作用域 > 嵌套作用域 > 全局作用域 > 内置作用域}\\
\textbf{LEBG}\\
\textbf{错误案例}
\begin{lstlisting}[language=Python]
var=100
def fun():
    print(var)
    var=200  # 在此赋值导致变量类型无法确认
fun()
\end{lstlisting}
\textbf{global语句} 如果想要在函数内对全局变量赋值,需要使用\textbf{global}语句,例如:
\begin{lstlisting}[language=Python]
var=100
def fun():
    global var
    print(var)
    var=200
fun()
\end{lstlisting}
在软件工程角度,不推荐直接修改全局变量。可以使用如下方法:
\begin{lstlisting}[language=Python]
var=100
def fun(var):
    print(var)
    var=200  # 这里的修改不会改动外界的var
fun()
\end{lstlisting}

\chapter{文件和数据库}
\section{文件概述}
\begin{itemize}
    \item \textbf{文件路径} 关键属性:路径、文件名
    \item \textbf{根文件夹} Windows中为 \texttt{D:\textbackslash\textbackslash};Linux中为 \texttt{/} 新加卷在 \texttt{/mnt} 文件夹下
    \item \textbf{绝对路径} 总是从根文件夹开始
    \item \textbf{相对路径} 指文件相对于当前工作目录所在的位置
    \item \textbf{示例} 工作目录为 \texttt{"C:\textbackslash Windows\textbackslash System32"}
    \begin{quote}
        若 \texttt{demo.txt} 位于这个System32文件夹下,则可表示为 \texttt{".\textbackslash demo.txt"}
    \end{quote}
    \item \textbf{使用} \texttt{.\textbackslash} 表示当前目录;\texttt{..\textbackslash} 表示所在目录的父级目录
    \item \textbf{字符编码} 汉字一般使用GBK,\textbf{UTF-8}(世界通用)
\end{itemize}

\section{文件操作}
\subsection{文件的打开}
\begin{lstlisting}[language=Python]
file = open(file_name,mode="r",buffering=-1,encoding=None)
\end{lstlisting}
\begin{itemize}
    \item \textbf{mode} \textcolor{red}{r只读(默认)} \textbf{w覆盖} \textcolor{red}{a写入追加} \textbf{b}以二进制格式操作\\
    \texttt{r+} 从开头写入新的数据,会覆盖原有内容\\
    \texttt{rb} 以二进制只读格式打开文件\\
    \texttt{rb+} 通常针对非文本文件\\
    \texttt{w} 打开时即清空文件中原有内容\\
    \texttt{w+} 对该文件拥有读写权限
    \item \textbf{buffering} 表示是否使用缓冲区
    \item \textbf{注意} 该函数仅获得文件句柄,即文件处理权限没有进行读数据的操作,未复制到内存
    \item \textbf{示例} 例如 \texttt{file=open("./a.txt") print(file)}\\
    输出相关文件信息:\\
    \texttt{<\_io.TextIOWrapper name='a.txt' mode='r' encoding='cp936'>}
\end{itemize}

\subsection{文件对象的属性}
\begin{itemize}
    \item \texttt{file.name} 返回文件名称
    \item \texttt{file.mode} 返回文件打开的模式
    \item \texttt{file.encoding} 返回文件使用的编码格式
    \item \texttt{file.closed} 判断文件是否已关闭
\end{itemize}

\subsection{文件的关闭}
\begin{itemize}
    \item \texttt{file.close()} 使用open()打开的文件对象,\textbf{必须手动进行关闭}
    \item \textbf{注意} 在向文本格式打开的文件中写入数据时,只有使用close()函数关闭文件时,才会将缓冲区中的数据真正写入文件中。写数据时,会先临时存储到缓冲区。
    \begin{quote}
        可以使用 \textcolor{red}{\texttt{file.flush()}} 方法可强制存盘
    \end{quote}
    \item \texttt{with open() as f:} 使用此法可在with as语句执行完毕后自动关闭已打开的文件
\end{itemize}

\subsection{读文件}
\begin{itemize}
    \item \texttt{read()} \textbf{逐个字节}或字符读取文件内容。
    \begin{quote}
        将文件的全部内容读出\\
        可使用size参数设置每次读取的最大字节\\
        \texttt{f.read(6)} 读取前6个字符
    \end{quote}
    \item \texttt{readline()} \textbf{逐行读取}文件中内容。以 \texttt{\textbackslash n} 作为读取一行的标志,会包含最后的换行符 \texttt{\textbackslash n},再加上print输出内容时默认会换行
    \begin{quote}
        \texttt{f.readline([size])} 其中size指读取每一行时,一次最多读取的字符数
    \end{quote}
    \item \texttt{readlines()} 一次性\textbf{读取多行内容},返回一个字符串列表,每个元素为每行内容(包含结尾换行符)
    \begin{quote}
        不可读取全部文件,最后会报错。\\
        可用于读完一行打印一行:
    \end{quote}
    \item 读取特定行数的函数:
    \begin{lstlisting}[language=Python]
def realine(filename,lines,encode="utf-8"):
    with open(filename,encoding=encode) as file:
        count=0
        for line in file:
            newline=line.strip()
            if len(newline)==0 or newline[0]=="#":
                continue
            print(newline)
            count+=1
            if count>=lines:
                break
realine("a.txt",5)
    \end{lstlisting}
    \item \textbf{收集操作系统的系统配置信息} psutil模块
\end{itemize}

\subsection{写文件}
\begin{itemize}
    \item \texttt{write()} 格式:\textbf{file.write(string)}
    \begin{quote}
        必须保证文件打开模式支持写入。写入数据时不会换行,需要手动输入换行符。\\
        例如:\texttt{f=open('a.txt','a')}\\
        \texttt{f.write('\textbackslash n新数据')} \texttt{f.close()}
    \end{quote}
    \item \texttt{writelines()} 将\textbf{字符串列表}写入到文件中
    \begin{quote}
        可以自动换行\\
        可轻松地将A文件的数据复制到B文件中:
        \begin{lstlisting}[language=Python]
f=open('A.txt','r')
n=open('B.txt','w+')
n.writelines(f.readlines())
f.close()       
n.close()
f.writelines(["123","456"])
        \end{lstlisting}
    \end{quote}
\end{itemize}

\subsection{文件指针 pointer}
\begin{itemize}
    \item \textbf{概念} 用于标明文件读写的起始位置。可实现读取文件中指定位置的数据。
    \item \texttt{tell()} 判断当前文件指针所在的位置。
    \begin{quote}
        若全为字节,则按照字节;若中英文混合,则按照字符来。\\
        格式:\textbf{file.tell()} 初始为0\\
        例如:读取三个字节 \texttt{f.read(3)} 此时若继续 \texttt{f.read()} 则输出未被读过的所有数据
    \end{quote}
    \item \texttt{seek()} 移动文件指针到指定位置\\
    格式:\textbf{file.seek(offset[,whence])}\\
    \textbf{whence} 可选,指定文件指针要放置的位置
    \begin{quote}
        \textbf{0文件头(默认)},1当前位置,2文件尾
    \end{quote}
    \textbf{offset} 表示相对于whence位置的偏移量(字节数),正为向后,负为向前\\
    \texttt{seek(3,0)} 移动到距开头处3个字节的位置\\
    \texttt{seek(5,1)} 移动到距当前位置5字节后
    \begin{quote}
        当offset非0时,要求文件\textbf{必须以二进制格式打开},否则报错。即使用 \texttt{'rb'}、\texttt{'ab'}、\texttt{'wb'}
    \end{quote}
    \item \textbf{示例程序}:输出最后5行数据
    \begin{lstlisting}[language=Python]
with open("D:/Temp/test.txt",mode="rb") as f:
    f.seek(-150,2)  # 设置一个大的数字
    for row in f.readlines()[-5:]:  # 切片最后五行
        print(row.strip().decode("utf-8"))   
        # 去掉空格换行等后把编码改变
    \end{lstlisting}
\end{itemize}

\subsection{文件系统相关命令}
\begin{itemize}
    \item \textbf{路径拼接} \texttt{os.path.join(path,name)} \texttt{os.sep.join(path,name)}
    \begin{quote}
        其可以自动拼接并返回文件路径字符串,并包含正确的路径分隔符\\
        \texttt{import os}\\
        \texttt{os.path.join("demo","path.txt")}
    \end{quote}
    \item \textbf{os中的一些小方法}
    \begin{itemize}
        \item \texttt{os.getcwd()} 获取当前工作路径的字符串
        \item \texttt{os.chdir("path")} 可改变目录
        \item \texttt{os.name} 获取os名称
        \item \texttt{os.system()} 执行shell命令
        \item \texttt{os.getenv()} 获取环境变量,例如 \texttt{os.getenv("path")}
        \item \texttt{os.remove()} 移动文件
        \item \texttt{os.path.getsize(path)} 返回文件大小、若文件不存在报错
        \item \texttt{os.path.exists(path)} 判断文件是否存在
        \item \texttt{os.path.isfile(path)} 判断是否为文件
        \item \texttt{os.path.split()} 路径拆分为路径名和文件名(以元组形式)
        \item \texttt{os.walk(path)} 遍历文件路径,返回三元组 \texttt{(root,dirs,files)}
    \end{itemize}
    \item \textbf{sys模块与提取命令行参数}\\
    可提供操作系统的属性信息\\
    \texttt{sys.version} 获取python解释程序的版本信息\\
    \texttt{sys.platform} 提供操作系统信息
\end{itemize}

\chapter{面向对象程序设计}
\textbf{结构化程序设计} \textbf{自顶向下逐步求精},将各种功能分解出来\\
把复杂问题逐步分解为易于理解实现的一系列功能模块\textbf{【分治】},主要由函数实现。\\
Google:map reduce架构/ Python中有 \texttt{map()} \texttt{reduce()} 函数实现数据并行计算\\
\textbf{面向对象程序设计}\\
\textbf{封装:}把方法、属性、事件集中到一个\textbf{统一的类中},对使用者屏蔽其中细节(有了外壳)\\
\textbf{继承:}在被继承的类(父类、基类、超类)基础上,进行扩展生成新的类-\textbf{子类}。\\
\textbf{多态性:}同样的函数对于不同的对象具有不同的实现。\\
例如+号,对 \texttt{3+4}、\texttt{"3"+"4"} 有不同的实现\\
\textbf{优势:} 提高代码复用性、使编码灵活可维护、提高可扩展性、提高开发效率

\section{类 Class}
包含有\textbf{属性}(→变量)\textbf{、方法}(→函数)\\
\textbf{概念} 描述具有相同属性和方法的\textbf{对象的集合},因此是抽象的。定义了该集合中每个对象所共有的属性和方法。\\
例如类:学生,每个学生都有姓名、性别、学号等\\
\textbf{实例化 instance} 创建一个类的\textbf{具体}实例-\textbf{类对象}。变化的这个动作叫做实例化。变化的结果叫对象,故对象有时也称为实例。一个类可以有众多的实例。\\
\textbf{方法} 类中定义的函数\\
\textbf{属性} 类变量,成员变量。\\
\textbf{定义语法} \texttt{class Name():} 一般括号内为空,若存在object,表示从object中继承而来\\
\textbf{初始化语法} \texttt{def \_\_init\_\_(self,x,y,z,......):} 也称为x,y,z,为希望传参进入的参量\\
第一个参数为self,无需传参,表示对象自己\\
\textbf{析构方法} \texttt{def \_\_del\_\_(self):} 用于销毁对象本身,释放对象所用的内存
\begin{quote}
    python有优秀的内存回收机制,一般不需要程序员重新定义或手动调用析构方法。当程序运行结束,python会自动触发 \texttt{\_\_del\_\_()} 方法
\end{quote}
\textbf{示例}
\begin{lstlisting}[language=Python]
class Myclass:
    var=1

# 一个关于圆形的类
class Circle(object):
    def __init__(self,radius):  # 初始化,一个构造方法__init__,用来构造对象;self就是对象自己
        self.radius=radius      # 对象属性,仅在类里面使用
    def area(self):             # 类方法,每个方法的第一个形参都是self
        return 3.14*(self.radius)**2    # 求圆面积

one_circle=Circle(2)            # 声明类对象,实例化
print(one_circle.area())
\end{lstlisting}

\section{对象 Object}
概述 在开发复杂庞大的程序时,需要有丰富的需求,同时要求开发简便易于维护,即智能化地实现。\\
\textbf{对象是一个整体,一切皆对象。具备了属性(n.)和功能(v.)的就是对象。}\\
变量类型、字符串、函数、面向对象也是对象。
\begin{quote}
    开发的开始为设计需求,先考虑属性与功能,随后再思考编程实现。此时设计程序不以变量来设计(零碎的)。在计算机中,object实现为class(类)。\\
    对象和类是一个对应的概念,\underline{类}是\textbf{抽象的},\underline{对象}是非常\textbf{具体的}。\\
    \texttt{"a".count("a")} 中的点·即代表调用对象的属性或是功能。如果有( )则为方法,没有括号则为属性。 例如 \texttt{arr.shape} 返回具体的属性值。
\end{quote}
\textbf{创建对象} 即类的实例化\\
\textbf{例:学生对象创建}
\begin{lstlisting}[language=Python]
class Student():     # 初始化代码不必要
    def set_id(self,x):  # 设置属性ID值,self表示对像自己,x是传进来的参数
        self.id=x    # 表示对象自身的ID,本质上类似变量,就是属性
    def show_id(self):
        print(self.id)

stu1=Student()   # 声明了具体的学生对象,对象实例化。即创建类对象
stu1.set_id(12345)       # 调用类里面的方法
stu1.show_id()

stu2=Student()
print(isinstance(stu2,Student))  # 判断stu2是否为Student的实例
\end{lstlisting}

\section{属性和方法}
\subsection{类属性和实例属性}
\begin{itemize}
    \item \textbf{类属性} 在类的名称下直接写变量,在各个方法之外。可以通过类的名称打点直接访问。\\
    例如 \texttt{Student.school}
    \item \textbf{对象属性} 形如 \texttt{self.aaa},可以在其他方法中直接使用(方法内共享)
    \begin{lstlisting}[language=Python]
class Student:
    school="某大学"     # 类属性
    def set_id(self,x):  # x就是普通局部变量
        self.id=x     # 对象属性,可以在其他方法中直接使用(方法内共享)
        y=0             # 一个方法内的局部变量
    def show_id():
        print(self.id)

Student.school()         # 调用类属性
stu3=Student()  
stu3.school()           # 可以通过实例访问类的属性
    \end{lstlisting}
\end{itemize}

\subsection{类的方法}
\textbf{对象方法 类对象使用的方法,}方法名的第一个形参是self\\
调用时由类对象来调用,\textbf{不可通过类名称直接调用(如 \texttt{Student.show\_id()} 报错)}\\
但可以 \texttt{Student.show\_id(stu5)} (传一个类对象)
\begin{lstlisting}[language=Python]
def show_id():               # 对象方法
    print(self.id)      # 可直接调用 Stu5.show_id() 使用类对象访问
\end{lstlisting}

\section{封装与访问控制}
\textbf{封装} 把数据和操作方法的细节隐藏,让调用的人方便调用。
\begin{quote}
    把一组有密切联系的数据和操作打包成一个不可分割的整体;对外只提供专门的接口(方法)供使用者访问部分数据和操作
\end{quote}
\textbf{访问控制}
\begin{itemize}
    \item \textbf{private 私有} 仅可通过方法对对象属性进行赋值,不对外公开【\textbf{命名时在名称左侧加双下划线}】
    \item \textbf{public 公有} 可直接赋值,完全公开【默认】
    \item \textbf{proteted 保护} 对外公开一部分【在左侧加一个下划线】,不允许使用import导入,即跨文件使用。
    \begin{lstlisting}[language=Python]
def set_id(self,x):  # x就是普通局部变量
    self.__id=x  # 创建私有对象
    self._id=x   # 创建保护对象
    y=0          # 一个方法内的局部变量
    \end{lstlisting}
\end{itemize}
还可以创建公有或私有方法,可使用公有方法调用私有方法。

\section{继承和多态}
\textbf{继承} 提高代码复用率。以原有的类(基类、父类)为基础,派生出新的类(派生类、子类)
\begin{quote}
    1. 子类继承了父类的数据和方法,可以定义新的方法,也允许重新定义一个旧的方法,为其增加新的特性。子类不能直接访问父类中的私有方法和私有属性。\\
    2. 派生类同样可以作为父类,再派生出新的子类,形成类的层次结构(UML图)。\\
    3. 如果不写继承,如 \texttt{class pa:},所有类似的均从\underline{抽象类object}继承,等价于 \texttt{class pa(object):}\\
    4. 构造方法内声明的变量\textbf{不会被自动继承}。继承方法:\textbf{super().\_\_init\_\_( )} 必须手动\textbf{显式地}调用父类构造方法。等价于 \texttt{Student.\_\_init\_\_(self,......)}
\end{quote}
\textbf{单继承} 一个派生类只有一个直接积累\\
\textbf{多继承} 一个派生类同时有多个基类(最好不要用多继承)\\
\textbf{语法结构} \texttt{class 子类名(父类名1,父类名2,......):}\\
\texttt{def \_\_init\_\_(self,父类参量,新参量):}\\
\texttt{supper().\_\_init\_\_(父类参量)}\\
\texttt{\# 类定义部分}\\
\textbf{覆盖} Override. 属于继承的一种。不改变名字,改变动作。
\begin{lstlisting}[language=Python]
class ParentClass:
    def show_info(self,info):
        print("A")

class ChildClass(ParentClass):
    def show_info(self,info):  
        print("B")
    def parent_show_info(self,info):    # 调用父类方法
        ParentClass.show_info(self,info)

obj=ChildClass()
obj.show_info()
obj.parent_show_info()
\end{lstlisting}
\textbf{多态}:根据外界的环境灵活变化自身功能

\chapter{模块和包}
\section{模块 modules}
\textbf{模块} 将代码存储为python文件,都称之为模块。每一个python脚本文件也为模块(a.py)\\
包含多个函数与类\\
\textbf{导入模块} \textbf{import}语句
\begin{quote}
    \textbf{模块导入仅发生一次,放在文件开头}\\
    \textbf{方法一} \textbf{import 模块名1 [as 别名1], 模块名2, [as 别名2], ......}\\
    导入全部成员,消耗内存大,调用成员时需要用模块名作为前缀,否则报错\\
    例如 \texttt{import math as m}\\
    \textbf{方法二} \textbf{from 模块名 import 成员名 [as 别名1], ......}\\
    只会导入模块中的指定成员,调用时无需附加前缀\\
    \texttt{from math import pi,e,sqrt}\\
    \texttt{from math import *} 可以导入math中全部成员,无需附加前缀了
\end{quote}
\textbf{查看模块} \texttt{dir()} 查看指定模块包含的全部成员

\section{包 package}
\textbf{包} 将多个功能模块存储在同一文件夹下,作为整体打包处理。文件夹下通常存在\textbf{\_\_init\_\_.py}的初始化文件(文件一般为空),作用为告诉python将此文件夹按包处理。\\
包名即为文件夹名\\
包含多个模块\\
\textbf{导入包} \textbf{from 文件夹名.模块名 import 成员名}\\
存在限制

\section{常用内置模块}
\begin{itemize}
    \item \texttt{math} 提供数学运算相关函数
    \item \texttt{random} 随机数函数
    \item \texttt{datetime} 提供日期和事件相关函数
    \item \texttt{logging} 提供日志相关函数
    \item \texttt{re} 提供正则表达式相关函数
    \item \texttt{os} 提供文件操作相关
    \item \texttt{sys} 操作系统相关信息函数
    \item \textcolor{red}{\texttt{time}} 提供时间戳,格式化时间
\end{itemize}

\section{math模块}
\begin{itemize}
    \item \textbf{常量} \texttt{pi} \texttt{e}
    \item \textbf{函数} \texttt{fabs} 返回绝对值
    \item \texttt{factorial(a)} 阶乘
    \item \texttt{gcd(a,b)} 最大公约数
    \item \textbf{幂函数} \texttt{log(a,b)} 以b为底的对数\\
    \texttt{log2(a)} 以2为底\\
    \texttt{log1p(a)} 以e为底\\
    \texttt{log10(a)} 以10为底\\
    \texttt{sqrt(a)} 平方根\\
    \texttt{pow(a,b)} a的b次幂
    \item \textbf{三角} \textcolor{red}{\texttt{degrees(a)}} 返回弧度a对应的角度\\
    \textcolor{red}{\texttt{radians(a)}} 返回角度a对应弧度
\end{itemize}

\section{random模块}
注意:生成的是伪随机数\\
\textbf{获取一个随机数}
\begin{itemize}
    \item \texttt{random()} 返回0-1之间的随机浮点数
    \item \texttt{randrange(start,stop,step)} 返回start到stop,步长为step的随机整数
    \item \textcolor{red}{\texttt{randint(a,b)}} 返回a和b之间的随机整数,一次一个\\
    随机数可以等于起始值,但不会等于终止值。
\end{itemize}
\textbf{获取一个随机序列}\\
更多的像统计中的\textbf{抽样}
\begin{itemize}
    \item \texttt{choice(seq)} 返回序列中随即一个数
    \item \texttt{choices(seq,k)} 随机选择k个序列中的数,得到一个新序列,可以重复(有放回)
    \item \texttt{sample(seq,k)} 随机选择k个序列中的数,得到一个新序列,不会重复(无放回)
    \item \texttt{shuffle(seq)} 随机打乱序列中的数
\end{itemize}

\section{datetime模块}
采用类似于字典的形式区分,\\
\textbf{类}
\begin{itemize}
    \item \texttt{date} 表示日期
    \item \texttt{time} 时间
    \item \texttt{datetime} 日期时间
    \item \texttt{timedelta} 时间跨度
    \item \texttt{tzinfo} 表示时区
\end{itemize}
\textbf{常量}
\begin{itemize}
    \item \texttt{MINYEAR} 最小允许年份,为1
    \item \texttt{MAXYEAR} 最大,9999
\end{itemize}
\textbf{方法}
\begin{itemize}
    \item \texttt{data(year,month,day)} 创建一个date对象
    \item \texttt{today()} 返回今天的date
    \item \texttt{now()} 返回详细的datetime对象
    \item \texttt{weekday()} 返回今天是周几
    \item \texttt{strgtime()} 格式化datetime对象
\end{itemize}
\textbf{格式化字符串}
\begin{itemize}
    \item \texttt{\%Y} 四位年
    \item \texttt{\%y} 两位年
    \item \texttt{\%m} 月
\end{itemize}

\section{time 模块}
\textcolor{red}{\texttt{time()}} 提取当前时间戳

\end{document}