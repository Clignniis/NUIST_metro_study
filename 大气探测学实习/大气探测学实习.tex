\documentclass[UTF8,11pt]{ctexbook}%后面几章有点问题,以后会修复
%数学符号
\usepackage{amsmath,amssymb,amsfonts}
\usepackage{tikz}
%合并单元格
\usepackage{multirow}
%表格复杂居中
\usepackage{array}
%删除线
\usepackage{cancel}
\usepackage{geometry}
%三线表
\usepackage{booktabs}
\usepackage[toc]{multitoc}%双栏目录
%颜色设置
\usepackage{xcolor}
\definecolor{nuist}{RGB}{0, 103, 156}%南信大蓝
\definecolor{sky}{RGB}{101, 170, 221}%天空蓝
\definecolor{tech}{RGB}{38, 96, 173}%科技蓝
\definecolor{gold}{RGB}{201, 160, 99}%高贵金
%引用
\usepackage[colorlinks]{hyperref}
%设置页眉页脚
\usepackage{fancyhdr}
\pagestyle{fancy}
\renewcommand{\sectionmark}[1]{\markright{\thesection\ #1}}
\fancyhf{}
\fancyfoot[CO]{\href{https://github.com/ZhangtongCN}{\textcolor{nuist}{\thepage}}}
\fancyfoot[CE]{\href{https://github.com/Clignniis}{\textcolor{nuist}{\thepage}}}
\renewcommand{\headrulewidth}{0pt}
%章节设置
\ctexset{
    chapter={pagestyle=fancy},%使得章节页页眉页脚格式一致
}
\geometry{a4paper,top=2.5cm,bottom=2.5cm}
\setcounter{tocdepth}{1}
\linespread{1.3}\selectfont

\begin{document}

\frontmatter

\newgeometry{scale=0.7}

\begin{titlepage}
    \begin{tikzpicture}[remember picture, overlay]
        \coordinate (A) at ([yshift=0.382\textheight]current page.south west);
        \coordinate (B) at (current page.north east);
        \fill[tech](A) rectangle (B);
        \node[xscale=-1] at (B){\includegraphics[scale=2]{1.eps}};
        \node[anchor= north east] at (B){\includegraphics[scale=2]{1.eps}};
        \node[anchor= north west] at (A){\includegraphics[scale=0.7]{2.eps}};
        \node at ([shift={(0,1)}]current page.center){
            \begin{tikzpicture}
                \foreach \a/\c in {0/90, 120/80, 240/70} {%
                    \path[fill=nuist!\c!sky] (0+\a :2.309)--++(150+\a :5)--++(270+\a :7)--++(150+\a :1)--++(90+\a :8)--++(330+\a :7)--cycle;
                }
            \end{tikzpicture}
        };
        \node[gold] at ([shift={(0,8)}]current page.center){\fontsize{72}{0}\heiti 大气探测学};
		\node[gold] at ([shift={(0,6)}]current page.center){\fontsize{40}{0}\heiti 实习};
        \node at ([shift={(0,-5)}]current page.center){\Huge\heiti\href{https://github.com/ZhangtongCN}{\textcolor{white}{Tong Zhang}}\quad\href{https://github.com/Clignniis}{\textcolor{white}{Cls}}};
        \node[tech, anchor= west] at ([shift={(-3,4)}]current page.south){\fontsize{48}{0}\heiti 笔记整理与汇总};
    \end{tikzpicture}
\end{titlepage}

\restoregeometry

\chapter*{版权声明}

本《大气探测学实习:笔记整理与汇总》一册电子版遵循有限的知识共享许可协议。本书授权包含署名-非商业性使用-相同方式共享(CC BY-NC-SA)。即您被允许在授权范围内对该电子书进行转载、节选、二次创作,但不得用于任何商业目的,且使用时须署原作名,且必须采用与本创作相同的协议(CC BY-NC-SA)进行授权。限于编者水平,本书难免有疏漏错误,敬请读者批评指正。

\begin{tikzpicture}[remember picture, overlay]
    \node [opacity=0.2] at (current page.center){
        \begin{tikzpicture}
            \path[fill=sky] (0:0)--++(90:2)--++(30:6)--++(150:6)--++(210:6)--++(330:6)--++(90:1)--++(150:4)--++(30:4)--++(330:4)--++(210:4)--++(270:3)--++(150:1)--++(210:3)--++(270:1)--++(30:5)--++(330:3)--++(270:1)--++(210:4)--++(150:4)--++(210:1)--++(330:5)--++(30:5)--cycle;
            \path[fill=cyan!50] (0:0)--++(270:1)--++(210:3)--++(150:1)--++(90:5)--++(150:1)--++(270:6)--++(30:6)--++(90:1)--++(30:4)--++(270:6)--++(210:5)--++(270:1)--++(30:6)--++(90:8)--++(210:6)--cycle;
            \path[fill=sky] (90:6)--++(90:1)--++(210:4)--++(330:1)--++(30:3)--++(90:1);
            \path[fill=nuist] (0:0)--++(270:1)--++(330:3)--++(30:1)--++(90:5)--++(30:1)--++(270:6)--++(150:6)--++(90:1)--++(150:4)--++(270:6)--++(330:5)--++(270:1)--++(150:6)--++(90:8)--++(330:6)--cycle;
            \path[fill=nuist] (90:6)--++(90:1)--++(330:4)--++(210:1)--++(150:3)--++(90:1);
        \end{tikzpicture}
    };
\end{tikzpicture}

\tableofcontents

\mainmatter

\chapter{绪论}

\section{地面气象观测业务}

\subsection{发展进程}

人工(表,雨量筒)\(\to\)半自动(机械式,例如双金属片温度计,毛发湿度表,翻斗雨量计,空盒气压表)\(\to\)自动化(称重式雨量计,铂电阻温度传感器,各类传感器)

\subsection{重大改革}

\begin{tabular}{ll}
	1999 & 开始部署自动化试点\\
	2008 & 研发阶段结束\\
	2014前 & 人工半自动观测。笔、纸、电报机。逐级上传,传送报文。\\
	2014-2020 & 人工与自动平行观测。确定自动站数据准确性。\\
	2020.4.1 & 地面气象观测全面自动化。仍然有8个基准站保留人工观察\\
\end{tabular}

\subsection{未来发展:未来站}

组成:一脑三网 气象大脑,基准网(标准配置,增强配置),感知网(简化配置),物联网(专用网络)

形成气象大脑设计方案,在河北保定雄安启动建设

发展相对迅速

\subsection{气象观测站分类}

3层7类18种

\subsubsection{基准气候站}

214部,等级最高,最先进。300-400km一个,观测项目最多,仪器最全,获取具有充分代表性的长期连续资料。

\subsubsection{基本气象站}

600+。150km一个。

\subsubsection{常规气象观测站}

9000+,配备设备少,按照行政区建立,密度很大。三次:8,14,20;五次包括11,17

(区域自动气象站):60000~ 【自动检测异常情况】

\subsubsection{高空气象观测站}

120+个

\subsubsection{应用气象观测站}

个人,社会团体,企业建设并运营的长期气象观测站点,研究各行业关系等志愿气象站

\subsubsection{地面观测}

志愿气象站,大气本地站(29个,基准站),农业气象站,试验基地【0-10m】

\begin{itemize}
	\item 综合观测站
	\begin{itemize}
		\item 大气本底站
		\item 气候观象台:研究气候变化
	\end{itemize}
	\item 观测站
	\item 观测试验基地
	\begin{itemize}
		\item 综合气象观测试验基地:比对观测,新技术实验
		\item 专项试验外场:建立补充站
	\end{itemize}
\end{itemize}

\subsubsection{高空层}

120个【10m-30km】

\begin{itemize}
	\item 观测平台:气象飞机飞艇
	\item 观测站
\end{itemize}

\subsubsection{空间层}

空间观测【30km以上】

\begin{itemize}
	\item 平台:气象卫星
	\item 观测站:空间天气观测站,气象卫星地面站,遥感校验站
\end{itemize}

\subsection{观测方式}

人工目测:云,能见度,天气现象、人工器测、自动观测

\subsection{观测任务}

观测(采集数据)\(\to\)数据处理\(\to\)编发气象报告(数据上报传输)

\subsection{观测项目}

\subsubsection{08、14、20时}

云,能见度,气压,气温,湿度,风向,风速,0-40cm地温(浅层地温)

\subsubsection{08时}

12小时降水量(由于夜间不值班),冻土,雪深,雪压

\subsubsection{14时}

降水量,80-320cm地温,地面状态

\subsubsection{20时}

降水量,蒸发量,最高最低气温,最高最低地面湿度

\subsubsection{日落后}

日日照时数

\subsubsection{说明}

天气现象连续观测

\begin{table}[htbp]
	\centering
	\caption{定时人工观测项目表(实习)}
	\begin{tabular}{|>{\centering\arraybackslash}m{1cm}|>{\centering\arraybackslash}m{2.3cm}|*{3}{>{\centering\arraybackslash}m{1.7cm}|}>{\centering\arraybackslash}m{1.2cm}|}
	\hline
	\multirow{2}{*}{\centering 时间} & \multicolumn{5}{c|}{北京时}\\
	\cline{2-6}	
	 & 08、14、20时 & 08时 & 14时 & 20时 & 日落后\\
	\hline
	观测项目 & 云、能见度、气压、气温、湿度、风向、风速、0-40cm地温 & 12小时降水量、冻土、雪深、雪压 & 降水量、80-320cm地温、地面状态 & 降水量、蒸发量、最高、最低气温、最高、最低地面温度 & 日日照时数 \\
	\hline
	\multicolumn{6}{|c|}{说明:天气现象连续观测} \\
	\hline
	\end{tabular}
\end{table}

\subsection{自动观测项目}

\subsubsection{每小时}

气压气温湿度风向风速地温极值

\subsubsection{20时}

日蒸发量

\subsubsection{地平时}

每小时辐射数据

\subsubsection{24时}

当日极值

\begin{table}[htbp]
	\centering
	\caption{定时自动观测项目表}
	\begin{tabular}{|>{\centering\arraybackslash}m{1cm}|*{4}{>{\centering\arraybackslash}m{2.1cm}|}}
	\hline
	\multirow{2}{1cm}{\centering 时间} & \multicolumn{2}{c|}{北京时} & \multicolumn{2}{c|}{地平时}\\
	\cline{2-5}	
	 & 每小时 & 20时 & 每小时 & 24时\\
	\hline
	观测项目 & 云、气温、湿度、风向、风速、地温及其极值和出现时间、时降水量、时蒸发量 & 日蒸发量 & 辐射时曝辐量、辐射辐照度及其极值和出现时间、时日照时数 & 辐射日曝辐量、辐射日最大辐照度及出现时间、日照总时数\\
	\hline
	\end{tabular}
\end{table}

\subsection{观测流程顺序}

\begin{enumerate}
	\item 正点前30min巡视观测场与仪器【确保仪器与数据正常】,尤其注意湿球温度表球部的湿润状态,做好湿球融冰等工作。
	\item 45-60min观测云,能见度,空气湿度,湿度,降水,风向风速,气压,地温,雪深等【正点必要发送】。
	\item 雪压,冻土,蒸发,地面状态等可在40min至正点后10min进行。
	\item 日落后换纸。
	\item 电线积冰观测时间不固定。
	\item 气压观测时间尽量接近正点,观测顺序必须统一少改。
\end{enumerate}

\subsection{自动观测流程}

\begin{enumerate}
	\item 每日日出后和日落前巡视观测场和仪器设备,具体时间各站自定,但站内必须统一。
	\item 正点前约10分钟查看显示的自动观测实时数据是否正常。
	\begin{itemize}
		\item 00分,进行正点数据采样。
		\item 00-01分,完成自动观测项目的观测,并显示正点定时观测数据。
		\item 01-03分,向微机内录入人工观测数据。
	\end{itemize}
	\item 逐时上传地面小时数据文件、辐射数据文件,按规定上传加密数据文件。
	\item 按规定编发重要天气报告。
	\item 电线积冰观测时间不固定,以能测得一次过程的最大值为原则。
	\item 日落后换日照纸,20时至23时45分上传日照数据文件,复验日照需更正的,在次日10时前更正上传。
	\item 每日20时上传日分钟数据。
	\item 若自动设备故障,经排查在1小时内无法恢复的,即启用备份自动站或便携式自动站,没有的话仅在定时观测时次进行人工补测。
	\item 每日监测并在值班日记中记录探测环境变化情况,探测环境有变化时应及时上报。
\end{enumerate}

地温场\(\to\)云能见度天气现象\(\to\)湿度\(\to\)温度\(\to\)降水\(\to\)风向风速\(\to\)气压\(\to\)输入数据发送

\begin{table}[htbp]
	\centering
	\zihao{-5}
	\caption{正点观测流程(45分~60分)}
	\begin{tabular}{|*{4}{>{\centering\arraybackslash}m{2.1cm}|}}
	\hline
	时间 & \multicolumn{3}{c|}{观测项目}\\
	\hline
	45~47分 & 地面温度(0厘米),地面最低温度,0-20曲管地温,40cm地温 & 地面温度(0厘米),0-20曲管地温,40-320cm地温 & 地面温度(0厘米),地面最高温度,地面最低温度,调整地面最高温度表,调整地面最低温度表,0-20曲管地温,40cm地温\\
	\hline
	47~48分 & \multicolumn{3}{c|}{云,能见度,天气现象}\\
	\hline
	48~50分 & 温度计、湿度计且做时间记号 & 温度计、湿度计且做时间记号 & 干、湿球温度表,最低温度表酒精柱(每月1-5日),最高温度表最低温度(游标),调整最高温度表,调整最低温度表\\
	\hline
	50~51分 & \multicolumn{3}{c|}{温度计、湿度计且作时间记号}\\
	\hline
	51~53分 & 降水量,雪深、雪压 & 降水量 & 降水量,蒸发\\
	\hline
	55~57分 & \multicolumn{3}{c|}{风向,风速}\\
	\hline
	58~59分 & \multicolumn{3}{c|}{气压表、气压计且作时间记号}\\
	\hline
	01~03分 & \multicolumn{3}{c|}{输入人工观测数据、编发报}\\
	\hline
	\end{tabular}
\end{table}

\subsection{时制}

\subsubsection{真太阳时}

人工器测、日照使用

\subsubsection{地方平均太阳时}

辐射、自动观测日照采用

其余采用北京时

\subsection{日界}

日照以日落为界

辐射、自动观测日照:地方平均太阳时24时为界

其余以北京20时为日界

\subsection{对时}

地面气象观测时钟采用北京时

通过互联网实时对时

\section{观测场}

\subsection{环境条件}

有良好的代表性

成立在风向的上风向

四周空旷平坦,避开烟雾,四周无障碍物,高楼等

观测场环境发生变化时应详细记录,并重新评估

\subsection{观测场}

\subsubsection{大小}

25*25m,或16*20(南北)。若有辐射仪器,向南扩展10m。高山,海岛,无人站不受限制

\subsubsection{位置}

确定经纬度,海拔高度,将数据与南北方位立碑

\subsubsection{要求}

1.2m高的稀疏围栏,场地平整,有均匀草层(草高<=20cm)

场内铺设0.3-0.5m宽的路【保护草层,铺设光缆,促进排水】

场内照明防雷装置适当布置

场内风向风速是最高的设备

\subsubsection{总要求}

互不影响便于操作

\subsubsection{仪器布置}

从北方接触仪器,高的在北,低的在南

东西成行不小于4 m,南北成列不小于3 m

主站在备份站的东面或北面

辐射仪器应当安置在南方

\section{地面气象观测系统}

\subsection{自动气象站组成}

硬件和系统软件

\subsection{硬件}

连接到采集器与综合硬件集成控制器

有传感器,采集器,通讯接口,电源计算机等

\subsection{系统软件}

采集软件、地面测报业务软件,此外还有远程监控软件

\subsection{采集系统【核心部分】}

DZZ4自动气象站采集箱

\subsubsection{主采集器}

完成传感器数据采集,预处理,存贮与传输

担当管理者角色,对其他分采集器进行管理

包含有气压传感器。

\subsubsection{分采集器}

例如降水、地温、辐射、雪深等。收到主采集器的信号后采集数据发送给主采集器。

主采集器、分采集器之间采用双绞线CAN总线方式连接,双工通信。

\subsubsection{综合集成硬件控制器}

用于多个自动气象观测设备的集约化管理,具有数据透明传输与数据格式转换功能。

硬件:通信控制模块光电转换模块(室内、室外)、交流防雷模块、供电单元、外围部件等。

软件:驱动程序、管理软件。

驱动程序用于虚拟串口,对综合集成硬件控制器进行配置管理。

管理软件基于TCP/IP协议,与综合集成硬件控制器进行交互、管理。

\subsubsection{观测业务软件}

ISOS:实现自动气象观测数据采集,业务处理,数据传输

\chapter{气象要素观测}

记录中均只写数值,不写单位

\section{人工观测-云}

\subsection{云的观测与记录}

\subsubsection{内容}

判定云状、估计云量、测定云高、选定云码

全天空成像仪:结合人工观察再进行算法判断

\subsubsection{要求}

\begin{enumerate}
	\item 尽量选择能看到全部天空和地平线的开阔地点
	\item 注意云的连续演变(应当实时观测)
	\item 观测时若阳光强,可带遮光镜
\end{enumerate}

\subsubsection{云状判定}

根据云的外形特征、结构、色泽、排列、高度以及伴见的天气现象,参照“云图”,对比判定。特别注意云的连续演变过程

\subsubsection{云状记录}

记录简写符号。记录29种云类符号。云量多的在前,云量少的在后,无云时,空白不填(最多纪录5种,挑主要的)。

\subsubsection{云量测定}

\begin{enumerate}
	\item 10-:天空完全为云所遮蔽,但只要从云隙中可见青天
	\item 0 微量云:天空有少许云,其量不到天空的十分之零点五时,总云量记 0
	\item 无云为0,以10划分
\end{enumerate}

\begin{table}[htbp]
	\centering
	\caption{三族十云}
	\begin{tabular}{|>{\centering\arraybackslash}m{1cm}|*{2}{>{\centering\arraybackslash}m{2.1cm}|}*{2}{l|}}
	\hline
	\multirow{2}{*}{云族} & \multicolumn{2}{c|}{云属} & \multicolumn{2}{c|}{云类}\\
	\cline{2-5}
	 & 学名 & 简写 & 学名 & 简写\\
	\hline
	\multirow{14}{*}{低云} & \multirow{3}{*}{积云} & \multirow{3}{*}{Cu} & 淡积云 & Cu hum\\
	 & & & 碎积云 & Fc\\
	 & & & 浓积云 & Cu cong\\
	\cline{2-5}
	 & \multirow{2}{*}{积雨云} & \multirow{2}{*}{Cb} & 秃积雨云 & Cb calv\\
	 & & & 鬃积雨云 & Cb cap\\
	\cline{2-5}
	 & \multirow{5}{*}{层积云} & \multirow{5}{*}{Sc} & 透光层积云 & Sc tra\\
	 & & & 闭光层积云 & Sc op\\
	 & & & 积云性层积云 & Sc cug\\
	 & & & 堡状层积云 & Sc cast\\
	 & & & 荚状层积云 & Sc lent\\
	\cline{2-5}
	 & \multirow{2}{*}{层云} & \multirow{2}{*}{St} & 层云 & St\\
	 & & & 碎层云 & Fs\\
	\cline{2-5}
	 & \multirow{2}{*}{雨层云} & \multirow{2}{*}{Ns} & 雨层云 & Ns\\
	 & & & 碎雨云 & Fn\\
	\hline
	\multirow{8}{*}{中云} & \multirow{2}{*}{高层云} & \multirow{2}{*}{As} & 透光高层云 & Ac tra\\
	 & & & 闭光高层云 & Ac op\\
	\cline{2-5}
	 & \multirow{6}{*}{高积云} & \multirow{6}{*}{Ac} & 透光高积云 & Ac tra\\
	 & & & 避光高云 & Ac op\\
	 & & & 荚状高积云 & Ac lent\\
	 & & & 积云性高积云 & Ac cug\\
	 & & & 絮状高积云 & Ac flo\\
	 & & & 堡状高积云 & Ac cast\\
	\hline
	\multirow{7}{*}{高云} & \multirow{4}{*}{卷云} & \multirow{4}{*}{Ci} & 毛卷云 & Ci fil\\
	 & & & 密卷云 & Ci dens\\
	 & & & 伪卷云 & Ci not\\
	 & & & 钩卷云 & Ci unc\\
	\cline{2-5}
	 & \multirow{2}{*}{卷层云} & \multirow{2}{*}{Cs} & 毛卷层云 & Cs fil\\
	 & & & 薄幕卷层云 & Cs nebu\\
	\cline{2-5}
	 & 卷积云 & Cc & 卷积云 & Cc\\
	\hline
	\end{tabular}
\end{table}

\subsection{云高观测}

\subsubsection{估测}

首先必须正确判定云状,同时可根据云体结构,云块大小、亮度、颜色、移动速度等情况,结合本地常见的云高范围进行估测

海拔高,云底相对低,和湿度有很大关系;云体结构松散高度低,云速快高度低,云体阴暗高度低,云快大高度低

\subsubsection{记录}

\begin{enumerate}
	\item 以米为单位,记录取整数:云底距离地面的垂直距离。
	\item 在云高数值前加记云状,记录十属和Fc,Fs,Fn(碎积云,碎层云,碎雨云)三个云类,记录最低云云底高度。例如Cu 800m(积云800m)。
\end{enumerate}

低云	<2500		层云最低

中云	2500-4500	

高云	>4500		卷云最高

\subsubsection{夜间云的观测记录}

	注意白天云的演变趋势,需要先适应黑暗一段时间,参照星光的疏密、清晰程度,云体的颜色、移动速度以及伴见的天气现象和实测云高。

	连续性,间歇性,阵性降水:观察天气现象做出判断

\subsubsection{特殊情况}

	天空状况不明时云状、云量的记录

\begin{enumerate}
	\item (能见度<1km)因雪暴、雾使天空的云量、云状无法辨明时,总、低云量记10,云状栏记录该现象的符号。
	\item (能见度<10km)因吹雪、雾、轻雾使天空的云量、云状不能完全辨明时,总、低云量记10,云状栏记该现象符号和可见的云状。
	\item 虽有吹雪、雾的现象,但天空可完全辨明时,按正常情况记录。
	\item 因烟幕、霾、浮尘、沙尘暴、扬沙等视程障碍现象使天空云量、云状全部或部分不明时,总、低云量记“-”。
\end{enumerate}

\subsubsection{云状演变}

\begin{enumerate}
	\item 云随天气系统的移动,不同种类的云依次经过上空,使得看起来像是云在发生变化(例如暖锋云系 卷云\(\to\)卷层\(\to\)高层\(\to\)雨层云,冷锋云系)。
	\item 云体自身的演变,如云增厚,变薄,延伸扩展或蒸发消失。例如积云\(\to\)浓积云\(\to\)秃积雨云\(\to\)鬃积雨云。
\end{enumerate}

\section{人工观察-能见度}

\subsection{要求}

选择在视野开阔,能看到所有目标物的固定地点作为能见度的观测点。和云、天气现象在用同一个点观测。

\subsection{记录}

以千米(km)为单位,取一位小数,例如10.2km,不足0.1千米记0.0

\subsection{白天能见度的观测}

  事先有选择目标物并测定距离,观测四周事先测定的各目标物,根据“能见”的最远目标物和“不能见”的最近目标物,从而判定当时的能见距离。

\subsection{特殊情况判定}

\begin{enumerate}
	\item 五倍以上:目标物的颜色、细微部分(如村庄的单个树木、远处房屋的门窗等)清晰可辨时,能见度通常可定为该目标物距离的五倍以上;根据观测积累与常年历史资料判断。
	\item 二倍半到五倍:目标物细微部分隐约可辨,能见度可定为二倍半到五倍。
	\item 不应超过二倍半:目标物的颜色、细微部分很难分辨时,能见度可定为大于该目标物的距离。
\end{enumerate}

运用以上几点时,应考虑到目标物的大小,背景颜色,以及当时的光照等情况。

目前基地可参考的:钟楼0.28km,基地标牌0.08km,龙王山雷达1.8km等

\subsection{夜间观测能见度}

  先在黑暗处停留5—15分钟,根据最远目标灯能见与否确定能见距离。

  在无条件利用目标灯进行观测的情况下,只能根据天黑前能见度的实况和变化趋势,结合观测时天气现象、湿度、风等气象要素的变化情况,以及实践经验加以判定。

  月光较明亮时,只要能隐约地分辨出比较高大的目标物的轮廓,该目标物距离就可定为能见距离;如能清楚分辨时,能见距离可定为大于该目标物的距离。

\section{天气现象的观测}

\begin{enumerate}
	\item 观测时间不固定,需要观测所有的出现的天气现象。夜间不守班的,对夜间出现的天气现象,应尽量判断记录,通过判断凌晨露、霜等判断记录。
	\item 为正确判断某一现象,有的时候还要参照气象要素的变化和其它天气现象综合进行判断。
	\item 凡与水平能见度有关的现象,均以有效水平能见度为准,并在能见度观测地点观测判断天气现象。
\end{enumerate}
共34种天气现象,如图\ref{weather}

\begin{table}
	\centering
	\linespread{1.8}\selectfont
	\caption{天气现象}\label{weather}
	\begin{tabular}{|*{8}{c|}}
		\hline
		现象名称 & 符号 & 现象名称 & 符号 &	现象名称 & 符号 & 现象名称 & 符号\\
		\hline
		雨 & \tikz\fill (0,0) circle[radius=2pt]; & 冰粒 & \tikz{\draw (90:0.2)--(210:0.2)--(330:0.2)--cycle; \fill (0,0) circle[radius=1pt];} & 吹雪 & \tikz{\draw[->](-0.2,0)--(0.2,0); \draw[->](0,-0.2)--(0,0.2);} & 极光 & \tikz{\draw (0.2,0) arc (0:180:0.2); \draw (-0.2,0)--(0.2,0); \draw (36:0.2)--(36:0.3); \draw (72:0.2)--(72:0.3); \draw (108:0.2)--(108:0.3); \draw (144:0.2)--(144:0.3); }\\
		\hline
		阵雨 & \tikz{\fill (90:0.2) circle[radius=1pt]; \draw (30:0.2)--(150:0.2)--(270:0.2)--cycle;} & 冰雹 & \tikz\draw (90:0.2)--(210:0.2)--(330:0.2)--cycle; & 雪暴 & \tikz{\draw[<->](-0.2,0)--(0.2,0); \draw[<->](0,-0.2)--(0,0.2);} & 大风 & F\\
		\hline
		毛毛雨 & \hspace{0.3cm}{\Huge,}\hspace{-0.3cm} & 冰针 & \tikz\draw[<->](-0.2,0)--(0.2,0); & 烟幕 & \tikz\draw (-0.2,-0.2)--(-0.2,0.2)--(-0.15,0.15)--(-0.1,0.2)--(-0.05,0.15)--(0,0.2)--(0.05,0.15)--(0.1,0.2)--(0.15,0.15)--(0.2,0.2); & 飑 & \tikz\draw (60:0.2)--(90:0.1)--(120:0.2)--(270:0.2)--cycle;\\
		\hline
		雪 & \tikz{\draw (-0.2,0)--(0.2,0); \draw (60:0.2)--(240:0.2); \draw (120:0.2)--(300:0.2);} & 露 & \tikz{\draw (0,0) circle[radius=0.2]; \draw (-0.2,-0.2)--(0.2,-0.2);} & 霾 & \(\infty\) & 龙卷 & 〕\hspace{-0.4cm}〔\\
		\hline
		阵雪 & \tikz{\draw (-0.2,0)--(0.2,0); \draw (30:0.16)--(210:0.16); \draw (150:0.16)--(330:0.16); \draw (330:0.2)--(270:0.3)--(210:0.2)--cycle;} & 霜 & \tikz\draw (-0.2,0.2)--(-0.2,-0.2)--(0.2,-0.2)--(0.2,0.2); & 沙尘暴 & \tikz{\draw[->](-0.2,0)--(0.2,0); \node at (0,0) {S};} & 尘卷风 & \tikz{\draw (0.13,0.175) arc (30:330:0.2 and 0.1); \draw (0.13,0.075) arc (30:330:0.2 and 0.1); \draw (0.13,-0.025) arc (30:330:0.2 and 0.1);}\\
		\hline
		雨夹雪 & \tikz{\draw (-0.2,0)--(0.2,0); \draw (60:0.2)--(240:0.2); \draw (120:0.2)--(300:0.2); \fill (90:0.15) circle[radius=1pt];} & 雾凇 & V & 扬沙 & \tikz{\draw[->](0,-0.2)--(0,0.2); \node at (0,0) {S};} & 积雪 & \tikz{\draw (-0.2,0)--(0.2,0); \draw (60:0.2)--(240:0.2); \draw (120:0.2)--(300:0.2); \draw (-0.2,-0.2) rectangle (0.2,0.2);}\\
		\hline
		阵性雨夹雪 & \tikz{\draw (-0.2,0)--(0.2,0); \draw (30:0.16)--(210:0.16); \draw (150:0.16)--(330:0.16); \draw (330:0.2)--(270:0.3)--(210:0.2)--cycle; \fill (0,0.1) circle[radius=1pt]} & 雨凇 & \(\sim\) & 浮尘 & S & 结冰 &  \tikz{\draw (-0.2,0.2)--(-0.2,-0.2)--(0.2,-0.2)--(0.2,0.2); \draw (-0.2,-0.1)--(0.2,-0.1)}\\
		\hline
		霰 & \tikz{\draw (-0.2,0)--(0.2,0); \draw (60:0.2)--(240:0.2); \draw (120:0.2)--(300:0.2); \draw (-0.2,-0.2)--(0.2,-0.2)} & 雾 & \tikz{\draw (-0.2,0.2)--(0.2,0.2); \draw (-0.2,0)--(0.2,0); \draw (-0.2,-0.2)--(0.2,-0.2);} & 雷暴 & \tikz\draw[->](-0.2,-0.2)--(-0.2,0.2)--(0.2,0.2)--(0,0)--(0.2,-0.2); & & \\
		\hline
		米雪 & \tikz{\draw (90:0.2)--(210:0.2)--(330:0.2)--cycle; \draw (-0.2,0)--(0.2,0);} & 轻雾 & \tikz{\draw (-0.2,0.1)--(0.2,0.1); 	\draw (-0.2,-0.1)--(0.2,-0.1);} & 闪电 & \tikz\draw[->](0,0.2)--(-0.2,0.15)--(0.2,-0.2); & & \\
		\hline
	\end{tabular}
\end{table}

当前保留观测和记录的有21种:雨、阵雨、毛毛雨、雪、阵雪、雨夹雪、阵性雨夹雪、冰雹、露、霜、雾凇、雨凇、雾、轻雾、霾、沙尘暴、扬沙、浮尘、大风、积雪、结冰

取消了13种:霰、米雪、冰粒、吹雪、雪暴、烟幕、雷暴、闪电、极光、飑、龙卷、尘卷风、冰针。虽然取消,但仍然需要记录雷暴,龙卷,在重要天气报内。

其中:雪暴、霰、米雪、冰粒出现时,记为雪,这4种天气现象与雨同时出现时,记为雨夹雪。自动气象站可能无法区分过于细致的天气现象。

\subsection{记录方法}

\begin{enumerate}
	\item 天气现象按出现的先后顺序记录。
	\item 记录开始与终止时间(15个):雨、阵雨、毛毛雨、雪、阵雪、雨夹雪、阵性雨夹雪、冰雹、雾、雨淞 、雾淞、沙尘暴、扬沙、浮尘、大风。
	\item 不记起止时间(6个):轻雾、露、霜、积雪、结冰、霾。
	\item 时间不足一分钟即已终止时,则只记开始时间,不记终止时间。例如阵雨。\par\qquad
	例如:毛毛雨15时18分出现,17时结束,则记为“\(15^{18}-17\)”。
	\item 正好出现在20时,不论该现象持续与否,均应记入次日天气现象栏;正好终止在20时,则应记在当日天气现象栏。
	\item 夜间不守班的气象站,观测簿中的天气现象栏划分“夜间(20-8时)”和“白天(8-20时)”两栏。夜间出现的天气现象记入“夜间”栏,只记符号,一律不记起止时间。\\
	如现象正好出现在08时,不论该现象持续与否,均应记入“白天”栏;如正好终止在08时,则记在“夜间”栏;如现象由夜间持续至08时以后,则按规定分别记入两栏。\par\qquad
	例如:15日19时12分有阵雨,20时转为小雨,持续到21时15分。16日5时30分有露凝结,直到8时28分消失。\\
	\begin{center}
		\begin{tabular}{ccll}
			\hline
			15日 & 白天 & 阵雨 & \(19^{12}-20\)\\
			\hline
			16日 & 夜间 & 雨\(\ \cdot\) & (夜间所有现象只记符号)\\
			 & & 露\(\Omega\) & \\
			 & 白天 & 露\(\Omega\) & \\
			\hline
		\end{tabular}
	\end{center}
	\item 凡同一现象一天内出现两次或以上时,其第二次及之后出现的起止时间,可接着第一次起止时间分段记入,不再重记该现象符号。
	\item 大风的起止时间,两段出现时间间歇\(\leq\)15min作为一次记载;若间歇时间>15min另记起止时间。\par\qquad
	例如:某日大风实际出现时间为:12时10分至12时25分,12时30分至12时42分,13时至13时零8分。\par\qquad
	大风\quad \(12^{10}-12^{42}\ 13-13^{08} (30-25=5<15)\)。
	\item 最小能见度的记录规定\begin{enumerate}
		\item 沙尘暴、雾以及浮尘、霾等现象出现能见度小于1.0km时,都应观测和记录最小能见度,记录加方括号“[ ]”。每一现象出现时,每天只记录一个最小能见度。
		\item 最小能见度是指最小有效水平能见度,以m为单位取整数。整个过程中出现的最低值。\par\qquad
		例如:雾 \(10^{11}-12^{50}[460]\)。
	\end{enumerate}
	\item 降雹时应测定最大冰雹的最大直径,以毫米(mm)为单位,取整数。当最大冰雹的最大直径大于10mm时,应同时测量冰雹的最大平均重量,以克(g)为单位,取整数,均记入纪要栏。
	\item 雷暴、龙卷的记录\par\qquad
	当有雷暴或龙卷出现时,不记录在气簿-1里,而是记录在值班日记中,作为重要天气报编发的依据。龙卷的记录规则同雨,需要记录起止时间。\par\qquad
	雷暴应从整体出发判别其系统,记录其起止时间和开始、终止方向,切忌零乱记载。
	\begin{itemize}
		\item 起止时间的记法:
		\begin{itemize}
			\item 以该系统第一次闻雷时间为开始时间,最后一次闻雷时间为终止时间。
			\item 两次闻雷时间\(\leq\)15min连续记载。
			\item 如两次间隔时间>15min另记起止时间。
			\item 如仅闻雷一声,只记开始时间。
		\end{itemize}
		\item 方向的记法:\par\qquad
		按八方位记载,以该系统第一次闻雷的所在方位为开始方向,最后一次闻雷的所在方位为终止方向,方位记载右下角。
		\begin{itemize}
			\item 若雷暴始终在一个方位,只记开始方向。
			\item 若雷暴经过天顶,要记天顶符号“Z”。
			\item 若起止方向之间达到180°或以上时,须按雷暴的行径,在起止方向间加记一个中间方向。
			\item 当起止方向不明或多方闻雷而不易判别系统时,则不记方向。\par\qquad
			例如:雷暴 \(14^{17}_{\mathrm{W-S}}15^{42}_{\mathrm{SE}}\quad 16_\mathrm{E}-16^{08}\)
		\end{itemize}
	\end{itemize}
\end{enumerate}

某夜22:27出现灰白稀薄雾幕,能见度5.0km,23:19雾增浓,能见度小于1.0km,23:36能见度0.8km,1:28能见度0.4km,3:20变薄能见度1.0km【变为轻雾】,直到8:09雾幕消失。14:22自浓积云【阵雨】降雨数滴,14:53又降,15:07止,16:22又降,到17:46止。 

\begin{center}
	\begin{tabular}{cl}
		\hline
		夜间 & 薄雾\\
		 & 雾[400]\\
		\hline
		白天 & 轻雾\bcancel{\(08-08^{09}\)}\\
		 & 阵雨\(\quad14^{22}\quad14^{53}-15^{07}\quad16^{22}17^{46}\)\\
		\hline
	\end{tabular}
\end{center}

11月27日21:09起有露,5:16露冻结成冰珠,到9:17消失。14:20高层云降间歇性的雨,17:45止,18:51又下,19:23降水在天空为雪片,但降自地面即融化成水滴;19:40蒸发器内水结冰;20:00转为雪,22:13地面积雪过半,23:45雪止。

\begin{center}
	\begin{tabular}{cl}
		\hline
		夜间 & 露\\
		 & 霜\\
		\hline
		白天 & 霜\\
		 & 雨\(14^{20}-17^{45}\quad18^{51}-19^{23}\)\\
		 & 雨夹雪 \(19^{23}-20\)\\
		 & 结冰\\
		 & 雪\\
		\hline
		夜间 & 结冰\\
		 & 雪\\
		 & 积雪\\
		\hline
	\end{tabular}
\end{center}	

8月6日13:28西北闻雷,13:32开始有阵雨,雷声频繁(间隔小于15分钟),13:52移至天顶,夹降冰雹,13:55-13:59纯为冰雹,以后又是阵雨冰雹夹降,14:06冰雹止,14:07雷声移到东方,14:20于东方消失。阵雨维持到15:22,19:20又降阵雨,19:44南方闻雷,19:55西方闻雷,20:05移至西北方,于20:21在北方消失,阵雨于23:11转为间歇性的降水,3:40终止

\begin{center}
	\begin{tabular}{ccl}
		\hline
		8月6日 & 夜间 & \\
		 & 白天 & 雷\(13^{28}_\mathrm{WN}-13^{52}_\mathrm{T}-14^{07}_\mathrm{E}-14^{20}\quad19^{44}_\mathrm{S}-20_\mathrm{N}\)\\
		 & & 阵雨\(13^{32}-13^{55}\quad13^{59}-15^{22}\quad19^{20}-20\)\\
		 & & 冰雹\(13^{52}-14^{06}\)\\
		\hline
		8月7日 & 夜间 & 雷\\
		 & & 阵雨\\
		 & & 雨\\
		\hline
	\end{tabular}
\end{center}

\section{气压观测}

\subsection{观测方法}

\begin{enumerate}
	\item 观测附属温度表。
	\item 调整水银槽内水银面,使之与象牙针尖恰恰相接。
	\item 调整游尺与读数。游尺下缘零线所对标尺的刻度即可读出整数。再从游尺刻度线上找出一根与标尺上某一刻度相吻合的刻度线,就是小数读数。
	\item 读数复验后,降下水银面。 旋转槽底调整螺旋,使水银面离开象牙针尖约2—3mm(防止象牙针尖长时间浸没入水银)。
\end{enumerate}

\subsection{记录方法}

\begin{enumerate}
	\item 附属温度表,读数精确到0.1℃,读数记入观测簿相应栏内。
	\item 并进行器差订正。以百帕(hPa)为单位,取一位小数。
\end{enumerate}

\section{温度}

\subsection{温度的观测}

气温/干、湿球温度/日最高气温/日最低气温

\subsection{地温}

地面温度/地面最高、最低温度/浅层地温:离地面5,10,15,20厘米/深层地温:离地面40,80,160,320厘米

温度记录:

各种温度表读数要准确到0.1℃。温度在0℃以下时,应加负号(“-”)。读数记入观测簿相应栏内,并进行器差订正。

注:实习时,器差统一为0.1℃,其他观测项目相同。

\subsection{干湿球温度表}

左边为干球,右边为湿球,上部为最高温度,下部为最低温度(酒精)

\begin{enumerate}
	\item 观测时必须保持视线和水银柱顶端齐平,以避免视差。
	\item 读数动作要迅速,不要对着温度表呼吸,尽量缩短停留时间,并且勿使头、手和灯接近球部,以避免影响温度示度。同时读数,记录并复读。
	\item 注意复读,以避免发生误读或颠倒零上、零下的差错。最小刻度为0.2℃,需要估读。
\end{enumerate}

\subsection{最高温度表}

\begin{enumerate}
	\item 观测时,水银柱若有上滑脱离窄道的现象,应稍抬温度表的顶端,使水银柱回到正常的位置,再读数。 在观测中发现断柱时,应稍抬温度表的顶端使其连接在一起。若不能恢复,则减去断柱的数值作为读数,并及时进行修复或更换。并在气簿的备注栏注明。
	\item 全天最高气温在-36.0℃以下时,停止最高温度表的观测,记录缺测,并在气簿备注栏注明。
\end{enumerate}

\subsubsection{重置与调整方法}

\begin{enumerate}
	\item 手握表身,感应部分向下,臂向外伸出约30°,用大臂将表前后甩动不超过45°,甩动方向与刻度磁板面平行。直至使示度接近于当时的干球温度。
	\item 调整时,动作迅速,尽量避免阳光照射,不能用手接触感应部分。不要甩动到使感应部分向上的程度,以免水银柱滑上又甩下,撞坏窄道。
	\item 调整后,把表放回到原来的位置上时,先放感应部分,后放表身。避免水银滑动。
\end{enumerate}

\subsection{最低温度表(酒精)}

观测时,眼睛应平直地对准游标右侧即为温度值;观测酒精柱示度时,眼睛应平直地对准酒精顶端凹面中点(即最低点)的位置。

当在观测读数发现最低温度表酒精柱中断时,最低温度记录作缺测处理,并在观测簿的备注栏注明;该表须及时修复或更换。 

\subsubsection{重置调整方法}

抬高温度表的感应部分,表身倾斜,使游标回到酒精柱的顶端。 

\subsection{地温表}

位于裸地,读数比气温高很多。

观测时,要踏在踏板上,按0cm(左边)、最低、最高(右边)和(左边)5、10、15、20cm(右边)地温的顺序读数。观测地面温度时,应俯视读数,不准把地温表取离地面。读数记入观测簿相应栏,并进行器差订正。禁止手碰。

\subsubsection{特殊情况处理}

地温表被水淹时,可照常观测,其中地面三支温度表应水平地取出水面,迅速进行读数。在拿取地温表时,须注意勿使水银柱、游标滑动,手也不能触及地温表感应部分。若遇地温表漂浮于水中,则记录从缺。 

\subsection{深层地温表}

(最左)40cm地温表于每日8、14、20时观测

80、160、320cm(最右)地温表于每日14时观测一次。观测和记录要求,同地面、曲管地温表。

观测深层地温时,应在台架上按由浅至深的顺序,把直管地温表从套管中迅速取出读数;观测后将表轻轻插回套管,盖好顶盖。

\section{风}

\subsection{记录内容}

风向---两分钟最多风向

风速---两分钟平均风速

\subsection{记录方法}

风速---记录自动站的两分钟平均风速 

风向---用十六方位对应符号记录。静风时,风速记0.0,风向记C。

\section{降水}

\subsection{每天08、14、20时测量}

\begin{enumerate}
	\item 观测液体降水时要换取储水瓶,将水倒入量杯,要倒净
	\item 将量杯保持垂直,使人的视线与水面齐平,以水凹面为准,读得刻度数即为降水量,记入相应栏内
	\item 降水量大时,应分数次量取,求其总和
\end{enumerate}

\subsection{降雪测量}

\begin{enumerate}
	\item 冬季降雪时,须将承雨器取下,换上承雪口,取走储水器,直接用承雪口和外筒接收降水。
	\item 观测时,将已有固体降水的外筒,用备份的外筒换下,盖上筒盖后,取回室内,将固体降水连同外筒用专用的台秤称量,称量后应把外筒的重量(或mm数)扣除。
\end{enumerate}

\subsection{特殊情况处理}

在炎热干燥的日子,为防止蒸发,降水停止后,要及时进行观测。

在降水较大时,应视降水情况增加人工观测次数,以免降水溢出雨量筒,造成记录失真。

纯雾、露、霜、雾凇等的量按无降水处理

\subsection{记录方法}

以毫米(mm)为单位,取一位小数。无降水时,降水量栏空白不填。不足0.05mm的降水量记0.0【特指微量降水】

\section{日照观测}

\subsection{暗筒式日照计}

\subsection{日照纸的更换}

每天日落后换纸,即使全日阴雨无日照,也照常换下,必备日后考查。

上纸时,使纸上10时线对准筒口的白线,14时线对准筒底的白线。

纸上两圆孔对准两个进光孔,压纸夹叉向上,将纸压紧,盖好筒盖。

\subsection{记录整理}

\begin{enumerate}
	\item 换下的日照纸,依感光迹线长短,在其下画铅笔线,计算各时的日照时数,以十分法记录,准确到小数一位。将各时的日照时数相加,则为全天的日照时数。
	\item 将日照纸放入清水中浸漂3-5分钟拿出(全天无日照的纸也应浸漂);待阴干后,再复验。如感光迹线比铅笔长,则因补描上这段铅笔线,并改正原计算的日照时数,填上日合计。如全天无日照,记0.0。
\end{enumerate}

\section{蒸发观测}

每天20时进行观测,测量前一天20时注入的20mm清水(即今日原量)经24小时蒸发剩余的水量,然后倒掉余量,重新量取20mm(干燥地区和干燥季节须量取30mm)清水注入蒸发器内,并记入次日原量栏。

蒸发量计算公式:蒸发量=原量+降水量-余量

\begin{enumerate}
	\item 有降水时,应取下金属丝网圈;有强降水时,应注意从器内取出一定的水量,以防水溢出;若结冰则放回室内进行称量。
	\item 因降水或其它原因,致使蒸发量为负值时,记0.0。蒸发器中的水量全部蒸发完时,按加入的原量值记录,并加“>”,如>20.0
\end{enumerate}

\chapter{湿度计算和海平面气压换算方法}

\section{本站气压换算}

使用水银气压表的台站,按下面公式计算本站气压
\begin{gather*}
	P_h=(P+C)\times\frac{g_{\varphi, Z}}{g_n}\times\frac{1+\lambda t}{1+\mu t}\\
	P_h=(P+C)\times\frac{g_{\varphi, Z}}{9.80665}\times\frac{1+0.0000184t}{1+0.0001818t}
\end{gather*}

\(P_h\)—本站气压(hPa)

\(P\)—水银气压表读数(hPa)

\(C\)—器差订正值(hPa)

\(g_n\)—标准重力加速度, \(g_n=9.80665\mathrm{m/s^2}\)

\(g_{\varphi, z}\)—测站重力加速度

世界气象组织推荐台站的重力加速度计算公式为(陆地站)
\[
g_{\varphi, z}=g_{\varphi, 0}-0.000003086h+0.000001118(h-h')
\]

—纬度处的平均海平面重力加速度(\(\mathrm{m/s^2}\))

\(h\)—测站海拔高度(m)

\(h'\)—以测站为圆心,在半径为150km范围内的平均海拔高度(m)

在周围地形较平坦的台站,设\(h=h'\)

\(\lambda\)—铜尺膨胀系数,其值为0.0000184/℃

\(\mu\)—水银膨胀系数,其值为0.0001818/℃

\(t\)—经器差订正后的水银气压表附温表读数(℃)

共三部:仪器差,重力差,温度差

探测基地海拔高度\(h\)为22m、纬度为32°12’

\section{海平面气压换算}

为比较各地气压的高低,便于气压场的分析,需把本站气压统一订正到海平面上,即作海平面气压订正。

由拉普拉斯压高公式:

\(p_0\):海平面气压(hPa)

\(p_h\):本站气压(hPa)

\(h\):测站海拔高度(m)

\(t_m\):假想气柱的平均温度(℃)

\(t\):观测时气温(℃)

\(t_{12}\):观测前12小时气温(℃)

用\(t_m\)和\(p_h\)查气压简表也可得\(P_0\)

横坐标为\(p_h\)整数,纵坐标为\(t_m\)

\begin{enumerate}
	\item tm查接近的度数、 ph用整数部分查表。
	\item 查得的气压值加上原ph的小数部分即为p0
\end{enumerate}

\section{湿度计算}

\subsection{水汽压}

E —水汽压(hPa)

Etw —湿球温度tw所对应的纯水平液面的饱和水汽压

ph —本站气压(hPa)

t —干球温度(℃)

tw —湿球温度(℃)

A —干湿表系数(℃-1),取A=0.8×10-3 (℃-1)

\subsection{露点温度}

Magnus公式

\(\to\)

E —实际水汽压(hPa)

E0 — 0 ℃时的饱和水汽压,取6.1078( hPa )

a —系数,取7.69

b —系数,取243.92

\subsection{相对湿度}

U—相对湿度(\%)

E —实际水汽压(hPa)

Ew —干球温度t所对应的纯水平液面的饱和水汽压(hPa)

其中Etw、Ew 可根据tw、t值查表求得

\chapter{天气现象电码}

\section{电码格式}

只剩一项天气现象需要编码

0段		AAXX	YYGG1		(SMG)	(SIG)

  识别组	日期时间	指示组

1段			IIiii  	iRiXhVV		Nddff		1snTTT		2snTdTdTd 		 

基本资料段	区站号	降水云高能见度	总云量风向风速	干球温度		露点温度		

			3P0P0P0P0	4PPPP		5appp		6RRR1		7wwW1W2

     海平面气压	本站气压		三小时变压    6小时降水量	天气现象

			8NhCLCMCH	9GGgg

			云码   		重要天气

3段			333XX		0P24P24T24T24				1snTXTXTX	2snTnTnTn

补充资料段	指示组		24小时变压变温(08时编报)	最高气温		最低气温

  3snTgTgTg	7R24R24R24R24	9SpSpspsp

     地温			24小时降水		重要天气报  

\section{天气现象编码}

7wwW1W2	7表示指示码,在目前的软件中隐藏

  ww表示现在的天气现象(观测时(45-00)与观测前1h内的天气现象)

  W1、W2表示过去天气现象,过去6小时内出现的天气现象

时间概念

	观测时			定时观测的15min

以08时为例,即07:45 — 08:00 (注:包括08:00,不包括07:45)

	观测前1小时	指定时观测前的45分钟。即07:00—07:45

	过去1小时		包括观测时和观测前1小时。即 07:00—08:00

	过去1小时前整个时段	不包括观测时和观测前1小时。即 20:00—07:00

	过去3小时		指前次基本天气观测到本次补充天气观测的3小时。以11时为例,即08:00—11:00。

	过去6小时		指前次基本天气观测到本次观测的6小时。

                             以14时为例,即08:00—14:00

	过去12小时		指前次基本天气观测到本次观测的12小时。

                           以08时为例,即20:00—08:00

	注:其他整点观测时次(9、10、12、13、15、16、18、19),只考虑过去1小时内的天气现象,就没有过去天气现象电码(W1W2=XX)

\section{天气现象电码表}

00 没有天气现象

霾尘沙	05	观测时有霾

		10	轻雾

观测前1小时有,但观测时没有	20-28

观测时沙尘暴30-35

观测时有雾	40-49	观测时没有降水

观测时毛毛雨50-59

观测时非阵性雨60-69

观测时非阵性固体降水70-75

阵性	80-90

\section{ww编码原则}
\begin{enumerate}
	\item 若有好几种天气现象,选择最大的码进行编报。但28比40优先。
	\item 15分钟内都算是观测时的现象,若15分钟内出现两种以上现象时,尽量合并选码,若不能,则选择大的电码。
	\item 只有观测时或观测前出现间歇性,才可以选用
	\item 雾霾同时存在时,无需区分各自对能见度的影响,只要能见度<1000,选择42-49即可
	\item 只有能见度>=1000时看到的雾才编码40、41. 如测站有效能见度<1000米,则应选择42-49中的适当电码编报。
	\item 有些天气现象编报时需区分强度。强度分为小(轻)、中常、大(浓、强)三级,根据观测时的降水情况或有效能见度参照天气现象强度表进行判定。
\end{enumerate}

\section{W1W2编码原则}

05-霾 10-轻雾 没有对应的过去天气现象编码
\begin{enumerate}
	\item W1W2编报应当从ww编码所对应的W1W2以外的其他码(0码除外)中选报
	\item 按1的要求若有两个或以上码可供选报W1、W2时,应以其中码数最大编报W1,次大的编报W2
	\item 按1的要求仅有另一种天气现象可编报W1、W2时,如果:
	\begin{enumerate}
		\item 如果ww所编报的天气现象开始出现在过去1小时以前,应重复编报此现象。此时,应从此现象和另一种现象中选电码大的编报W1,小的编报W2。
		\item 如果ww所编报的天气现象只出现在过去1小时以内,就不再重复编报它,而以另一种现象编报为W1;至于W2的编报,要看该现象持续的时段而定。
		\item 如果W1码现象持续占满过去1小时之前的整个时段(间歇性或阵性降水等现象的固有间断应看作是持续存在),则W2重复编报W1码(即W1= W2)
		\item 如果W1码现象未持续占满过去1小时之前的整个时段,则W2编报0。
	\end{enumerate}
	\item 过去天气时段内只出现ww所编报的一类天气现象时,如果这种天气现象只出现在过去一小时以内,则W1W2报 00;如果这种天气现象出现在过去一小时以前,就用它报W1,至于W2要看是否占满过去一小时之前的整个时段。
	\item ww编报00或过去天气现象电码表下注(2)中各码时,W1W2直接从过去天气时段内出现的电码 3-8 现象中选取。如果整个时段内没有出现W1W2规定要编报的现象时,则W1W2编报00。 
\end{enumerate}

\section{重要报内容}

大风、龙卷、冰雹、雷暴和视程障碍现象(雾、霾、浮尘、沙尘暴)

雷暴、龙卷两种现象,记录在值班日记中,作为编发的依据。

\subsection{基本格式}

0段		(WS)		GGggW0    	IIiii

识别组	 	时数分钟指示码		区站号

1段			911fxfx		915dd		919MwDa		939nn

统一资料段	极大瞬时风速	风向资料		龙卷,海陆,方位	冰雹直径	

			94917		95VVV		957ww

     出现雷暴		视碍能见度	视程障碍类型

2段			555XX		

补充资料段	指示组		

\subsection{发报标准}

\subsection{发报原则}
\begin{enumerate}
	\item 不定时编发,即观测到编发项目中所列现象达到发报标准时,就应在10分钟内编发出重要天气报告。
	\item 当同时有两种或两种以上重要天气现象达到发报标准(包括前一种现象的报还没有发出,又有另一种或几种现象达到发报标准)时,合并编发一份重要天气报告,各有关电码组一一编发。此时,0段中的GGgg编报最后一种现象达到发报标准的时间。
	\item 在08、14、20时整点前半小时(31~00分)内观测到大风、冰雹现象达到发报标准时,其相关内容合并在正点Z文件中,不另发重要天气报。
	\item 夜间重要天气现象的编发原则
	\begin{enumerate}
		\item 20:01—07:30,出现时间可以确定且在编发时效内的重要天气现象,尽量编发。不能确定具体时间的可不编发。
		\item 由前一日持续至本日20时后的视程障碍现象,以20:01分为发报时间编发;由前一日持续至本日20时后的雷暴以第一声闻雷时间编发。
		\item 20:01—07:30之间未发且持续到07:30之后的重要天气现象,如达到始发或续发标准,则龙卷、视程障碍现象以07:31分为发报时间编发;雷暴以07:30以后第一声闻雷时间编发;大风、冰雹现象合并在08时Z文件中,不单独编发。
	\end{enumerate}
\end{enumerate}

\chapter{自动气象站安装}

\section{温湿度传感器}

装了4支表,左右分别为温度、湿度,前后也装一组,双套设备。离地面1.5m

\subsection{准备器材}

温度传感器、湿度传感器、支架、温度白色套管、湿度黄色保护套、温度信号线、湿度信号线、CAN线、万用表、水平尺、十字螺丝刀  

\subsection{接线}

	铂电阻温度,湿敏电容湿度传感器\(\to\)温湿度分采集器\(\to\)(CAN线)\(\to\)主采集器
\begin{enumerate}
	\item 首先分别将湿度、温度传感器信号线从支架底部穿入支架,左温度,右湿度。保持两个仪器感应部位在同一高度上。(用水平尺打一下)
	\item 将湿度、温度传感器信号线连接温湿度分采集器相应接口上。 分别接到对应的口就行了。
	\item 将CAN线的一端接入温湿度分采集器(电源口),另外一端接主采集器相应端口。插线一定注意不要用蛮力,垂直对准,避免损坏变形。
	\item 测试
	\begin{itemize}
		\item 打开温湿分采集器外壳
		\begin{enumerate}
			\item 拆开外壳,先找到温度传感器端子
			\item 打开万用表,选择电阻档的合适量程如:\(0-400\Omega\)或\(0-200\omega\) 
			\item 测量电阻值
			\begin{itemize}
				\item 测量棕、黑线(或白、黑)之间的电阻值\(\mathrm{R}_1\)
				\item 测量棕、白线(或蓝、黑)之间的电阻值\(\mathrm{R}_2\) 
			\end{itemize}
			\item 计算温度值 Rt= R1- R2  t=(Rt-100)/0.385
		\end{enumerate}
		\item 用电压值估算相对湿度值\par\qquad
		由于湿敏电容传感器的感应量是电容,但在实际应用中,电容量变化较小,不利于测量。为了提高测湿的精度,实际设计中,利用高频振荡电路,使得输出的电信号变为电压值。
		\begin{enumerate}
			\item 合上供电电源,把万用表调到直流0-20V档
			\item 用红黑笔分别测量分采集器湿度端口红(H+)、黄(H-)端子处电压,即为湿度输出电压值。如测出来的电压为0.55V则相对湿度值为55\%。
			\item 棕(12V)、绿线(G)端子处,为湿度传感器供电电压,数值为13V左右 
		\end{enumerate}
	\end{itemize}
	\item 安装温湿分采外壳;取下湿度传感器的防尘罩(黄色);安装完毕
\end{enumerate}

\section{风向风速传感器}

准备器材:风向传感器、风速传感器、风横臂、底座、线缆、内六角扳手、指南针、水平尺、十字螺丝刀、万用表、固定螺母(至少六个)
\begin{enumerate}
	\item 将风横臂置于支架上,调节横臂水平
	\item 用指北针确认正北方向,南北向放置
	\item 将风横臂固定在支架上
	\item 拼装三个半球形风杯,将风速传感器安装到横臂的南侧
	\item 连接风速传感器与风横臂的航空插头,将风速传感器安装到风横臂上并固定
	\item 风向传感器装在北侧,调整风向传感器,使指北针、风横臂、红色指北杆在同一直线上,均指向正北
	\item 连接风向传感器的航空插头,将风向传感器安装到风横臂上并固定
	\item 将风信号线连接到主采集器上
	\item 测试\begin{itemize}
		\item 风速的测试\begin{enumerate}
			\item 在主采集器箱内,测量风速供电电压
			\item 人为持续吹动风杯,在”维护终端”窗口输入“SAMPLES”命令,检查输出的风速数据\par\qquad
			Samples表中,////缺测,表示传感器有故障;---- 表示使能(senst命令)没开该传感器\par\qquad
			SENST:传感器状态查询命令,返回值须为1。传感器代码---气温T0;湿度U;气压P;降水RAT;风速WS;风向WD
			\item 当风杯速度渐缓或停止时,再输入“SAMPLES”命令,检查此时输出的风速数据
		\end{enumerate}
		\item 风向的测试\begin{enumerate}
			\item 人为让风向标固定指向一个位置,如:正北。在”维护终端”窗口输入“SAMPLES”命令,检查输出的风向数据
			\item 调整风向标指向如正南,再输入“SAMPLES”命令,检查此时输出的风向数据
			\item 可以多指向几个方向进行测试
			\item 测量风向供电电压。
		\end{enumerate}
	\end{itemize}
\end{enumerate}

\section{翻斗雨量计}

准备器材:雨量传感器、雨量专用量杯、信号线、万用表、水平尺、螺丝刀、水盆、抹布
\begin{enumerate}
	\item 首先检查外筒清洁程度、过滤网及下方连接铁漏斗是否被泥沙、昆虫等杂物堵塞。
	\item 拿开外筒,检查内部水流通路情况,检查是否被堵塞。
	\item 用螺丝刀或人为倒水检查各个翻斗翻动情况,查看是否卡滞,禁止用手触摸
	\item 测试\par\qquad
	使用万用表的导通档,用红黑笔分别连接信号线接线柱,检查导通次数与翻斗次数是否一致。如倒入2mm的水,蜂鸣档应响20下,否则更换干簧管
	\item 误差调整\par\qquad
	信号线两端分别接传感器和主采集箱。倒入定量水量,查看降水量示值。最大误差:\(\pm0.4\mathrm{mm}\)(降水量\(\leq10\mathrm{mm}\))\(\pm4\)\%(降水量\(\geq10\mathrm{mm}\))如果雨量示值偏大,将两只容量调整螺钉沿轴线向外旋转,旋转一只容量螺钉一周,可使示值减少0.3mm,同时旋转两只容量螺钉一周,可使示值减少0.6mm
	\item 检查底座是否水平,如果不平,则利用底座螺旋调整水平,安装外筒,固定螺母,并注意检查外筒水平情况
\end{enumerate}

\chapter{仪器维护}

\section{激光云高仪}
\begin{enumerate}
	\item 定期检查仪器是否正常工作。
	\item 及时清理蜘蛛网、鸟窝、灰尘、树枝、树叶等影响数据采集的杂物。
	\item 定期清洁光学透镜、可见光成像子系统、红外成像子系统光学部件及环境感应装置等可根据设备附近环境的情况,延长或缩短擦拭透镜的时间间隔,遇沙尘、降雨(雪)等天气现象时,应及时清洁。
	\item 每月检查供电设施,保证供电安全。
	\item 每年春季对防雷设施进行全面检查,复测接地电阻。
	\item 激光云高仪的校准周期应不超过2年。
	\item 当设备故障时应及时进行维护维修。
\end{enumerate}

\end{document}

\section{前向散射能见度仪}
\begin{enumerate}
	\item 切忌长时间直视发射端镜头,避免损伤眼睛;应避免用手电筒等光源直接照射能见度仪采样区域。
	\item 及时清理传感器附近(尤其是采样区)的蜘蛛网、鸟窝、灰尘、树枝、树叶等影响数据采集的杂物。
	\item 出现大风、沙尘、降雨(雪)等影响能见度的天气后,应及时清洁。
	\item 定期清洁传感器透镜,清洁时,用柔软不起毛的棉布或脱脂棉沾无水乙醇擦拭透镜。
	\item 每月检查供电设施,保证供电安全。每3个月要对蓄电池进行充放电1次。
	\item 每年春季对防雷设施进行全面检查,复测接地电阻。
\end{enumerate}

\section{降水天气现象仪}
\begin{enumerate}
	\item 维护时应关闭传感器,如不能关闭,需戴上保护镜,勿直视激光器,以免损伤眼睛。
	\item 定期检查降水现象仪,发现采样区有蜘蛛网、鸟窝、灰尘、树枝、树叶等影响数据采集的杂物,应及时清理。
	\item 每月检查供电设施,保证供电安全。
	\item 每三个月定期清洁激光发射和接收装置,用柔软不起毛的棉布或脱脂棉沾无水乙醇擦试窗口玻璃,注意不要划伤玻璃表面,如果窗口加热功能良好,其表面将很快变干,勿用其他物品清洁。根据设备附近环境的情况,延长或缩短维护的时间间隔,遇沙尘、降雨(雪)等影响观测时,应及时清洁。
	\item 每年春季对防雷设施进行全面检查,复测接地电阻。
	\item 按业务要求定期进行现场核查。
\end{enumerate}

\section{气压传感器}
\begin{enumerate}
	\item 安装或更换气压传感器应在断电状态下进行。
	\item 气压传感器应避免阳光的直接照射和风的直接吹拂。
	\item 保持静压气孔口畅通,以便正确感应外界大气压力。
	\item 定期查看静压管,若发现异物或破损时应及时处理或更换。
	\item 定期检查干燥剂颜色,若潮湿变色应及时更换。
	\item 每年春季对防雷设施进行全面检查,复测接地电阻。
	\item 按业务要求定期进行检定,检定周期应不超过1年。
	\item 当设备故障时应及时进行维护或维修。
\end{enumerate}

\section{温、湿度传感器}

\subsection{百叶箱的维护}
\begin{enumerate}
	\item 保持洁白,内外箱壁每月至少定期清洁一次,时间以晴天上午为宜。
	\item 只能用湿布擦拭或毛刷刷拭、不能用水洗。
	\item 百叶箱内的温、湿传感器不得移出箱外;清洁百叶箱不能影响观测数据的准确性。
	\item 定期检查百叶箱顶、箱内和壁缝中有无沙尘等影响观测的杂物,若有则用湿布擦拭或毛刷刷拭干净。
	\item 百叶箱内不得存放多余的物品。
\end{enumerate}

\subsection{铂电阻温度传感器的维护}
\begin{enumerate}
	\item 维护时,注意避开正点数据采集;百叶箱门打开时间不宜过长、身体部位尽量远离感应部分以免影响观测数据的准确性。
	\item 定期用干布或毛刷清洁传感器,保持其清洁干燥。
	\item 切勿强烈碰撞感应部位,以免内部铂电阻被打碎而造成永久性损坏。
	\item 定期检查传感器和线缆连接处是否松动。
	\item 按业务要求定期进行检定,检定周期应不超过2年。
	\item 当设备故障时应及时进行维护或维修。
\end{enumerate}

\subsection{湿敏电容温度传感器的维护}
\begin{enumerate}
	\item 维护时,注意避开正点数据采集;百叶箱门打开时间不宜过长、身体部位尽量远离感应部分以免影响观测数据的准确性。
	\item 防止传感器的感应部分附着水、灰尘等污染物,禁止手触摸感应部分。
	\item 定期检查传感器和线缆连接处是否松动。
	\item 每月维护传感器的防护罩,查看滤膜是否需要更换。
	\item 按业务要求定期进行检定,检定周期应不超过1年。
	\item 当设备故障时应及时进行维护或维修。
\end{enumerate}

\subsection{翻斗式雨量传感器}
\begin{enumerate}
	\item 维护期间,应将信号线从传感器上拆下,避免翻斗误翻产生多余的雨量数据。
	\item 定期检查雨量传感器的安装高度,检查传感器底盘上的水平泡,检测器口是否水平、有无变形,发现不符合要求时及时纠正;维护中应避免碰撞承水器的器口,防止器口变形而影响测量准确性。
	\item 定期检查承水器,清除内部进入的杂物,检查过滤网罩,防止异物堵塞进水口。
	\item 定期检查和清除漏斗、翻斗和出水口沉积的泥沙,保证流水畅通,计量准确,可用干净的脱脂毛笔刷洗。翻斗内壁切勿用手触摸,以免沾上油污影响翻斗计量准确性。
	\item 定期检查翻斗翻转的灵活性。发现有阻滞感,应检查翻斗轴向工作游隙是否正常,轴承如有微小的尘沙,可用清水进行清洗;翻斗轴如有变形或磨损,应更换轴承。切勿给轴承加油,以免粘上尘土使轴承磨损。
	\item 结冰期停用翻斗雨量传感器的台站,应将承水器加盖,断开信号线;启用前接回信号线,将盖打开。
	\item 每年春季对防雷设施进行全面检查,复测接地电阻。
	\item 按业务要求定期进行校准,校准周期应不超过1年。
	\item 当设备故障时应及时进行维护或维修。
\end{enumerate}

\subsection{称重式雨量传感器 }
\begin{enumerate}
	\item 维护之前应先断开称重式降水传感器电源,拔下数据线;维护完毕后,再接上数据 线和电源线。
	\item 定期检查承水口水平、高度,检查内筒内液面高度,发现不符合要求时及时纠正。
	\item 定期检查清洁仪器,清除承水口的蜘蛛网及其他堵塞物。如遇有承水口沿被积雪覆 盖,应及时将口沿积雪扫入桶内,口沿以外的积雪及时清除。
	\item 每次较大降水过程后及时检查,防止溢出。
	\item 预计将有沙尘天气但无降水,应及时将桶口加盖;沙尘天气结束后及时取盖。
	\item 降水过程中,因降水量较大可能超过量程时,应在降水间歇期及时排水。
	\item 每月检查供电设施,保证供电安全。
	\item 每年春季对防雷设施进行全面检查,复测接地电阻。
	\item 按业务要求定期进行校准,校准周期应不超过1年。
	\item 当设备故障时应及时进行维护或维修。
\end{enumerate}

\subsection{超声蒸发传感器}
\begin{enumerate}
	\item 维护尽量选择蒸发量小的时段。
	\item 蒸发传感器维护期间,应当暂停蒸发观测,维护完成后,再启动观测。
	\item 定期清洁通风防辐射罩。
	\item 蒸发桶定期清洗换水,检查清理不锈钢测量筒内的异物,一般每月1次。
	\item 蒸发桶内水位过低时应及时加水,水位过高时应及时取水,以免影响测量准确性。
	\item 定期检查蒸发器,如发现不水平、高度不符合要求等,要及时予以纠正。
	\item 超声波蒸发传感器测量精度高,切勿撞击或用手触摸超声传感器的探头。
	\item 每年春季对防雷设施进行全面检查,复测接地电阻。
	\item 按业务要求定期进行检定,检定周期应不超过2年。
	\item 当设备故障时应及时进行维护或维修。
\end{enumerate}

\subsection{地温、草温传感器}
\begin{enumerate}
	\item 雨后及时耙松地面温度和浅层地温传感器场地板结的地表土,保持疏松、平整、无草;保持草面(雪面)温度传感器观测场草株高度不超过10 cm。
	\item 深层地温的观测地段应与观测场地面齐平并保持同样的下垫面。若有洼陷,应及时垫平并移植与观测场现有草层同高的草层。
	\item 保持地面温度传感器和草面(雪面)传感器的清洁干燥,当有露、霜或灰尘等附着时,宜在早晨用干布或毛刷清理干净。保持深层地温传感器套管内干燥。
	\item 测量雪面温度时,保持草面(雪面)传感器始终置于积雪表面上。
	\item 按业务要求定期进行检定,检定周期应不超过2年。
	\item 当设备故障时应及时进行维护或维修。
\end{enumerate}

\subsection{雪深传感器}
\begin{enumerate}
	\item 入冬前,应检查雪深仪供电、防雷接地、数据线连接等情况。平整好雪深观测地段,清除杂草,标定基准面,校准测距探头的高度。
	\item 雪深仪工作期间,定期检查设备外观、运行情况,保持基准面平整,禁止任何物体进入观测区域。定期检查超声波测距探头干燥剂,若失效应及时更换;定期检查并保持激光测距探头的清洁。
	\item 雪深仪长时间不用时,断开电源线和数据线;清洁激光测距探头,加防护罩,定期给蓄电池充放电。
	\item 每月检查供电设施,保证供电安全。
	\item 每年春季对防雷设施进行全面检查,复测接地电阻。
	\item 按业务要求定期进行检定,检定周期应不超过1年。
	\item 当设备故障时应及时进行维护或维修。
\end{enumerate}

\subsection{辐射传感器}

\subsubsection{全自动太阳跟踪器的维护}
\begin{enumerate}
	\item 定期在日出后和午后1~2h内检查仪器运行状态
	\item 检查四象限传感器玻璃窗口是否清洁,如有灰尘、水汽凝结物应及时用柔软不起毛的棉布或脱脂棉沾无水乙醇擦净。如窗口内侧有水汽凝结,应及时更换。
	\item 检查四象限传感器能否对准太阳,遮光球能否精确遮光。检查跟踪器跟踪状态,如果跟踪不准,应及时进行调整。
	\item 检查仪器是否稳固、紧固件有无松动脱落,运动部件是否卡滞,内部传动机构是否有异响。
	\item 检查电缆是否有损伤,连接是否可靠。
\end{enumerate}

\subsubsection{辐射传感器的维护}
\begin{enumerate}
	\item 定期在日出后和午后1~2 h内辐射表进行检查和维护
	\item 检查石英玻璃窗口是否清洁,如有灰尘、水汽凝结物应及时用软布沾酒精擦净。如窗口内侧有水汽凝结现象,应及时更换备份直接辐射表。
	\item 通过直接辐射表的瞄准器小孔投影光点检查跟踪情况。如果跟踪不准,应及时进行调整。
	\item 传感器导线如果一直处于运动状态,容易损伤,定期检查电缆是否有损伤,连接是否可靠。
	\item 为保持光筒中空气干燥,应定期检查干燥器内硅胶是否受潮,发现硅胶失效要及时更换。
	\item 每年春季对防雷设施进行全面检查,复测接地电阻。
\end{enumerate}

按业务要求定期进行检定,检定周期应不超过2年。

\subsection{光电式数字日照传感器}
\begin{enumerate}
	\item 定期检查光筒玻璃罩是否清洁,如有灰尘、雨、雪、水汽凝结物应及时用软布将光筒擦净。
	\item 每周定期检查日照计内干燥剂状况,注意及时更换。
	\item 定期查看设备的各个部分是否被腐蚀或者自然损坏,如果有损坏或者腐蚀应当立即进行处理、更换相关部件。
	\item 每月检查仪器的水平、方位、纬度等是否正确,发现问题,及时纠正。
	\item 每月检查供电设施,保证供电安全。
	\item 每年春季对防雷设施进行全面检查,复测接地电阻。
	\item 按业务要求定期进行校准,校准周期应不超过2年。
	\item 当设备故障时应及时进行维护或维修。
\end{enumerate}

\subsection{风向、风速传感器}
\begin{enumerate}
	\item 定期巡查风向标、风杯转动是否灵活、平稳。
	\item 台风、冰雹、冻雨等恶劣天气可能会造成风标板、风杯或轴承变形,致使传感器转动不灵活,强低温雨雪天气可能会使风向传感器冻结。出现上述天气时,要密切观察传感器工作情况,发现异常(如风向长时间在一个范围稳定不变或少变、风速长时间为0 m/s),应及时处理。
	\item 每年定期维护1次传感器,检查、校准风向标指北方位,当风向传感器指北标识模糊时,可用油性记号笔重新标示。
	\item 每年春季对防雷设施进行全面检查,复测接地电阻。
	\item 按业务要求定期进行检定,检定周期应不超过2年。
	\item 当设备故障时应及时进行维护或维修。
\end{enumerate}

\end{document}

