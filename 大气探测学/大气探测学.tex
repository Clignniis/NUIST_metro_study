\documentclass[UTF8,11pt]{ctexbook}
%数学符号
\usepackage{amsmath,amssymb,amsfonts}
\usepackage{tikz}
%合并单元格
\usepackage{multirow}
\usepackage{geometry}
%三线表
\usepackage{booktabs}
\usepackage[toc]{multitoc}%双栏目录
%颜色设置
\usepackage{xcolor}
\definecolor{nuist}{RGB}{0, 103, 156}%南信大蓝
\definecolor{sky}{RGB}{101, 170, 221}%天空蓝
\definecolor{tech}{RGB}{38, 96, 173}%科技蓝
\definecolor{gold}{RGB}{201, 160, 99}%高贵金
%引用
\usepackage[colorlinks]{hyperref}
%设置页眉页脚
\usepackage{fancyhdr}
\pagestyle{fancy}
\renewcommand{\sectionmark}[1]{\markright{\thesection\ #1}}
\fancyhf{}
\fancyfoot[CO]{\href{https://github.com/ZhangtongCN}{\textcolor{nuist}{\thepage}}}
\fancyfoot[CE]{\href{https://github.com/Clignniis}{\textcolor{nuist}{\thepage}}}
\renewcommand{\headrulewidth}{0pt}
%章节设置
\ctexset{
    chapter={pagestyle=fancy},%使得章节页页眉页脚格式一致
}
\geometry{a4paper,top=2.5cm,bottom=2.5cm}
\setcounter{tocdepth}{1}
\linespread{1.3}\selectfont

\begin{document}

\frontmatter

\newgeometry{scale=0.7}

\begin{titlepage}
    \begin{tikzpicture}[remember picture, overlay]
        \coordinate (A) at ([yshift=0.382\textheight]current page.south west);
        \coordinate (B) at (current page.north east);
        \fill[tech](A) rectangle (B);
        \node[xscale=-1] at (B){\includegraphics[scale=2]{1.eps}};
        \node[anchor= north east] at (B){\includegraphics[scale=2]{1.eps}};
        \node[anchor= north west] at (A){\includegraphics[scale=0.7]{2.eps}};
        \node at ([shift={(0,1)}]current page.center){
            \begin{tikzpicture}
                \foreach \a/\c in {0/90, 120/80, 240/70} {%
                    \path[fill=nuist!\c!sky] (0+\a :2.309)--++(150+\a :5)--++(270+\a :7)--++(150+\a :1)--++(90+\a :8)--++(330+\a :7)--cycle;
                }
            \end{tikzpicture}
        };
        \node[gold] at ([shift={(0,8)}]current page.center){\fontsize{72}{0}\heiti 大气探测学};
        \node at ([shift={(0,-5)}]current page.center){\Huge\heiti\href{https://github.com/ZhangtongCN}{\textcolor{white}{Tong Zhang}}\quad\href{https://github.com/Clignniis}{\textcolor{white}{Cls}}};
        \node[tech, anchor= west] at ([shift={(-3,4)}]current page.south){\fontsize{48}{0}\heiti 笔记整理与汇总};
    \end{tikzpicture}
\end{titlepage}

\restoregeometry

\chapter*{版权声明}

本《大气探测学:笔记整理与汇总》一册电子版遵循有限的知识共享许可协议。本书授权包含署名-非商业性使用-相同方式共享(CC BY-NC-SA)。即您被允许在授权范围内对该电子书进行转载、节选、二次创作,但不得用于任何商业目的,且使用时须署原作名,且必须采用与本创作相同的协议(CC BY-NC-SA)进行授权。限于编者水平,本书难免有疏漏错误,敬请读者批评指正。

\begin{tikzpicture}[remember picture, overlay]
    \node [opacity=0.2] at (current page.center){
        \begin{tikzpicture}
            \path[fill=sky] (0:0)--++(90:2)--++(30:6)--++(150:6)--++(210:6)--++(330:6)--++(90:1)--++(150:4)--++(30:4)--++(330:4)--++(210:4)--++(270:3)--++(150:1)--++(210:3)--++(270:1)--++(30:5)--++(330:3)--++(270:1)--++(210:4)--++(150:4)--++(210:1)--++(330:5)--++(30:5)--cycle;
            \path[fill=cyan!50] (0:0)--++(270:1)--++(210:3)--++(150:1)--++(90:5)--++(150:1)--++(270:6)--++(30:6)--++(90:1)--++(30:4)--++(270:6)--++(210:5)--++(270:1)--++(30:6)--++(90:8)--++(210:6)--cycle;
            \path[fill=sky] (90:6)--++(90:1)--++(210:4)--++(330:1)--++(30:3)--++(90:1);
            \path[fill=nuist] (0:0)--++(270:1)--++(330:3)--++(30:1)--++(90:5)--++(30:1)--++(270:6)--++(150:6)--++(90:1)--++(150:4)--++(270:6)--++(330:5)--++(270:1)--++(150:6)--++(90:8)--++(330:6)--cycle;
            \path[fill=nuist] (90:6)--++(90:1)--++(330:4)--++(210:1)--++(150:3)--++(90:1);
        \end{tikzpicture}
    };
\end{tikzpicture}

\tableofcontents

\mainmatter

\chapter{绪论}

\subsection{书面定义}

表征气象要素、天气现象及其变化过程进行个别或系统的,连续的观察和测定,并对获得的记录进行整理分析。

反演:将电信号反算为大气要素指标。

\subsection{作用与地位}

重要分支,研究基础(大气探测引领大气科学),发展前沿,形成多学科交叉融合独立学科,使理论与技术相容。

\section{基本要素}

\subsection{对流层大气分层结构}

大气边界层(大气与地面充满湍流,约1km)、近地面层(100m,注重制备与人类影响,粗糙度)

\subsection{运动基本特点}

空间范围广,运动无休止,影响因素多,运动状态复杂。

尺度:微小尺度<100m,小尺度(局地尺度)100-3km,中尺度3km-300km(台风),大尺度300km-3000km(锋面系统),行星尺度大于3000km(罗斯贝波)。

\subsection{常规分类}
\begin{enumerate}
    \item 
    \begin{enumerate}
        \item 近地层大气探测:地面观测-3-10m:云,能见度,天气现象,地温,温度,湿度,压力,风速,风向,降水,蒸发,辐射。
        \item 近地面观测0-3000m:雷达降水。气象铁塔约200-300m。
    \end{enumerate}
    \item 高空大气探测:3km-30km,探空气球,温湿压。无线电探空系统:提供高空风速测量对比参考。江宁基准气候站,L波段雷达跟踪。
    \item 专业性大气探测:边界层、热岛、风场,极光等。海洋气象浮标。
\end{enumerate}

\section{发展简史}
\begin{enumerate}
    \item 创世阶段:对大气现象的认知,感官为主,经验性观测,气象谚语。
    \item 地面气象观测开始发展时期:欧洲中世纪,出现探测大气的仪器。伽利略气温计,托里拆利气压计,毛发湿度计。
    \item 高空大气探测开始发展时期:1783开始法国人利用氢气球,需要仪器回收。
    \item 高空大气探测迅速发展时期:发明无线电,向高空发展的第一次突破。
    \item 大气探测遥感时期:探空火箭,测风雷达,200-500km以下,高空探测第二次突破;上升探测与返回探测。
    \item 遥感时期:卫星技术,第三次突破,即探测范围的突破 欧洲MRT、风云(奇数极轨135,余下静止卫星)、美国占30\%
\end{enumerate}

\subsection{发展趋势}

人工观察→自动化观察;模拟→数字化;地基→空天基;计算机→人工智能。

我国发展历程:自立阶段→恢复发展阶段1967~1989→现代化建设阶段1990。

\subsection{新一代天气雷达观测网}

大型天气雷达:主要S,C波段,气象楼顶C波段双偏振多普勒雷达,波长3cm,S波段8cm。波长越长对更大的离子更敏感,所以南方用S波段。有218部,但分布不均。

\section{气象观测站分类}

\subsection{3层7类18种}
\begin{itemize}
    \item 基准气候站:214部,等级最高,最先进。
    \item 基本气象站:600+。
    \item 常规气象观测站:10000-,配备设备少。
    \item (区域自动气象站):60000~ 【自动检测异常情况】。
    \item 高空气象观测站:120+个。
\end{itemize}

\subsection{地面观测}

应用气象观测站、志愿气象站,大气本地站(29个,基准站),农业气象站,试验基地。

\subsection{高空层}

120个。

\subsection{空间层}

空间观测。


\section{探测原理分类}

\subsection{人工目测}

天气现象,云状。

\subsection{直接探测}

直接放入大气介质中探测,原位测量。

\subsection{遥感探测}

仪器与被测目标相隔一定距离,例如卫星,雷达
\begin{enumerate}
    \item 主动式:探测器本身主动发射某一波长的电磁波,例如雷达,声纳,激光。
    \item 被动式:接受来自于被测目标的电磁波,例如微波辐射计(水汽发射之远红外信号)。
\end{enumerate}

\subsection{其他探测}

拖放示踪物质探测,模拟探测(水槽、风洞)。

\subsection{气象仪器属性与误差}

\subsubsection{一般属性}

精确度、灵敏度、惯性(滞后效应,响应能力快慢)、分辨率、量程等。

\subsubsection{误差}
\begin{itemize}
    \item 系统误差:系统性的,设备本身存在的问题。有仪器差(可通过仪器订正)、条件误差(可通过订正或选取适当的条件减小误差(水银气压室进入后等待一会儿再读数))、读数误差(可以通过统一的观测方法解决)。能测量,能消除。
    \item 随机误差:正态分布随机出现,大量取平均消除。
    \item 过失误差:由于观测者失误导致的错误。
\end{itemize}

\section{三性要求}

\subsection{大气运动特点}

具有局地性、随时间迅速脉动变化、微分布存在差异、变化比较缓慢。

\subsection{代表性}

【前提,最重要】仪器不仅反映局地情况,也代表一定时空范围的气象状况。(假设一栋楼,同时降雨刮风,楼前楼后均有雨量桶,所以都不具有代表性。所以观测站应建立在空旷区域以确保平均性)
		
空间代表性和时间代表性(只有同时性才有可比性),常规业务站每区一个。一般平原代表性好,城市、山区台站资料代表性差。同时随用途需求而变化。

\subsection{准确性}

准确性归结为仪器的精度(需要考虑代表性),例如使用高精度仪器来说,对一定时间内的平均值来说,其代表性就很差。一般来说,只要获取的资料能满足实际工作的进度要求就认为是准确的。

\subsection{可比较性}

指同一测站不同时间可以对比,不同测站同一时间可以对比。建立在一致的基础上,要求观测时间,方法,仪器类型,观测规范,台站地理地形地貌等一致。即使是单个仪器,其也可能受各种条件影响,如温度,湿度等等。不仅看两个仪器,也指同个仪器的可比较性。


\chapter{云的观测}

\section{概述}

\subsection{定义}

云是悬浮在空中的小水滴或冰晶微粒,或两者混合组成的可见集合体

有时也包含一些较大的云滴,冰或雪粒。云的底部不接触地面。

\subsection{意义}
\begin{enumerate}
    \item 反应物理状况。其外形,数量,分布,移动和变化。
    \item 预示未来天气变化的重要征兆。云的形成与演变是大气中运动的具体表现。(凝结核:破坏张力)
    \item 云与地气系统能量平衡(云自身能发生辐射,也能反射太阳辐射,能吸收地面长波辐射等等,对辐射贡献的正负性仍然较难判断,尤其是冰云,其散射很难判断(目前发现有30多种冰结构))
    \item 云与人工影响天气,云与航空安全。
\end{enumerate}

\subsection{观测云的任务}

云状 云量 云高


\section{云状的分类}

三族十属二十九类
\subsection{三族:(按云底高度划分)}
\begin{enumerate}
    \item 低云 	<2500		赤道地区相对更高
    \item 中云 	2500-6000	
    \item 高云	>6000	
\end{enumerate}

\subsection{十属:(按形成机理分)}

\subsection{二十九类:(按外形特征分类)}

\subsection{WHO分类}

分为3族、10属、31类;将雨层云划分在了中云族

\section{积云 Cu}

\subsection{淡积云}

下方相对平坦,中间有黑影,形似花菜,横向宽度大于纵向深度,扁平

\subsection{碎积云}

淡积云消散或风吹时,边缘有不规则的边界,很薄很碎。

\subsection{浓积云}

纵向深度高,厚度可能会超过宽度,臃肿高耸,顶部开始有向上的突起。由淡积云增长而来。浓积云一般为单个出现。

\subsection{形成机理}

靠对流形成,在阳光照射下地面增温使空气变热,气块密度降低,低于背景大气密度导致上升,上升过程为绝热,气块对外界做功,温度下降,直到凝结温度,热对流导致积云产生 在不稳定层结的空气中,由于热力和动力产生对流作用,使水汽凝结而成的云

一团空气上升,开始时其内部水汽含量和温度的水平分布基本均匀,从而水汽产生凝结的高度是一致的,因此,积云通常拥有平整水平的底部。由于在形成阶段,云内为上升气流且云顶中央上升气流最强,四周较弱,云外为下沉气流,造成积云具有圆拱形向上突起的顶部与明显的轮廓。

\subsection{淡积云}

抬升高度取决于背景大气的稳定程度。不稳定时斜率高,稳定时斜率低。最高上升至对流层顶。

\subsection{影响}

旺盛的积云可降小阵雨

\section{积雨云 Cb}

是阵性降水云,有时有冰雹和龙卷风。云顶由冰晶组成,有白色毛丝般光泽的丝缕结构。

\subsection{秃积雨云}

浓积雨云继续向上增长,上方有明显突起,有尖耸突起。是一种过渡阶段。

\subsection{鬃积雨云}

积雨云的成熟阶段,顶部到达对流层顶部,向两边延展,顶部有明显的白色毛丝般冰晶结构,砧状结构。其是一边抽水,一边降水的循环机制。 大量碰撞产生电场相互作用,使正电荷与负电荷产生距离,导致放电。	

\subsection{悬球状云底}

在降水之前,鬃积雨云云底混乱不堪,但有时会有规律,形成悬球状云底,通常预示着降水到来


\section{层积云 Sc}

\subsection{形成机理}

由于空气的波状运动和湍流混合作用使水汽凝结而成。一般表示天气稳定,若层积云逐渐加厚,甚至融合成层,则表示天气将有变化。

\subsection{对天气的影响}

影响日气温变化:白天降低日照,夜晚又阻碍冷空气的扩散,所以形成凉爽、潮湿的天气。

\subsection{透光层积云}

呈现一条一条,波状运动形成的云,云条平滑且透光性差,但云缝多且及明显。

\subsection{蔽光层积云}

云较厚,不透直射光,仅剩散射光,结构与透光层积云基本一致,厚处很黑,薄处不透光。可能会有多个方向波状运动导致混乱,底部有明显起伏。

\subsection{堡状层积云}

积云发展时有逆温层存在,朝有城堡云,午后雨淋淋。若冷锋过境,会导致逆温层减弱减薄甚至消失,使得下方图示中上升运动能够突破逆温层,导致继续对流,在这些位置形成顶部凸起(需要大量出现)。未来可能变为积雨云。局部垂直发展。

\subsection{积云性层积云}

云快大而厚,成条状,其顶部仍然有积云特点,常出现于夏季的傍晚。有很厚的逆温层,导致向两侧扩展。呈灰色条状,表明天气系统趋向稳定

\subsection{荚状层积云}

类似UFO,若有多层表明存在震荡。中间厚边缘薄,个体分明。

\section{层云 St}

\subsection{形成机理}

层云是在气层稳定的情况下,由于夜间强烈的辐射冷却或湍流混合作用使水汽凝结或由雾抬升而成。

\subsection{层云}

低且均匀的云层,像雾但不接地,呈灰色或灰白色。其厚度很薄很低,常见于夏季山间与海岸线,晴朗夏季山区夜晚形成冷却雾,到早晨时地面二次加热,随即地表雾蒸发,雾被抬升。冬季可能维持数日。早晨的雾一定会被抬升

\subsection{碎层云}

不规则的松散碎片,形状多变,呈灰色灰白色,高度非常低(区别于碎积云)。山地可随时生成,一般为层云导致。

\section{雨层云 Ns}

\subsection{雨层云}

厚而均匀的降水云层(至少2-3km),云顶可延伸至4-6km,含水量高,可完全遮蔽日月,呈现暗灰色,底面十分平坦,有连续性降水。冬日持续性降雪也为雨层云。例如江淮准静止锋梅雨

\subsection{形成}

常出现在暖锋云系中,由潮湿空气系统滑升,绝热冷却而成。常由高层云加厚,云底降低而成。

\subsection{碎雨云}

雨层云下方一块一块的小云朵,低而破碎的云,形状多变,灰色,移动速度快,常出现在降水。高度十分低。

\subsection{形成}

自上而下产生,在雨层云的下方形成小的降水,越往下温度越高,从而又蒸发为水汽。

\section{高层云 As}

带有条纹或丝缕结构的云幕,灰白或灰色,可见日月,云底没有明显的起伏

\subsection{透光高层云}

较薄而均匀的云层,呈灰白色,日月轮廓模糊,像透过一层毛玻璃,地物无影(冰水混合云而致)
蔽光高层云:云层较厚,厚度变化较大,看不见日月,呈灰色,有时微带蓝色,有些微微透亮的地方有丝缕结构。

\subsection{形成}

高层云常由卷层云变厚或雨层云变薄而成。多在中纬度地区出现。

\subsection{对天气的影响}

它的出现表明该地区有上升空气。预示着移动的气旋会到达,形成长期固定的降雨或降雪。夏季与风暴或热带气旋有关。

\section{高积云 Ac}

\subsection{透光高积云}

个体明显,排列规则,但各个部分透明度不同,云缝中可见青天。云层不厚,目测云条较窄。注意:其可能与絮状高积云混淆,所以要连续观测。

形成:逆温层,十分稳定。

\subsection{蔽光高积云}

连续,云快深暗而不规则,几乎完全不透光

\subsection{积云性高积云}

由积雨云、浓积云延展而成,常常伴随积云出现。透光性相对好一些,云底不太圆润,预示天气稳定

\subsection{堡状高积云}

外形和堡状层积云很像,但云快较小,高度较高。

\subsection{絮状高积云}

云块边缘部分破碎,像破碎的棉絮团,云块大小不同,高低不同(晴空湍流导致的温度不均匀产生高空对流)晚霞的那种,零碎不成连续片

形成:常由天气系统过境,层级不稳定,温差过大。由于空中潮湿气层不稳定,有强烈乱流,预示有雷雨天气。

\subsection{荚状高积云}

边缘有丝缕结构,中间厚,呈现豆荚状,长产生虹彩。

\section{卷云 Ci}

\subsection{毛卷云}

纤细分散的云,呈现丝条、羽毛、马尾形,是独立的个体,预示着晴天。若厚度增加,可能发展为卷层云。

\subsection{密卷云}

较厚的,成片的卷云,中部时有暗影,边缘部分卷云特征明显,底部易被风吹散,常有雪幡。表示天气稳定。

雪幡:固态降水粒子,从云中降落呈现幕状

\subsection{伪卷云}

鬃积雨云顶部脱离母体而形成,云体大而厚。雨过天晴后出现。

形成:降雨导致的云体下降,导致鬃积雨云顶部砧状结构被留下。

\subsection{钩卷云}

形似逗号,水平,由于垂直风速不同切削而成(风切变),头部有变形,尾部拖尾相当长。将在短期内有阴雨天气(大量出现)。

卷云的形成机制是多样的,仍未有定论,不太好观测(雷达截面小,飞行高度不足),观测资料少。

由于高空对流而形成,往往带有积云的形状;由于卷层云边缘展裂而成;由于高积云抬升转化而成;由高积云所降雪幡残留空中而成。


\section{卷层云 Cs}

卷层云由湿空气作大范围缓慢斜升运动而膨胀冷却所造成,标示了高海拔地区的潮湿地域及大面积的空气上升运动

\subsection{毛卷层云}

不均匀。白色丝缕结构明显,云体厚薄不很均匀 形如大量的毛卷云成层出现 大量的冰晶,有环天顶弧。可编织成网状结构

\subsection{薄幕卷层云}

十分均匀。日月轮廓清楚可见,有晕,地面有影,暖锋过境。或者过冷气候也可形成。

\section{卷积云Cc}

云块很小,白色细鳞片状,常成行、成群排列整齐,很象微风吹拂水面而成的小波纹。由高空层结不稳定产生波动而形成的:大波和小波交错融合,大波通常由底层波上传,波长与振幅不断增大(底层密度大,顶层密度小),导致不稳定,波产生破碎,形成小波与湍流。常有暴风雨天气
三指判断法	若三指能遮挡住云条,就不是层积云;卷积云一个小指头能挡住。

\section{云的形成机理}

\subsection{基本过程}

云的形成过程是空气中的水汽达到饱和或过饱和状态而发生凝结、冻结、凝华的过程。途径主要有:
\begin{enumerate}
    \item 水汽增加
    \item 降温冷却
    \item 同时出现
\end{enumerate}

\subsection{对流云}

由于对流作用,使大气中的水汽上升,达到凝结高度形成的云叫对流云。对流云边界轮廓清晰,底部较平整。

取决于背景大气的减温率,每公里8k,湿绝热6k;取决于对流强度、水汽含量和凝结高度

\subsubsection{形式}

表面加热自由对流、地形被迫抬升、空气辐合引起的大范围抬升、沿锋面的抬升

\subsubsection{冷云}

出现于鬃积雨云,受表面张力影响导致中部冷云形成大量过冷水(<0℃),给冰雹形成提供非常良好的条件。

\subsection{波状云}

大气层受地形扰动抬升,达到凝结高度形成的云

在逆温层附近上升、下沉运动产生的云

\subsection{荚状云}

在局部上升和下沉气流汇合处产生的云。上升气流携带暖湿空气的上升中遇到下沉气流的阻挡,云体不能向上扩展;另外,下沉气流的绝热增温使云体的边缘发生蒸发,导致云体边缘变薄,使云体形成豆荚状

\subsection{堡状云}

空中对流云在低层空气较稳定,而中空不稳定时,空中对流突破逆温层而形成的云

\subsection{絮状云}

与堡状云相似,只不过云中湍流和对流更强一些,致使云层发生形变

\subsection{钩状云}

由于冰晶云的下垂部分因高空风速有较大的垂直切变,并伴有蒸发现象而形成的;意味着大的温度梯度→坏天气

\subsection{锋面云系}

由于气流在锋面的辐合、近地层的摩擦辐合和高空锋区环流的作用下加速辐合,形成大范围的垂直气流

冷锋:积雨云;

暖锋:活跃的暖锋云带最宽,云系序列:卷云(Ci),卷层云(Cs),高层云(As),雨层云(Ns)


\section{云量观测}

云量指云遮蔽天空视野的成数,包括总云量与低云量。总云量记为分子,低云量记为分母,例如10/6
云量记录规则:
\begin{itemize}
    \item 全天无云,总云量记:0
    \item 天空完全为云所遮蔽,记:10
    \item 布满天空但是又有缝隙时记为:\(10^-\)
    \item 全天无低云或虽有少许低云但其量不到十分之零点五时,低云量记:0
    \item 但如有云隙能见到青天或看到上层云时,低云量记:\(10^-\)
\end{itemize}

\subsection{例题}
\begin{enumerate}
    \item 下层为层积云,布满全天,上层为卷积云,布满全天,则为:\(10^-/10^-\)。
    \item 天空有微量毛卷层云,不足1/20,则为:\(0/0\)。
    \item 云布满全天有空隙,毛卷云6成,淡积云2成,层积云2成,则为:\(10^-/4\)。
\end{enumerate}

\subsection{测量方法}

\subsubsection{目测}

补贴法、等分法、球带法。

\subsubsection{器测}

全天空成像仪 适用于白天	 由向下观测的固态CCD成像仪获取被加热的半球形旋转镜面上形成的天空图像进行云量测量的. 只能应用于白天太阳高度角大于5~10°的情况。
\begin{enumerate}
    \item 主体部分由鱼眼镜头、数字CCD相机构成。
    \item 需要一个特殊装置对太阳进行遮挡。
    \item 由于CCD的特性,只能适用于可见光波段,在白天使用。
    \item 半球保护罩要配以加热通风装置,保证表面不不结冰(露)。
    \item 后期需要图像处理来计算全天空的总云量。
\end{enumerate}

为防止过曝:降低分辨率,增大感光面积,50微米大像素。

CCD电荷耦合器件 CCD是一种用电荷量表示信号大小,用耦合方式传输信号的面阵传感器,具有波谱范围宽、畸变小、体积小、重量轻、功耗小、寿命长、可靠性高等一系列优点,并可做成集成度非常高的组合件。

红外云分析仪 适用于夜间	通过扫描来测量天空红外辐射,可用于夜间观测。硅半导体不能满足要求,所以选择别的。
\begin{enumerate}
    \item 使用制冷(降噪,提高信噪比)红外焦平面阵列(InGaAs or XenICs)\\高端成像红外CCD基本无国内产品。
    \item 获取8~14微米波段的全天空向下的云水红外辐射数据。
    \item 昼夜连续进行针对云量、云状、云高自动观测的仪器。
    \item 适用范围广,但成本昂贵。
\end{enumerate}

\section{云高观测}

\subsection{目测}

准确性差,不关注。
\begin{enumerate}
    \item 凭经验
    \item 根据云状估测:根据云状来估测云高,判定云状,同时根据云体结构,云块大小、亮度、颜色、移动速度等情况进行估测。
    \item 根据已知目标物高度估测:当测站附近有山、高的建筑物、塔架等高大目标物时,可以利用这些物体的高度估测云高。
    \item 根据经验公式测算。
\end{enumerate}

积云、积雨云云高(计算露点高度,则可知云高)
\[
H=\frac{t-t_d}{\gamma_d-\gamma_z}\approx124(t-t_d)
\]

式中\(H\)为云高(m),\(t\)为地面气温(℃),\(t_d\)为地面露点温度(℃)。 

\subsection{器测}

\subsubsection{云幕灯}
\[
H=L\tan E
\]

要求\(L\)基线值尽可能长,使仰角较小,一般在1-2km左右。广泛应用于机场。

\subsubsection{激光雷达}

台站观测云高的常规方式。

仪器由发射望远镜、接收望远镜和电子门组成。

当激光通过发射望远镜发射激光的同时由参考脉冲使电子门打开,于是计数电路就对时标脉冲计数。

激光脉冲遇到云层被云滴散射,其中后向散射部分被接收望远镜接收后,通过光电转换系统指令使电子门关闭,计数停止。

计数电路记下从电子门开放到关闭的时间间隔,即为激光在测云仪和被测目标物之间往返一次所经历的时间。通过光速和时间,可以得到云底高度。 

需要脉冲激光器,一秒钟50多个脉冲激光,采集器能够将1s中分为109份,足以精细化判断云高。

云在天空的分布常常是分散的,所以测得的云底高度值与激光云高仪的位置和仰角有关。

为了得到最低云,或者不同位置的云底高度,可以发射仰角不断变换的激光。发射的激光脉冲以0°~85 °仰角旋转。

云底的概念比较模糊,随着激光雷达的波长不同而有不同取值,其为真实为平滑的渐变层。 

\subsubsection{气球测云高}

已知升速的氢气球,观测其从施放到进入云底的时间,乘以气球升速:云底高度=气球升速×(分钟数+秒数/60)。

\chapter{能见度}

\section{能见度与气象光学视程}

\subsection{定义}

视力正常的人,在当时天气条件下,能从天空背景中看到和辨认目标的最大水平距离。可以用气象光学视程量化。

\subsection{有效能见度}

四周视野中二分之一以上范围里都能看到的最大距离

\subsection{影响因素}

观测者视觉性能、天气条件(大气透过率)、目标物与背景性质(对比度)

\subsection{大气透明度}
\[
F=F_0\mathrm{e}^{-\sigma L}
\]

光强会被大气削弱。\(\sigma\)为大气消光系数。路径为\(L\),前后的亮度分别为\(F_0\)和\(F\)

有100份光强,有四层大气,每一层大气消光系数为0.1,最终还剩余\(0.9^4=0.6561\)份。

\subsection{对比度:视亮度对比值}

用来表示目标物和背景之间的亮度差异的指标。

\(C\)在0-1之间,\(C=0\)则物体与背景融为一体,\(C=1\)则目标清晰可见

\subsection{人眼视觉性能}

亮度对比感阈\(\varepsilon\):人眼能够看到的最小的C值,最好的为0.02

\subsection{能见度量化:气象光学视程MOR}

白炽灯发出色温2700k(冷光)平行光束的光通量(单位时间内通过单位面积的能量),在大气中削弱至初始值的5\%所通过的路径长度

可以使用比尔朗伯定律描述。\(F=F_0\mathrm{e}^{-\sigma L}\rightarrow0.05=\mathrm{e}^{-\sigma L}\)

\(\mathrm{MOR}=L_{0.05}=-\ln(0.05)/\sigma=3/\sigma\)

能见度观测以km为单位,取一位小数,不足0.1记为0.0

\section{能见度的人工观察}

\subsection{判断依据}

白天能见度是指视力正常(亮度对比感阈为0.02)的人,在当时气象条件下,能够从天空无云背景中看到和辨认出地面黑色目标物(大小适度,常取视场角为0.5°~5°)的最大水平距离。绘制能见度目标物分布图。

夜间:可以看到中等强度的灯光的最大距离

\section{器测原理}

\subsection{原理}
\begin{itemize}
    \item \begin{enumerate}
        \item 方法一:散射式,测量\(sigma\)。由吸收和散射共同造成,但在短距离内吸收可忽略,则散射\(=\)消光
        \item 方法二:透射式,测量\(T\)。\(\sigma\)测量的不太准(散射理论限制),消除\(\sigma\),联立推导,得到\(\mathrm{MOR}=L(\ln0.05/\ln T)\),\(L\)代表观测距离,\(T\)代表透过率。该方式更趋近于定义。
    \end{enumerate}
\end{itemize}

\subsection{透射式能见度仪}

\subsubsection{原理}

要求有基线\(L\),测得透过率\(T\),则可确定MOR。\(\mathrm{MOR}=L(\ln0.05/\ln T)\)

\subsubsection{组成}

发射器提供经调制的定长平均功率光通量源,接收器由光检测器组成。需求基线长度应尽量长,通常在1km左右。
	
\subsubsection{优点}

采样体积大,测量精度高,是标准仪器:通常作为器测能见度的标准。

\subsubsection{缺陷}

仪器占地面积大,光源探测器间难以保持准确的光轴(大风所致仪器颤动误差大)

\subsection{前向散射式能见度仪}

\subsubsection{原理}

测量\(\sigma\)。光通过大气经历两个过程:吸收与散射\(\sigma=b+c,c\rightarrow0\),散射在短距离内吸收可忽略,则散射\(\approx\)消光 
\(S\cdots \beta(\theta)\),\(S\)为散射截面,\(\beta(\theta)\)为相函数,由散射类型决定

\subsubsection{散射类型}
\begin{itemize}
    \item 瑞利散射(入射光波长>>离子尺度,产生相对对称的散射光,向前向后基本对称,垂直最少,水平最大)【厘米波、分米波】
    \item 米散射(入射光波长~离子尺度,出现多方向散射,总体前向散射大于后向)【可见光、近红外】【一般使用】
\end{itemize}

\subsubsection{散射截面\(S\)}

一个物体越大,其散射能力越强

\subsubsection{组成}

发射器、接收器成一定角度(30-50°)(透射光远大于散射光,所以需要避开正前方向),两束光交叉的地方即为采样空间(1L左右)。测量其前向散射。两者距离一般在1-1.5m之间。发射器与接收器前方有遮光板buffer、遮光环(反射杂散光),避免杂光噪声背景光影响。

\subsubsection{规格}

输出间隔60s,前向散射测量角33°,光源波长930nm

\subsubsection{优点}

体积小,占地空间小,成本低

\subsubsection{缺陷}

代表性弱(采样空间小)

\subsection{侧向散射式能见度仪、后向式能见度仪}

两者均没有前向式的优秀。

后向散射式安装最方便,仪器可以安装在室内,通过窗户向外发射并接收其回波信号,也可制成便携式。

侧向散射式接收较宽视角的散射光,使它能具有较好的代表性。

\subsection{激光雷达}

与后向散射仪类似。接受测量大气后向散射的光信号。

激光雷达发射激光脉冲,激光脉冲在大气中传输时,会因大气中的空气分子和气溶胶粒子而产生散射和吸收。通过接收和测量大气后向散射的光信号,便可以提取出不同距离处空气分子和气溶胶粒子光学参数的有关信息,进一步反演获取大气消光系数和大气水平能见度。激光雷达不仅能测量水平能见度,而且也能测量倾斜能见度和垂直能见度。
\[
P(R)=CR^{-2}\beta\cdot\mathrm{e}^{-2\sigma R}
\]
	
\(P\)为效率,能探测到不同距离上回来的能量值。\(C\)为一常数,\(R\)为距离,与距离平方成反比,\(\beta\)为散射系数,成幂指数变化。

两边取对数,对\(R\)求导。\(P(R)\)斜率即为\(-2\sigma\)。

探测距离近,成本昂贵。

\subsection{CCD摄像式}

高速公路上常用,实质上即为摄像机。在一端放置摄像机,另一端放置目标板。

目标板为黑色,测量黑板与天空的对比度
\[
V_h=\frac{-3.912R}{\ln(1-B_t/B_s)}
\]

\(B_t/B_s\)为天空亮度与黑板亮度。受直射光影响大,准确度低,不能用于气象站。

\chapter{天气现象}

\section{各类天气现象特征}

有降水现象、地面凝结现象、视程障碍现象、雷电现象、特征风、大气光学现象。

\subsubsection{降水}

阵雨、雨、毛毛雨、雪、米雪
\begin{itemize}
    \item 冰雹:鬃积雨云中慢慢生长,由内向外生长。
    \item 冰粒:过冷水,下降过程中外界温度过冷,由外向内冻结,内部会有水,空心。
    \item 霰:原先形状规则,有雪花,冰的结构,但下降过程中接触了过冷水,冰晶碰撞,造成非常不规则的团状物体。
    \item 云体零度层:内含有大量的霰。
\end{itemize}

\subsection{地面凝结}
\begin{itemize}
    \item 露:气态变为液态。夜间辐射冷却。
    \item 霜:气态变为固态,薄薄的一层,更多出现在无风时,各方向生长。
    \item 雾凇:气态变为固态,有风,多出现于树上,有方向性:迎风生长。
    \item 雨凇:液态变为固态。过冷却液态降水,接触地面后结冰。
\end{itemize}

\subsection{视程障碍}
\begin{itemize}
    \item 雾:  <1km 雾(0.5-1)浓雾(0.05-0.5)强浓雾(0.05)。
    \item 轻雾:1-10km。
    \item 吹雪:把地面的雪吹起。直接影响飞机起降。
    \item 雪暴:<1km,白茫茫。
    \item 扬沙、沙尘暴。
    \item 浮尘:成因与吹雪类似,吹起。
    \item 霾:  干霾发黄,加上雾发灰。
    \item 烟尘:燃烧或化学反应散布于空中的极小固体微粒。
\end{itemize}

\subsection{雷电现象}
\begin{itemize}
    \item 雷暴、闪电。
    \item 极光:大量太阳抛射质子流,而地球磁场赤道稀疏,两级密集,在两级部分受密集洛伦兹力作用,与大气分子碰撞,激发恢复基态发光。\\绿光红光在不同高度,中纬度氮气发光更低。
\end{itemize}

\subsection{特征风}
\begin{itemize}
    \item 大风:8级以上,17.2m/s。
    \item 飑:突然发作的强风。瞬时风速突增,风向突变,气象要素剧烈变化,常随雷雨出现。
    \item 龙卷,尘卷:尘卷1-3m直径,龙卷>3m。出现于鬃积雨云云底。
\end{itemize}

\subsection{其他天气现象}
\begin{itemize}
    \item 冰针:漂浮于空中的微小片状针状冰晶,多出现于高纬度或高原。
    \item 结冰、积雪。
\end{itemize}

\subsection{光学}
\begin{itemize}
    \item 虹:日月光经云滴折射反射形成的彩虹大弧,出现于日月相反方向,可能由霓。
    \item 晕:多出现于卷层云上,日月光圈、光柱、光弧的总称。
    \item 华:日月光经过云滴,由衍射形成的环绕日月光轮外的彩色。
    \item 霞:清晨傍晚天空的色彩现象。
    \item 海市蜃楼。
\end{itemize}

\section{天气现象的器测}

\subsection{原理}

降水类型的识别:降水微粒的大小、形状、降落速度和浓度导致了光信号的强度、多样性、频率特征。

\subsection{天气现象综合观测仪}

能够对能见度以及引起能见度变化的天气现象、降水类型、降水量和降水强度等进行探测。
	

这些仪器所采用的技术主要是光学技术,利用降水粒子对不同波段光的散射、衰减、吸收等特性来实现对降水类型、强度和降水量的探测。

\chapter{温度测量}

\section{概述}

温度是表征物体冷热程度的物理量,是一个宏观量。

\subsection{微观}

反应物体系统中大量分子热运动的激烈程度或平均动能大小。

\subsection{宏观}

温度概念建立在热平衡基础上,当两个物体系统通过热交换达到动态平衡,具有一个共同的温度。

\subsubsection{温标}
\begin{itemize}
    \item 开尔文温标:\(\mathrm{K}={^\circ\mathrm{C}}+273.15\)
    \item 摄氏温标
    \item 华氏温标:\(^\circ\mathrm{F}=\frac{9}{5}{^\circ\mathrm{C}}+32\)
\end{itemize}

\subsection{观测项目}

气温:日最高,日最低(定时)。

低温:地面最高,地面最低,浅层低温5、10、15、20cm,深层地温40、80、160、320cm,草面温度(定时)。

\section{玻璃液体温度计}

\subsection{基本原理}

充满液体的玻璃球,温度变化,引起测温液体体积膨胀收缩。

\subsection{常见液体介质性质}
\begin{table}[htbp]
    \centering
    \begin{tabular}{|*{5}{c|}}
        \hline
        介质 & 膨胀系数 & 熔点 & 沸点 & 比热容\\
        \hline
        水 & 0.21 & 0 & 100 & 4200\\
        \hline
        水银 & 0.18 & -39 & 357 & 0.14\\
        \hline
        酒精 & 1.1 & -114 & 78 & 2400\\
        \hline
    \end{tabular}
\end{table}

\subsection{不选用水的原因}
\begin{enumerate}
    \item 水的温度测量范围有限,在\(0^\circ C\)到\(100\circ C\)。
    \item 水的比热容是水印的30倍。
    \item 同样体积的液体升高\(1^\circ C\),酒精膨胀的体积约是水的5倍。
    \item 水有反常膨胀现象。
\end{enumerate}

\subsection{水印温度计的其他优势}
\begin{enumerate}
    \item 水银金属颜色容易读出。
    \item 水印的温度变化的线性关系理想。
    \item 水银的金属性质使它的蒸汽压很小,因温度变化而使温度计内液体蒸发量以及因为液体蒸发导致管内压力上升的危险性也小。
\end{enumerate}

这才是水印被遴选为最佳温度计物质的原因,像煤油或酒精有一定的挥发性,温度未到沸点就开始变气态了。

\subsection{种类}

干湿球,最高,最低,地面,曲管,直管。

\subsection{百叶箱位置}

干球左,湿球右,最高上,最低下。

\subsection{干湿球温度表}

作用:用于测量空气温度与湿度。

原理:湿球蒸发吸热,温度总低于干球,温差越大,温度越低。

\subsection{最高温度表}

\subsubsection{原理}

毛细孔,温度升高水银膨胀,水银面向右端移动;温度降低,左端截面大,流量大,有段截面小,流量小,中间出现真空层,流动停止,液面无法回缩。在狭管处的摩擦力超过了水银的内聚力,水银就在此中断。

\subsubsection{放置}

右端高左端1cm,倾斜放置,不让水银由于重力作用出现自然断裂。

\subsection{最低温度表}

\subsubsection{原理}

使用酒精,酒精的表面张力作用很大,表面的膜难以突破。内部有一小游标(铝合金),温度升高,酒精膨胀,酒精无法携带游标移动。但温度降低时,由于酒精液面的表面张力比游标对管壁的摩擦力要大,酒精右端液面在表面张力作用下会拉动游标向左,并停止在最低温度位置。


\subsubsection{复位}

抬高感应部分即可。

\subsection{地温}

定义:地面表层土壤的温度为地面温度。是大气与地表结合部的温度状况。

特点:有明显的日、季节性周期性变化。

\subsection{地面温度表}

\subsubsection{特点}

刻度与量程较大,可以到80℃;在土壤中安放方式:感应球部一半土壤,一半气温,实际测得为平均温度。最下方为定时。

\subsubsection{包括}

地面最高温度表、地面最低温度表、定时地温表。

\subsection{浅层地温表(曲管地温表)}

\subsubsection{原理}

5、10、15、20cm 感应部分上端弯折,与表身呈现135°夹角(方便读数)由东向西、从左到右由浅入深。

\subsubsection{深层地温表(直管地温表)}

\subsubsection{原理}

40、80、160、320cm 装在带有铜底帽的管型保护框。由东向西排列,由浅而深。
\begin{enumerate}
    \item 使用碎棉花木屑等,使温度表具有必要的滞后性,用于缓冲(冬季寒冷天气,让读数时温度变化速率低)
    \item 木头杆上有多个小绒圈,隔断空气,防止管内空气流动导致的温度混合。
\end{enumerate}

\subsection{双金属片自动温度计}

\subsubsection{感应元件}

双金属片(一般为铜片/锌片或铝片/铁片),有膨胀系数相差较大的两片金属焊接而成(耐温膨胀合金)。

\subsubsection{原理}

将金属片一端固定,另一端随温度变化产生位移,带动杠杆出现摆动(传递放大),杠杆在自记钟上记录(发条驱动24h一周)。

\section{电学温度传感器}

有热电偶、金属电阻、热敏电阻、石英、红外。

\subsection{热电偶温度传感器 Thermocouple}

\subsubsection{基本原理}

温差电现象(贝塞克效应):使用两根不同导电率的金属丝,一根纯铜,一根掺了杂质的铜(康铜),形成闭合回路。

若两个接点在不同温度下,可见电流。 电子在热端聚集多,电势差增大,产生温差电动势。

电动势\(E=a(T_1-T_2)+b(T_1-T_2)^2\),其中\(a\gg b\),可近似为\(E=a(T_1-T_2)\)

使用康铜原因:
\begin{enumerate}
    \item 灵敏度高
    \item 稳定性好
    \item 工艺简便
    \item 成本低
\end{enumerate}

\subsubsection{应用}

测量辐射差值。

\subsubsection{缺陷}

测温时参考端温度固定(例如参考端0°时线性关系好,但参考端温度变化后线性曲线发生变化),参考端温度的稳定程度直接影响观测的一致性与测量精度。

\subsection{金属电阻温度传感器 RTD}

\subsubsection{原理}

金属导体的电阻值与温度呈现一定的函数关系。温度升高引起金属晶格振动加剧,影响自由电子有序快速移动,导致电子迁移率降低。

大多数金属电阻温度上升1K时,其电阻值增大0.4\%-0.6\%,约200mV。

\subsubsection{应用}

其量程很大,广泛用于测量-200-850℃范围,特殊情况下,量程可扩展至1K-1250K。

目前广泛应用于业务环境,地温场等。

\subsubsection{特点}

线性度良好
\[
R_T=R_0(1+\alpha(T-T_0))
\]

\(\alpha\)为金属电阻在\(T_0\)时的温度系数。

\subsubsection{选材}

铂电阻(灵敏度高、线性度好,化学性质稳定,不会生锈)、铜电阻(成本低),价格低廉。
	
\subsection{热敏电阻温度传感器 Thermistor}

\subsubsection{原理}
\[
R=a\cdot\mathrm{e}^{\frac{b}{T}}
\]
温度越高,阻值越低(负温效应)。吸收热能,会增加感生载流子,增强导电能力,电阻值降低。

材料使用半导体热敏电阻\(R_T\)(负温度系数热敏电阻)。

\subsubsection{应用}

一般阻值大便于测量,适应于测量地温,可探空,灵敏性适应能力好。

\subsubsection{优点}

感应时间常数小、体积小,重量轻、电阻输出变化量较大、便于生产,成本低廉、珠状热敏电阻误差小,机械强度也比棒状好。

运动中测量的特点、热惯性角度、测量分辨力角度、成本和生产角度。

\subsubsection{缺陷}

数据过于灵敏,会抖动。

\subsection{红外测温}

原理:黑体辐射,与探测视场范围有极大关系。

\subsection{石英传感器}

原理:有石英晶体,其具有自身固有的振动频率,这个频率会随温度、压力、加速度而变化。在切割石英时,不按照晶面切割,沿与晶阵纵向成一定角度切得晶体破坏晶体完整性,其振荡频率具有较大的温度系数。

应用:胎压检测

\subsection{地温变化行业应用}

土体含水量、浅层土体破坏形式变化、建筑材料腐蚀性影响、地温应力变化影响。

\section{热滞效应}

\subsection{概念定义}

需要使测温元件与被测对象想接触并建立热平衡。这种热交换需要一个过程,测温元件对温度的变化的响应总是滞后的。

\subsection{热滞误差}

热滞效应将引起测温误差。

\subsection{定量描述(热滞系数)}

描述温度随时间变化的滞后过程。\(\frac{\mathrm{d}T}{\mathrm{d}\tau}\)

\subsubsection{两个物理过程}
\begin{enumerate}
    \item 由于温差的存在,将从外界吸收热量或者向外界释放热量【热交换】。
    \item 吸收或释放能量后自身的温度会上升或下降,直至与外界环境温度相同【比热容】。
\end{enumerate}

外界接触→热交换→热平衡→比热容,热量转换为温度。

\subsubsection{热交换方程}
\[
\mathrm{d}Q=-hs(T-\theta)\mathrm{d}t
\]

在\(\mathrm{d}t\)时间内交换的热量\(\mathrm{d}Q\),与温差有关,温差越大,转换能量越多。

\(h\)热交换系数,\(s\)感应球部表面积,\(T\)温度表温度,\(\theta\)介质温度,\(-\)负号表示热量从高温流向低温。

\subsubsection{比热容方程}
\[
\mathrm{d}Q=CM\,\mathrm{d}T
\]

自身温度变化值取决于材料比热容。

\(M\)是测温元件的质量,\(C\)是测温元件的比热容。

\subsubsection{联立}

\(\lambda=(hs/CM)^{-1}\)为热滞系数(取倒数,得到s单位)。
\begin{gather}
    \frac{\mathrm{d}T}{\mathrm{d}t}=-\left(\frac{1}{\lambda}\right)(T-\theta)\\
    \lambda=\frac{CM}{hs}\quad(\mathrm{s})
\end{gather}

\subsubsection{特性}

元件的热容量MC 越大,散热面积越小,则\(\lambda\)越大;热交换系数\(h\)的大小取决于环境介质性质和通风量。

\subsection{不同环境温度变化情况下,热滞系数如何影响滞差(\(T-\theta\))}。

\subsubsection{环境温度恒定}

即\(\theta\)为常数。经过一段时间\(t\)以后,热滞误差随时间\(t\)按负指数规律减小
\[
\frac{T-\theta}{T_0-\theta}=\mathrm{e}^{-\frac{\tau}{\lambda}}
\]

热滞系数确定:此时,当介质温度保持不变时,减小到起始温差的37\%时,所需要的时间即为热滞系数。

\subsubsection{环境温度呈线性变化时}
\[
T-\theta=-\beta\lambda\left(1-\mathrm{e}^{-\frac{\tau}{\lambda}}\right)
\]

当\(\frac{\tau}{\lambda}>5\),即感应时间远大于热滞系数\(\lambda\)时,上式简化为:
\[
T-\theta=-\beta\lambda
\]

说明在观测时间足够长以后,热滞误差将近似为常数,并且等于介质温度的变化率与测温元件热滞系数的乘积\(\beta\)为正时,\(T<\theta\),即仪器温度示数偏低;\(\beta\) 为负时,\(T>\theta\),即仪器温度示数偏高。

\subsubsection{环境温度呈周期性变化}
\[
T=\theta_0+\frac{A_0}{\sqrt{1+\frac{4\pi^2\lambda^2}{T_\mathrm{sr}^2}}}\sin\left(\frac{2\pi\tau}{T_\mathrm{sr}}-\tan^{-1}\frac{2\pi\lambda}{T_\mathrm{sr}}\right)
\]
\begin{enumerate}
    \item 温度表示度也呈周期性变化,其周期也是\(T_\mathrm{sr}\)
    \item 温度表示度的振辐小于介质温度的振辐,为介质振辐的\(\left(1+\frac{4\pi^2\lambda^2}{T_\mathrm{sr}^2}\right)^{-\frac{1}{2}}\)
    \item 温度表示度有位相落后,其落后相位角为: \(\varphi=\tan^{-1}\frac{2\pi\lambda}{T_\mathrm{st}}\)
\end{enumerate}

\subsection{元件测温的自动平均能力}

每个时刻测温元件的输出值,是此时刻之前一段时间内外界温度输入值的不等权平均值,其权重随着输入与输出时刻间隔的增大呈指数减小。

适当的热滞系数不仅可以自动有效地削弱短周期的温度脉动,而且使温度观测资料能代表更长时段内的平均值,从而代表更大空间范围内的平均值。

\subsection{WMO对地面观测中气温测量元件的要求}

当通风速度为5m/s时,热滞系数在30\~60s之间。

\section{温度测量的防辐射}

白天,空气对太阳辐射的吸收能力弱于任一种温度感应元件;夜晚,空气的红外辐射能力又弱于任一种温度感应元件的表面。

任何直接暴露在太阳辐射中的测温元件,测量值在白天将偏高于气温,夜间则偏低。为避免这种辐射误差,必须对测温元件采取有效的辐射屏蔽措施。

\subsection{防辐射设备}

\subsubsection{百叶箱}

四壁是由双层百叶玻璃钢片组成。

作用是防止太阳对仪器的直接辐射和地面对仪器的反射辐射,保护仪器免受强风、雨、雪等的影响,并使仪器感应部分有适当的通风。

\subsubsection{人工通风屏蔽罩}

\subsubsection{防辐射罩}

防辐射罩有轻便防辐射罩和小型防辐射罩两种。主要在野外考察时使用。

两块金属板之间嵌有两块透明的有机玻璃,传感器就安置在它们之间的夹层中,板的作用是为了隔绝金属板上的热对流。

\subsection{防辐射的途径和手段}

防辐射屏蔽;增加元件的反射率;人工通风,促使元件散热;采用极细金属丝元件,细丝具有较大散热系数。

\chapter{空气湿度的测量}

\section{湿度的参数及其观测}

\subsection{定义}

表示空气中的水汽含量和潮湿程度的物理量(离地面1.5米高度处)。

\subsection{混合比\(\gamma\)}

湿空气中水汽质量\(m_v\)与干空气质量\(m_a\)的比值\(\gamma=\frac{m_\mathrm{v}}{m_\mathrm{a}}\)。

\subsection{比湿\(q\)}

湿空气中水汽质量\(m_\mathrm{v}\)与湿空气总质量\(m_\mathrm{a}+m_\mathrm{v}\)的比值\(q=\frac{m_\mathrm{v}}{m_\mathrm{a}+m_\mathrm{v}}=\frac{\gamma}{\gamma+1}\)。	

\subsection{水汽压\(e\)}

湿空气中的水汽在单位面积上产生的压力。在大气压为\(P\),混合比为\(\gamma\)时,\(e=x_\mathrm{v}P=\frac{\gamma}{0.622+\gamma}P\),单位:hPa。

其中\(x_\mathrm{v}\)为水汽的相对摩尔分数,有\(x_\mathrm{v}=\frac{n_\mathrm{v}}{n_\mathrm{v}+n_\mathrm{a}}=\frac{\gamma}{\gamma+0.62198}\),\(n_\mathrm{v}\)为水汽摩尔质量,\(n_\mathrm{a}\)为干空气摩尔质量。

\subsection{饱和水汽压}


\subsubsection{水面饱和水汽压\(e_\mathrm{sw}\)}

气压气温不变时,水汽和水面达到气液两相中性平衡时,纯水蒸气产生的压力。

水汽压的极限值,超过极限值的部分凝结为液态水或凝华为冰。此极限值即为饱和水汽压
\[
e_\mathrm{sw}=6.112\mathrm{e}^{\frac{17.62t}{243.12+t}}
\]

随温度上升而上升

\subsubsection{冰面饱和水汽压\(e_\mathrm{si}\)}

气压气温不变时,水汽和冰面达到气固两相中性平衡时,纯水蒸气产生的压力
\[
e_\mathrm{si}=6.112\mathrm{e}^\frac{22.46t}{272.62+t}
\]

\subsubsection{关系}

同一温度下,水面上饱和水汽压大于冰面,水>冰。

\subsubsection{水汽动态包和过程}

离开水面(或冰面)的水分子与从空气中回到水中(或冰中)的数目相等。

\subsection{露点温度\(t_\mathrm{d}\)}

空气中水汽含量与气压不变时,冷却达到饱和的温度称露点温度。
空气中水汽饱和时,\(t=t_\mathrm{d}\),未饱和时,使用\(t-t_\mathrm{d}\)露点温度差表征湿度。

\subsection{霜点温度\(t_\mathrm{f}\)}

同上,对冰面达到饱和的温度。

\subsection{相对湿度\(U\)}

空气中实际水汽压\(e\)与饱和水汽压\(e_\mathrm{sw}\)比值
\[
U=\frac{e}{e_\mathrm{sw}}_{(p,T)}\times100\%
\]

\subsection{绝对湿度\(\rho_\mathrm{w}\)、水汽密度或水汽浓度}

体积为\(V\)的湿空气中,水汽质量为\(m_\mathrm{v}\)
\[
\rho_\mathrm{w}=\frac{m_\mathrm{v}}{V}\quad\text{单位}\mathrm{kg\cdot m^{-3}}
\]

\subsection{常用换算}
\[
\rho_\mathrm{w}=216.6\frac{e}{P}\quad q=0.622\frac{e}{P}\quad\gamma=0.622\frac{e}{P-e}
\]

\subsection{观测项目}

水汽压\(e\)、相对湿度\(U\)、露点温度\(t_\mathrm{d}\)

\subsection{测湿手段分类}
\begin{enumerate}
    \item 热力学方法:蒸发表面冷却降温的程度随湿度而变。干湿表。
    \item 吸湿法:吸湿物质吸湿后的尺度变化或电性能变化来测湿度。有毛发、肠膜元件、氯化锂元件、氧化铝元件等。
    \item 凝结法:测量凝结面降温产生凝结时的温度,即露点温度。主要仪器是露点仪。
    \item 光学方法:测量水汽对光辐射的吸收衰减作用。主要仪器有红外湿度计和赖曼\(\alpha\)湿度计。
    \item 称量法:直接称量出一定体积湿空气中的水汽含量。测量的要求较高,时间长,操作较繁,但其测湿准确度相当高,是湿度计量基准的一级标准,作为检定校准的基准。
\end{enumerate}

\section{干湿球温度表}

\subsubsection{原理}

由于蒸发,湿球表面不断有耗散蒸发潜热,使湿球温度下降;由于湿球与四周空气有温差,四周空气与湿球产生热交换。

当达到稳定的热平衡状态时,湿球温度表蒸发支出的热量应等于与从四周空气中热交换得到的热量。

\subsubsection{概述}

目前普遍使用的精度较高的一种测湿方法。干湿球温度表二者温度之差(t-tw)就指示空气的湿度。

干湿表测湿方程:\(=e_{\mathrm{w}}-Ap(t-t_\mathrm{w})\)。干湿表系数\(A=\frac{h_\mathrm{c}}{CL}\)可查表,\(A\)值一般为实验测定。

\subsubsection{注意}

当湿球纱布结冰时,会增加温度表滞后效应而造成干湿球温度表的读数误差。在-10℃时,即停止使用干湿表测湿。

\section{毛发湿度表}

\subsection{原理}

当相对湿度从0\%变到100\% 时,毛发总伸长量是原有长度的2.5\%

\subsection{关系}
\[
\lg U_\mathrm{h}=1.086\frac{\Delta L}{\Delta L_0}+0.918
\]

\(\Delta L_0\)相对湿度从0\%到100\%时毛发的总伸长量,\(\Delta L\)毛发随相对湿度的改变量。

\subsection{使用}

在相对湿度>30\%时使用

\subsection{低湿瘫痪}

毛发在相对湿度低于30%的空气中放置过久时,当湿度再回升时,毛发示度总是低于空气的实际湿度。消除的办法是将毛发放在饱和空气中,使其逐渐复原。


\section{冷镜式露点仪}

\subsection{原理}

若使空气通过一个光洁的金属镜面时等压降温,直到镜面上开始出现露,读取这瞬间的露点温度。通过露点可以推算出实际水汽压\(e\)、相对湿度等。

\subsection{光信号探测露点仪}

\subsubsection{工作原理}

当镜面温度高于露点温度时,镜面呈干燥状态,无凝结物形成,此时光源发出的入射光经镜面被完全反射至光电感应器。制冷器对镜面进行降温。

当镜面温度降至露点温度时,镜面出现凝结物,入射光在镜面上呈现漫反射,此时,光电信号减弱,通过电路反馈调节制冷装置,使制冷功率减小直至加热。

经过动态二阶响应后,镜面温度稳定在气体的露(霜)点温度。

\subsubsection{误差来源}
\begin{enumerate}
    \item 过冷水对霜点温度的影响:由于气体和镜面干净, 以致缺少足够数量的露核。可对镜面进行反复加热和冷却
    \item 开尔文效应:露滴的曲面饱和水汽压稍高于平面饱和水汽压,低于真实露点温度,误差约0.1 ℃
    \item 拉乌尔效应:空气与镜面之间有杂质,特别是有一定量的可溶性物质时,使饱和水汽压低于洁净空气和镜面的饱和水汽压,使露点温度值偏高。
\end{enumerate}

\section{电学湿度传感器}

\subsection{综述}

利用吸湿物质的电学参数进行湿度测量。

\subsection{电阻式湿度传感器}

\subsubsection{碳膜湿敏电阻}

基本原理:高分子聚合物基底(极性),形成电棒,能够吸水。在上表面涂碳粒,形成大量粉末状胶体。一旦有湿气吸附,则高分子聚合物膨胀,则碳粒之间的间隔增大。大量的碳相连导电能力强,但间隔增大后碳粒导电性能下降,则电阻增高。

特点:工艺要求复杂,探空仪标配

\subsubsection{氯化锂湿敏电阻}

基本原理:溶液浓度随着湿度变化。拉乌尔特定律。基底使用干的绝缘的聚四氟乙烯等,上面涂上氯化锂干粉,两端书状电极。当没有湿度时,干粉不导电,若湿度增加,氯化锂强吸湿,湿度越大,氢离子越多,导电率越高,电阻下降。

特点:成本低,便宜。有良好的线性关系,但曲线随温度变化而变,需要标定。

应用:广泛应用于探空(一次性)

\subsection{电容式湿度传感器}

基本原理:中部使用高分子聚合物为电介质,上下镀薄电极多孔金膜,湿度改变电介质介电系数与板间距\(d\),导致电容器电容发生变化。

电容量的变化正比于相对湿度。

特点:最为常用,精度最高。极板需要全面吸水,强透水性,电极需要极薄0.2微米。

\section{光学湿度传感器}

\subsection{综述}

比尔朗伯定律E指数关系 利用空气中的水汽对某特定波段的光通量产生的衰减量进行湿度测量。水汽在红外有强吸收2500-2900。发射L0的光,测得的L1光,得到消光系数=水汽吸收系数*水汽密度,则可得水汽密度。

\subsection{基本原理(差分吸收)}

选择两个不同的波长,一个对应于强水汽吸收带3.95,一个弱水汽吸收带2.59,通过比较两个吸收系数的差异得到消光系数与水汽含量。(可以消去噪声)

\subsection{应用}

水汽、二氧化碳等测量。

\chapter{气压}

\section{概述}

\subsection{大气压强}

单位面积上从所在地点往上直至大气上界整个空气柱的重量为压强。\(\int_h^{+\infty}\rho g\,\mathrm{d}z\)其随大气高度变化而变化。

\subsection{单位}

\begin{table}[htbp]
    \centering
    \begin{tabular}{ccc}
        \toprule
        单位名称 & 标注大气压 & 物理含义\\
        \midrule
        帕斯卡 & \(1.01\times10^5\) & \(1\mathrm{Pa}=1\mathrm{N/m^2}\)\\
        百帕 & \(1.01\times10^3\) & \(1\mathrm{hPa}=100\mathrm{Pa}\)\\
        毫巴 & \(1.01\times10^3\) & \(1\mathrm{mb}=1\mathrm{hPa}\)\\
        汞柱高 & \(760\mathrm{mmHg}\) & \(p=\rho gh\)\\
        \bottomrule
    \end{tabular}
\end{table}

\section{水银气压表}

\subsection{原理}

利用水银柱衡量大气压。直接读取水银柱高度Hbg(若用水,则有10m高)(托里切利真空)
\[
P_\mathrm{H}=\rho(t)\cdot g\cdot H_\mathrm{hg}[t,g]
\]

\subsection{优点}

密度大、蒸汽压小、水银性能稳定、水银不沾湿玻璃、水银面凸起弯月面,易判断水银柱顶位置。

\subsection{类型}

动槽式(福丁式)、定槽式

\subsubsection{动槽式水银气压表(福丁式)}

有测定水银柱高度的固定零点,每次测量需要调整水银面的高低,使其符合固定零点位置,然后才能读数

测定附表温度→调整象牙针相切→游标尺相切→记录气压值→降下水银面

\subsubsection{定槽式水银气压表(寇乌式)}

已经做好,不用修改。需要补偿液面下降导致的零刻度线变化,高度补偿。

高度补偿:\(xa=y(A-a')\)。其中\(a\)为内管横截面积,\(A\)为水银槽内横截面积,\(a'\)为玻璃管外横截面积。

水银柱上升\(x\mathrm{mm}\),槽部水银面下降\(y\mathrm{mm}\)。

\subsection{仪器误差}
\begin{enumerate}
    \item 基点标尺不准确
    \item 真空度降低
    \item 毛细管现象
    \item 温度的不确定
\end{enumerate}

\section{本站气压订正}

\subsection{目的}

仅针对水银气压表,大气压力与水银气压表所处环境的温度、重力加速度及纬度有关。

\subsection{原理}

测量大气压测量的是水银柱的高度\(H_\mathrm{bg}[t,g]=\frac{P_\mathrm{H}}{\rho(t)\cdot g}\),故需要对气压进行订正,核心问题为\(a\)、\(b\)的确定。
\[
P=a\rho_0bg_0
\]

大气压与温度、重力加速度、维度有关。为便于比较,国际统一规定:
\begin{enumerate}
    \item \(rho\)以温度0℃为标准
    \item \(g\)以维度45°海平面为标准,\(9.80665m/s^2\)
\end{enumerate}

若不在标准下,读的水银柱要订正。

\subsection{订正三大步骤}

\subsubsection{仪器误差订正}

由于制成气压表材料的物理特性变化和技术条件限制导致的误差。可在检定证上查询。

\subsubsection{温度订正}

来源:水银的密度会随温度的变化,液态时热胀冷缩,要把水银的密度订正到\(0^\circ\mathrm{C}\)时的标准密度

铜尺的长度随温度变化的伸缩,由于水银柱膨胀较铜尺膨胀得大,气压会比实际偏高

理论:
\[
P=\rho(0^\circ\mathrm{C})\cdot g(45^\circ,0)\cdot H_0=\rho(t)\cdot g(\varphi,h)\cdot H\rightarrow H_0=\frac{g(\varphi,h)}{g(45^\circ,0)}(H+\Delta H)
\]

动槽\(\Delta H=-H\frac{0.0001634T}{1+0.0001818t}\),定槽\(\Delta H=-H\frac{0.0001634t}{1+0.0001818t}1.33\frac{V}{A}(\mu-3\eta)t\)。

\(A\)为水银槽的截面积;\(V\)为气压表内的水银体积;\(\eta\)为铜的热膨胀系数;\(\mu\)为水银的膨胀系数

\subsubsection{重力订正}

来源:重力加速度随纬度\(\varphi\)和海拔高度\(h\)的变化而变化。

理论:
\[
H_0=\frac{\rho(t)\cdot g(\varphi,h)}{\rho(0^\circ\mathrm{C})\cdot g(45^\circ,0)}H\quad g(45^\circ,0)=9.80665
\]

纬度订正:在海平面(\(h=0\)),重力加速度随纬度的变化\(g_{\varphi,0}=980.616(1-0.0026373\cos2\varphi+0.0000059\cos^2\varphi)\)

高度订正:\(g_{\varphi,h}=g_{\varphi,0}-0.0000003086h+0.0000001118(h-h')\)

\(h\)为测站的海拔高度,\(h'\)为以测站为中心的半径150km范围内的平均海拔高度。

\subsection{本站气压}
\[
\begin{cases}
P_h=\rho(0^\circ\mathrm{C})\cdot g(45^\circ,0)\cdot H\\
H_0=\frac{g(\varphi,h)}{g(45^\circ,0)}(H+\Delta H)+C
\end{cases}
\]

\(P_h\):本站气压,\(C\):仪器差订正值,\(g_{\varphi,h}\):测站重力加速度,\(g(45^\circ,0)\):标准状态下的标准重力加速度,\(\Delta H\)经温度订正后的水银柱高度变化量。

\section{常见的气压传感器}

\subsection{空盒气压表}

\subsubsection{感应元件}

空盒(内真空),三面硬质,顶面金属皮。

\subsubsection{原理}

盒内空气几乎抽空,根据空盒弯曲的程度反应气压。 一般做成螺旋状。将空盒形变量转换为指针摆动量。

\subsubsection{空盒气压计}

利用空盒感应元件制成的连续记录气压的仪器,包括感应部分、传动放大部分、自记部分。

\subsubsection{特点}

精确度低,但便于携带,适用于野外。

\subsection{振筒式压力传感器}

\subsubsection{感应元件}

高导磁率、高弹性的金属制成的薄壁圆筒。

\subsubsection{原理}

弹性金属圆筒在外力作用下发生振动,当筒壁两侧存在压力差时,其振动频率随压力差而变化。根据振动频率与大气压力间的关系,可以得到大气压力。\(\Delta L=L_0-L'\)

\subsubsection{特点}

感应元件测压精度高,输出电参量,性能稳定,准确度高。

\subsection{膜盒式电容气压传感器}

\subsubsection{感应元件}

真空膜盒。

\subsubsection{原理}

膜盒形变改变电极板间距,测量电容变化量。

\subsection{压阻式气压传感器}

\subsubsection{感应元件}

真空膜盒。

\subsubsection{原理}

气压改变,使真空膜盒弹性膜片产生形变,引起膜片材料的电导率改变(单晶硅)。

电流引起测量元件自身温度变化导致测量不准确。

\subsubsection{特点}

耐用,目前业务上最常用。

\subsection{沸点式气压传感器}

\subsubsection{反应原件}

测量液体。

\subsubsection{原理}

测量液体的沸点温度而获得相应的大气压力。液体沸点温度随外界压力\(P\)变化而变化:
\[
\ln P=C-\frac{L}{RT}
\]

其中\(C\)为常数,\(L\)为蒸发潜热,\(R\)为液体蒸汽的气体常数,\(P\)为蒸汽气压。

\subsubsection{优点}

将气压测量转化为温度测量、精度比空盒高。

\subsubsection{应用}

探空常用,成本低,体积小。

\section{气压的海平面订正}

\subsection{解决问题}

不同测量高度的测量站的数据进行比较。为了进行各地气压高低的比较,分析气压场。

\subsection{拉普拉斯压高公式}
\begin{gather}
    \lg\frac{p_0}{p_h}=\frac{h}{18410(1+\frac{t_\mathrm{m}}{273})}\\
    t_\mathrm{m}=\frac{t_h+t_0}{2}\quad t_h=\frac{t+t_{12}}{2}\quad t_0=t_h+\gamma h
\end{gather}

其中:\(p_0\)为海平面气压,\(t_\mathrm{m}\)为海平面与测站之间空气柱的平均温度,\(p_h\)为本站气压,\(t_h\)为本站气温,可用当时气温与12小时前气温\(t_{12}\)的平均值确定,\(t_h\)避免大气潮汐影响。(迁徙潮(周日潮),由太阳加热导致),\(t_0\)可利用温度线性递增来决定。取\(0.5^\circ\mathrm{C}/100\mathrm{m}\)

解方程得:
\[
p_0=p_h\cdot10^m
\]

\chapter{地面风的观测}

\section{概述}

\subsection{分类}
\begin{enumerate}
    \item 地面风
    \item 高空风
\end{enumerate}

垂直分量称为空气的垂直运动

\subsection{定义}

大气相对地面运动的水平分量叫做风。

风的观测包括风向风速两项。

\subsection{风速}

单位时间气流运动水平距离

\subsubsection{平均值}

一定时段内的平均
天气报告中的风速指2min内的平均风速(人工每5s记录一次)

\subsubsection{瞬时值}

相当短的取样时段内平均

\subsubsection{静风}

风速<0.25m/s

\subsection{风向}

风的来向,共计16个方位

\subsection{风玫瑰图}

在极坐标底图上,各风向出现的频率的统计图,为风向玫瑰图。

在极坐标地图上,各风向的平均风速的统计图,为风速玫瑰图。

风玫瑰图的应用:主要应用于机场建设、起落等。(飞机一般顺风起飞)


\subsection{风级划分}
\begin{table}[htbp]
    \centering
    \begin{tabular}{cccc}
        \toprule
        风级 & 名称 & 风速m/s & 事件\\
        \midrule
        0 & 静风 & 0-0.25 & \\
        8 & 大风 & 17.2-20.7 & 大风预警\\
        12 & 飓风 & 32.7 & 台风\\
        \bottomrule
    \end{tabular}
\end{table}

\subsection{气象预警}
\begin{table}[htbp]
    \centering
    \begin{tabular}{cccc}
        \toprule
        警告 & 等级\\
        \midrule
        蓝色 & 6级以上\\
        黄色 & 8级以上\\
        橙色 & 10级以上\\
        红色 & 12级以上\\
        \bottomrule
    \end{tabular}
\end{table}

\subsection{代表性资料}

取一段时间内的平均风速和最多风向。

气象台站中,一般取2min平均风速与最多风向。

\subsection{平均方法}

\subsubsection{风速}

算术平均。

矢量平均法【最常用】:每一个风矢量分为x、y方向,叠加全部取平均。

\subsubsection{风向}

算术平均、标量平均、矢量平均、单位矢量。

在处理风向平均时特别注意风向过北(风向变化跨零)导致结果失真。(例如10°、350°)

\subsubsection{阵风}

强湍流引起的风的急剧变化称为风的阵性。

在规定的时间间隔内,风速对其平均值的持续时间不大于2min的正负偏离。
\begin{itemize}
    \item 标准偏差:表征风的脉动大小。
    \item 阵风峰值:最大风速。
    \item 阵风持续时间。
\end{itemize}

\section{风向}


\subsection{风向标测风}

\subsubsection{形容}

首尾不对称平衡装置,一般由尾翼、指向杆、平衡锤及旋转主轴四部分组成。指向杆所指示的方向,即为风的来向。

\subsubsection{原理}

利用风压原理,直到尾翼两侧风压相同时,受力平衡,不再产生力矩。

\subsubsection{起动风速}

0.25m/s当尾翼所受风的压力矩>风向标静摩擦力矩。
\begin{itemize}
    \item 灵敏性:很快地反映风向变化。
    \item 稳定性:分享改变时,本身惯性引起的摆动要小。
\end{itemize}

测量误差启动误差、动态偏角、惯性误差、转换误差、零位误差。

在2min平均后一般能被忽略不计。

\subsection{风向信号转换与记录}

\subsection{光电码盘(格雷码)}

每一个角度状态的变化只有一位二进制数发生变化,持续沿用到现在。

\section{旋转式风速仪}

\subsection{风杯风速计}

\subsubsection{原理}

风压原理
\[
V=K\omega=2\pi KN
\]

与风速呈现线性关系,式中\(V\)为风速,\(K\)为风杯系数,\(\omega\)为角度,\(N\)为转数。\(K\)一般在2.2-3之间(普遍使用)。

\subsection{旋桨式风速计}

将风速与风向相结合,能逆风绕水平转动,转速正比于风速。
\[
V=\frac{N}{B\tan\theta}
\]

\(B\)为风车系数,\(N\)为单位时间翼片转数,\(\theta\)为螺旋桨翼片迎风角。

\subsubsection{缺点}

成本较高。

\subsubsection{误差分析}

启动误差,滞后误差,惯性误差。

\section{热线式风速仪和超声风速仪}

\subsection{热线式风速仪}

\subsubsection{结构}

非常细的金属丝,通过金属丝的散热率来测风

\subsubsection{原理}

通电让金属丝加热,风吹让金属丝散热。有两种方法,恒温变流,恒流变温(测量阻值)。

\subsubsection{方向}

与风向垂直【局限性】,对工作环境要求高,灰尘、水附着干扰,金属丝易损坏。

\subsubsection{优点}

体积相当小,对测量环境没有破坏性,测量容易,敏感度高,响应快,惯性小。

\subsubsection{缺点}

金属丝易断,工作环境要求高(灰尘)。

\subsubsection{误差}
\begin{enumerate}
    \item 环境气温变化造成的污染。
    \item 热线方向与气流方向不垂直。
    \item 空气密度改变导致的误差。
\end{enumerate}

\subsection{超声风速仪}

\subsubsection{结构}

三个爪形结构,收发一体,三个方向,采样空间的大气密度与流速的改变影响速度。

\subsubsection{原理}

声速随风速产生变化,声速会与风向上的气流速度叠加,顺风变快。

同时测得正向、反向信号传输时间,即可联立方程组解得风速。
\[
u=\frac{(t_\mathrm{b}-t_\mathrm{o})c^2}{2d}
\]

式中\(c\)为声速\(c=\frac{2d}{t_\mathrm{b}+t_\mathrm{o}}\)。同理可得到其他两个方向的风速。

同时,可以给出温度与湿度的量(声速与温湿度有关)。

\subsubsection{误差(阴影效应)}

由于绕流作用,迎风面探头背后产生一定的尾流,导致声波传播路径偏长,使风速偏低。

\subsubsection{特点}

快速响应,极端精确,微机控制,无活动部件,易安装,低功耗。

\section{地面风的特征与应用}

\subsection{仪器安装}

高度:10-12m

地点:开阔空旷,远离障碍物等,具有代表性
\begin{enumerate}
    \item 天气预报,风的测量是相当大的难点
    \item 城市规划
    \item 森林防火
    \item 军事气象
    \item 能源发电
\end{enumerate}

\chapter{辐射及日照时数的观测}

\section{基础知识}

\subsection{辐射}

电磁波传递能量的一种方式。太阳辐射是地表获能主要来源

\subsection{辐射分类}
\begin{itemize}
    \item 短波:0.29微米-3微米,占据太阳97\%
    \item 长波:3微米-100微米
    \item 太阳辐射:集中于可见光辐射0.4-0.76\(\mu\)m
    \item 地面辐射: 长波辐射99\%的>5\(\mu\)m
\end{itemize}

\subsection{黑体辐射定律}

维恩位移定律:\(\lambda_\mathrm{max}=\frac{2897.8}{T}\)
\begin{table}[htbp]
    \centering
    \begin{tabular}{cc}
        \toprule
        辐射源 & 温度与波长\\
        \midrule
        太阳 & \(6000\mathrm{K}\rightarrow0.483\mu\mathrm{m}\)蓝光\\
        地面 & \(300\mathrm{K}\rightarrow9.659\mu\mathrm{m}\)\\
        大气 & \(200\mathrm{K}\rightarrow14.489\mu\mathrm{m}\)\\
        \bottomrule
    \end{tabular}
\end{table}

\subsection{辐照度}

\(\mathrm{W\cdot m^{-2}}\),是瞬时值,单位时间投射到单位面积上的辐射能。

\subsection{曝辐量}

\(\mathrm{MJ\cdot m^{-2}}\),兆,累积量,单位时间辐照度的总量。

\subsection{辐射组成}

\subsubsection{\(S_\mathrm{L}\)太阳直接辐射}

包括来自太阳面的直接辐射和太阳周围一个狭窄的环形天空辐射。通过直接辐射表测量。
\[
S_\mathrm{L}=S\cdot\sin H_\mathrm{A}=S\cdot\cos Z
\]

\(H_\mathrm{A}\)为太阳高度角,\(Z\)为天顶角,\(S\)为总辐射。

太阳常数\(S_0\):在日地平均距离,地球大气上界垂直于太阳光束方向上接收到的太阳辐照度。\(S_0=1367\pm7\mathrm{W/m^2}\)

\subsubsection{\(E_\mathrm{d}\downarrow\)散射辐射}

太阳辐射经过大气散射或者云的反射,从天空\(2\pi\)立体角以短波形式向下。

使用总辐射表遮住太阳辐射来测量。

\subsubsection{\(E_\mathrm{g}\downarrow\)总辐射}
\[
E_\mathrm{g}\downarrow=S_\mathrm{L}+E_\mathrm{d}\downarrow\text{直接辐射+散射辐射(约为800W)}
\]

太阳被云遮蔽,\(E_\mathrm{g}\downarrow=E_\mathrm{d}\downarrow\),夜间\(E_\mathrm{g}\downarrow=0\)。

\subsubsection{\(E_\mathrm{r}\uparrow\)反射辐射}

总辐射到达地面后被下垫面向上反射的那部分短波辐射。将总辐射表感应面向下安装,则可测量。

地表反照率\(E_\mathrm{k}=\frac{E_\mathrm{r}\uparrow}{E_\mathrm{g}\downarrow}\)反射比,表示下垫面的反射能力。

\subsubsection{\(E_\mathrm{L}\downarrow\)大气长波辐射}

大气以长波形式向下发射的辐射。辐射计接收球面使用单晶硅黑色材质透光玻璃,可吸收短波,便于长波透过。

\subsubsection{\(E_\mathrm{L}\uparrow\)地面长波辐射}

地球表面以长波形式向上发射的辐射,与地面温度有密切联系。包括地面长波反射辐射。

\subsection{全辐射}
\[
E=E_\mathrm{g}\downarrow+E_\mathrm{L}\downarrow+E_\mathrm{r}\uparrow+E_\mathrm{L}\uparrow
\]

短波辐射与长波辐射之和,波长覆盖0.29~\(100\mu\mathrm{m}\)

\subsection{净全辐射}
\[
E^*=E_\mathrm{g}\downarrow+E_\mathrm{L}\downarrow-E_\mathrm{r}\uparrow-E_\mathrm{L}\uparrow
\]

净全辐射代表测站处辐射收支的平衡状态。

\subsection{净短波辐射}
\[
E_\mathrm{g}^*=E_\mathrm{g}\downarrow-E_\mathrm{r}\uparrow
\]

\subsection{净长波辐射}
\[
E_\mathrm{L}^*=E_\mathrm{L}\downarrow-E_\mathrm{L}\uparrow
\]

\subsection{地球辐射收支}

太阳入射,大气(云)辐射/吸收/反射,地表辐射/吸收/反射

\section{辐射的观测仪器及原理}

\subsection{辐射传感器}

由感应面和热电堆组成(热电堆为多个热电偶成排)
\begin{itemize}
    \item 感应面:黑色的,吸热性能好的薄金属片
    \item 热电堆:工作段位于感应面下部,参考端位于没有辐射的隐蔽处产生温差变化,产生温差电动势。
\end{itemize}

\subsection{太阳直接辐射的测量(绝对直接辐射表)}
\begin{itemize}
    \item PACRAD型,绝对日射表,一级标准仪器。
    \item 埃斯川姆(\r{A}ngstr\"{o}m)补偿式绝对日射表,二级标准仪器。
\end{itemize}

\subsection{太阳直接辐射表}

用于测量包括来自太阳面的直射辐射和太阳周围一个非常狭窄的环形天空辐射,含有大量遮光环以消除杂散光。
\begin{itemize}
    \item 进光筒:金属圆筒,筒口有石英玻璃。
    \item 瞄准器:为对准太阳,进光孔两端分别固定两个金属圆环
    \item 感应件:感应面和热电堆组成,安装在光筒后端,面垂直于太阳
    \item 跟踪架:使进光筒自动准确的跟踪太阳
\end{itemize}

\subsection{总辐射表}

用途较广,可测量总辐射、散射辐射(加入遮光环)、短波反射辐射(朝下放置)
\begin{itemize}
    \item 方位响应:保证旋转一周,读数应当不变。
    \item 余弦响应:天顶角改变时,仪器读数应当与天顶角满足余弦关系。
\end{itemize}

\subsection{长波辐射表}

构造、外观与总辐射表基本相合,使用硅单晶玻璃罩阻隔\(3\mu\mathrm{m}\)以下的短波辐射。

\subsection{净全辐射的测量}
\[
E^*=E_\mathrm{g}\downarrow+E_\mathrm{L}\downarrow-E_\mathrm{r}\uparrow-E_\mathrm{L}\uparrow
\]

使用聚乙烯塑料罩,热电堆与上下的两个感应面相连。

\section{日照时数的观测}

\subsection{定义}

太阳直接辐射辐照度达到\(120\mathrm{W\cdot m^{-2}}\)的各时段时间的总和。单位:小时,取一位小数。

\subsection{可照时数}

太阳从地平线到地平线的光线照射到地面所经历的时间。

\subsection{日照百分比}
\[
\frac{\text{日照时数}}{\text{可照时数}}\times100\%
\]

\subsection{暗筒式日照计(烧痕法)}

利用阳光透过仪器上的小孔射入筒内,使涂有感光药剂的日照纸上留下感光痕迹线。

\subsection{聚焦式日照计(康培司托克)}

实心玻璃起聚光作用,太阳经玻璃球聚焦后烧灼日照纸(卡片)留下的焦痕。用于高纬地区。

\subsection{直接辐射表观测日照时数}

直射表每日自动跟踪太阳输出的信号,自动测量系统把\(S\ge120\mathrm{W\cdot m^{-2}}\)的时间累加起来,作为每小时的日照时数与每天日照时数。

\section{光电式日照传感器}

综合性传感器,不使用热电堆,使用光电二极管。

\subsection{原理}

前端有凹面镜,可接收全天空辐射

3个光电探测器位于筒身上、中、下三个位置,分别是一级、二级、三级光电探测器。

第一级光电探测器始终暴露在太阳辐射中,可以测量太阳总辐射;第二、三级光电探测器被设计为在任何时候都有一个被遮光筒遮蔽,不能接收到太阳直接辐射,该光电探测器测量出的信号经过计算处理可以得到太阳的散射辐射

通过太阳总辐射和直接辐射进行计算,可以得到当前的太阳直接辐射辐照度,再与120W阈值进行比较,输出日照信号或者通过计算,得到分钟日照、小时累计日照以及日累计时数。

\chapter{积雪、冻土与电线积冰}

\section{三个参量}

\subsection{积雪}

雪被,雪盖。覆盖在陆地和海冰表面的雪层。

\subsubsection{气象观测规范}

测站四周能见面积被雪覆盖超过一半,并记该日为积雪日。

\subsubsection{分类}

永久积雪(终年不化,雪平线以上),稳定积雪(60日不化),不稳定积雪(空间上积雪不连续,呈现斑状分布,10-60天),瞬间积雪(寒潮、强冷空气等导致,快速消融)。

\subsubsection{特点}

冷圈中分布最广泛,年际变化和季节变化最显著的一员。每年全球被积雪覆盖的面积占表面积23\%,2/3在陆地。
\begin{itemize}
    \item 好处:牧草越冬保温作用,增加土壤水分,供给水源。
    \item 坏处:雪灾,畜牧呀困难,破坏交通电力运输等。
\end{itemize}

\subsection{冻土}

\subsubsection{分类}

短时冻土(暂时冻结,受天气变化影响),季节性冻土(冬季冻结,夏季融化),永久冻土(常年冻结)。

季节性冻土:地表层包括季节冻结层与季节融化层。季节融化层的冻结一是自地表往下,一是多年冻土上限往上。

\subsubsection{危害}

冻胀,融沉。青藏高原铁路:热棒,内有液氮,让铁路下方的土地始终寒冷。

\subsection{电线积冰}

雨凇、雾凇凝结在导线上或湿雪冻结在导线上的现象。附着在导线上的霜或小的结冰不算积冰。

\subsubsection{分类}

降水积冰,云雾积冰。

\subsubsection{危害}
\begin{enumerate}
    \item 重量承压
    \item 改变金属性质,输电效率下降
\end{enumerate}

\section{积雪观察}

主要有雪深、雪压。
\begin{itemize}
    \item 雪深:表面到地面的垂直深度,cm,整数。
    \item 雪压:单位面积上的积雪重量,\(\mathrm{g/cm^2}\),一位小数。
    \item 观测地点:专门的雪深场。
    \item 人工观测:量雪尺。
\end{itemize}

\subsection{超声波雪深传感器}

声阻抗不同的界面上产生反射,测量距离。
\[
H_\mathrm{S}=H-0.5c\cdot t
\]
\begin{itemize}
    \item 优点:穿透能力强,有发散性,可同时得到上下表面的反射。
    \item 缺点:收到空气湿度的影响极大,影响声速。
\end{itemize}

\subsection{激光雪深传感器}

发射调制的可见激光,通过比较接收回波时间差,测量其与物体的距离。激光无法穿透积雪,可在上表面产生一次散射,得以测量。

优点:功耗低,省电,精度高。

\subsection{雪压人工观测}

目前基本不测量雪压,使用体积量雪器。

\section{冻土}

测定冻结层次及其上下限深度,单位cm,取整数。每日8时观测一次,直至土壤完全解冻。

\subsection{基本仪器(冻土器)}

由外管和内管组成,外管为一标有0cm的硬橡胶管,内管为一根有cm刻度的橡皮管(内有固定附着冰用的铁链线绳)。内管中灌注当地干净的水至刻度的0线处,安装在有自然覆盖物的地段。

\section{电线积冰}

\subsection{积冰过程}

从积冰架上导线开始形成积冰起直至积冰消失。观测一次积冰过程的最大直径与厚度(mm,整数)。当单纯雾凇直径达到15mm或雨凇湿雪混合积冰达到8mm时,测定一次积冰最大重量(g/m)

\subsection{观测仪器}

电线积冰架

\subsection{积冰直径与厚度}
\begin{itemize}
    \item 直径:积冰层切面上的最大直径线段。
    \item 厚度:垂直于直径方向的积冰层最大直线段,使用外卡钳。
\end{itemize}

\subsection{积冰重量}

指1cm长导线上的积冰重量。

\subsection{观测方法}

利用合页箱固定冰层,融化后称重。

\chapter{降水和蒸发}

降水为各系统要素耦合的关键,对人民生产生活具有最直接影响。

\section{基本知识}

\subsection{降水}

降水指降到地面上的水汽凝成物,包括雨、雪、冰、雹、雾、露。

\subsection{降水量}

指落到地面上的液态和固态降水,没有经过蒸发、渗透和流失而在水平面上积聚的深度,单位mm,一般没有小数点。

\subsection{降水强度}

单位时间间隔内的降水量。
\begin{table}[htbp]
    \centering
    \begin{tabular}{cc}
        \toprule
        雨量强度 & 24h深度\\
        \midrule
        小雨 <10mm\\
        中雨10-24.9\\
        大雨25-49\\
        暴雨50-99.9\\
        大暴雨100-199.9\\
        特大暴雨>200\\
        \bottomrule
    \end{tabular}
\end{table}

\subsection{蒸发}

温度低于水的沸点时,水汽从水面、冰面表面逃逸过程。

\subsection{蒸发量}

一定口径的蒸发器中,在一定时间间隔内因为蒸发而失去的水层深度,以毫米为单位,取一位小数(蒸发能力的测量)。

\section{降水观测}

\subsection{人工观测}

雨量筒

\subsection{自动观测}

虹吸式、翻斗式、双阀容栅式

\subsection{雨量筒}

上方为漏斗,内为烧瓶(储水瓶)。注意刻度1:25的量筒刻度转化比率。

\subsection{虹吸式雨量计}

连通器原理,直到液面高于虹吸管,水便流净。测量斜率可得降雨强度,积分可得降雨量。

\subsection{翻斗式雨量计}

有多个翻斗,起缓冲作用,避免强降雨冲力导致降水偏高。目前一般使用双层翻斗。
\begin{itemize}
    \item 原理:翻动一次最小分辨率0.1mm,雨强范围不大于4mm/min。
    \item 缺点:不适用于弱降水。
\end{itemize}

\subsection{称重式雨量计}

利用电子秤称量收集的降水重量。适用于测量强降雨。

\subsection{双阀容栅式雨量计}

利用浮子,全部使用电学记录。浮子上方为电极板,与固定的电极板构成电容,测量电容即可测得降水量。

双阀为进出水阀门,可以在放水时关闭进水阀门,并瞬时释放积水,使得精确性大大提高。

\section{蒸发观测}

\subsection{小型蒸发器}
\[
\text{蒸发量}=\text{原量}+\text{降水量}-\text{余量}
\]

铁丝网:防止鸟兽饮水。

\subsection{大型蒸发器}

外圈水池:隔离地面对蒸发的影响,有隔热作用。

使用超声蒸发传感器测量。\(\text{蒸发量}=\text{前一日水面高度}+\text{降水量}-\text{测量时水面高度}\)

\subsection{超声蒸发传感器}

根据超声波测距原理,\(H=C_\mathrm{w}=\cdot t/2\),\(C_\mathrm{w}\)水中声速,\(t\)超声波脉冲往返于水面高度\(H\)经历的时间。

\chapter{高空探测}

\section{探空气球与无线电探空仪}

\subsection{常用手段}

雷达、无线电探空仪、卫星探测、火箭探测、飞机探测。

\subsection{高空气象观测}

测量近地面层以上大气的物理、化学特性的方法和技术。

测定大气各高度上的温度、湿度、气压、风向、风速为主。

\subsection{探空气球}

悬挂无线电探空仪,由充有氢气的探空气球携带无线电探空仪上升,进行温压湿的测量。

飞到30-40km,上升速度6-8m/s,大约1.5h。

\subsubsection{分类}
\begin{itemize}
    \item 膨胀型:最常用,不断上升,一般用于垂直探测;由伸缩性大的橡胶制成,内有氢气。
    \item 非膨胀型:用于水平探测,制作定高气球,系留气球;球皮由聚乙烯塑料薄膜,在超压状态下工作,无伸缩性。
    \item 上升气球、平移气球、下降气球(火箭或飞机释放,后释放降落伞,下降探测)。
\end{itemize}

\subsection{无线电探空仪}

一种遥测仪器,将感应的气象要素值转换为无线电信号,不断地向地面发送。

悬挂于探空气球下方8m左右。约300-500g,覆盖有防水纸与泡沫构成。

\section{高空温压湿观测}

GTS1无线电探空仪,内有温度传感器、湿度传感器、智能转换器(含气压)、发射机(应答机(二次雷达))、电池五部分。
\begin{itemize}
    \item 温度:棒状热敏电阻 -80~40\(^\circ\mathrm{C}\)。
    \item 湿度:高分子湿敏电阻 出厂时都密封在带有干燥剂的玻璃瓶中,一旦打开,要及时使用。
    \item 气压:硅阻固态压力传感器 10-1050hPa 气压传感器安装在电路板上 (在智能转换器上)。
    \item 智能转换器:将物理量转换为二进制代码。
\end{itemize}

\section{高空风观测}

\subsection{观测方法}
\begin{enumerate}
    \item 利用飞机、气球将测风仪器带到高空。
    \item 通过跟踪测风气球的运动轨迹来计算风向风速,会产生水平位移。
\end{enumerate}

\subsection{受力分析}

气球上升过程的受力,气球升速的变化情况。

气球浮力:\(F_\text{浮}=\rho gV\),氢气重力:\(G=\gamma gV\),球皮和附加物:\(B\)(\(\rho\)空气密度,\(gamma\)氢气密度,\(V\)气球的体积)。

\subsubsection{总举力}

与气球同体积的空气重力与氢气重力之差:
\[
E=(\rho-\gamma)gV
\]

总举力不能够完全决定气球能否上升,只有净举力才能完全决定。

\subsubsection{静举力}
\[
A=E-B=(\rho-\gamma)gV-B
\]

其一般不随气球上升而改变。因此,认为升速为一恒定值。

假设:
\begin{enumerate}
    \item 球内部的气体压力\(\approx\)周围的空气压力。
    \item 气球内外温度是相等的。
    \item 氢气的扩散逸出可忽略,球内气体质量不变
\end{enumerate}

由假设可得:球内氢气密度正比于外界空气密度的变化,反比于气球的体积的变化。
\begin{gather}
A=(\rho–\gamma)gV–B=(\rho V–\gamma V)g–B\\
A_0=(\rho_0–\gamma_0)gV_0–B=(\rho_0V–\gamma_0V)g–B\\
V\rho=V_0\rho_0=\text{常量}\quad V\gamma=V_0\gamma_0=\text{常量}\quad A_0=A
\end{gather}

\subsubsection{空气阻力}
\[
R=c\frac{\rho}{g}s\omega^2
\]

\(s\)为气球面积,\(\omega\)为气球速度,\(c\)为阻力系数。

\subsection{常用气球升速公式}
\[
\omega=b\rho^{-\frac16}\frac{\sqrt{A}}{\sqrt[3]{A+B}}
\]

\(b\)为一略微变化的固定值。

\subsubsection{影响因子}
\begin{enumerate}
    \item A与B:若要增加升速,则增加A,减小B。
    \item 空气密度\(\rho\)的变化,气球的升速是随空气密度的减小而缓慢增大的。
    \item 空气阻力系数\(c\),在乱流很强时,空气阻力系数减小,使计算值偏低。
    \item 垂直气流对升速的影响很大(尤其在复杂地形区)。
    \item 渗透和扩散的影响,导致气球升速变小。
\end{enumerate}

\subsection{误差}

实际升速与计算升速差别为两头大中间小。近地面实际升速更大(空气的乱流运动的影响),高空实际升速更小(氢气渗透量增加)。

\section{高空风速风向计算}

看似高深的内容,实质上考察简单的立体几何。

\subsection{主要方法}

单经纬仪定点测风、双经纬仪定点测风、矢量法计算气球高度、二次雷达测风

\subsection{单经纬仪定点测风}

\subsubsection{原理}

使用一台经纬仪,在一个固定的地点,观测跟踪气球移动的轨迹来确定高空的风向风速

经纬仪能够测的仰角与方位角,由垂直高度得到升速。

升速确定,某一瞬时气球在空间的位置对地面有一个垂直投影点,测得某一个时段内气球投影点的位移,就是气球在这段时间内的水平位移,由此测得对应高度上的风向风速。

\subsubsection{例题}

测风气球的升速\(w=100\)米/分;第2分钟的仰角\(\alpha_2=45^\circ\),方位角\(\beta_2=30^\circ\);第4分钟的仰角\(\alpha_2=30^\circ\),方位角\(\beta_2=45^\circ\);求:第2-4分钟平均风速和风向?
	   
\subsubsection{优点}

简单易操作。

\subsubsection{缺电}

假定气球升速为定值,计算气球高度存在误差,较为古老。

\subsection{雷达测风}

\subsubsection{原理}

借助仰角、方位角和斜距来定位气球,测量准确。雷达波段\(L\)

二次雷达测风:所跟踪的目标是有源的,气球上有应答机,有效提高测量距离(200km水平30km垂直)。

\subsubsection{优点}

全自动观测。

\subsubsection{缺点}

要求雷达发射机发射频率大,耗电量大,探测距离低,距离远时回波弱(平均高度10km,斜距最大约60km)。

\chapter{雷达气象基础}

\section{雷达基础知识}

\subsection{雷达}

一种利用电磁波探测目标特性和距离的装置

\subsection{雷达气象}

降水是气象要素中最重要的一个,重视降水的探测与预报。利用气象雷达探测大气降水等现象、研究天气系统演变过程、并服务于降水的临近预报等。

\subsection{气象雷达}

用于气象探测研究的雷达,主要用于探测降水。

\subsection{分类}
\begin{enumerate}
    \item 功能:常规雷达、多普勒雷达、双偏振多普勒雷达。
    \item 波长:厘米波雷达、毫米波雷达。
    \item 波段:S、C、X、Ku、Ka、W,其中S、C、X为天气雷达。
\end{enumerate}

\subsection{主要作用}

测量降水、测量多普勒速度、探测中尺度天气过程、探测台风、检测天气移动、提供高密度数据、短临预报

\subsection{雷达概况}

218部业务大型雷达。

\section{传统天气雷达}

\subsection{原理}

发射电磁波,通过回波判断降水强度及其变化。通过接收回波的时间判断雨雪云层距离。发射的为脉冲波(两个脉冲0.25-4微秒的间隔直接决定了探测范围)(一般在500-600km)。

雷达使用瑞利散射。期望为小粒子的散射,瑞利散射与形状关系不大。

\subsection{雷达气象方程}
\[
P_\mathrm{r}=\frac{\pi^3P_tG^2\tau C\theta\varphi}{1024\ln2\lambda^2}\times\frac{1}{R^2}\times|K|^2\times Z
\]

\(P_\mathrm{r}\)为回波信号功率,\(R\)为距离雷达距离,\(Z\)为反射因子(和雨滴大小有直接关系,与单位体积内的粒子大小和数量有关,通常简称为回波强度),\(K\)为常数,其余为雷达硬件参数。

\subsection{回波强度}
\[
Z=\int_{D_\mathrm{min}}^{D_\mathrm{max}}n(D)D^6\,\mathrm{d}D
\]

取对数\(\mathrm{dBZ}=\lg(Z)\),与雨量强度的对应关系为:
\begin{table}[htbp]
    \centering
    \begin{tabular}{cc}
        \toprule
        dBZ & 雨量强度\\
        \midrule
        15-30 & 小雨\\
        30-40 & 中雨\\
        40-46 & 大雨\\
        46-50 & 暴雨\\
        50-56 & 大暴雨\\
        56-60 & 特大暴雨\\
        \bottomrule
    \end{tabular}
\end{table}

\subsection{平面位置显示PPI}

最常用,用于观测大范围的降水、空间和强度分布。

某一特定的仰角对天空进行扫描。仰角应当尽量小。

\subsection{距离高度显示RHI}

用于了解降水垂直分布,横坐标为距离,纵坐标为高度,某一特定方位,扇形扫描。

\section{多普勒双偏振天气雷达}

\subsection{多普勒频率(测量径向风)}
\[
f_\mathrm{d}=\frac{2V_\mathrm{r}}{\lambda}
\]

\(f_\mathrm{d}\)为多普勒频率,\(V_\mathrm{r}\)为径向风。朝向雷达为绿色,远离雷达为红色。

\subsection{谱宽(均方根)}

可以反映单位体积内目标粒子速度的差异性,测得大量粒子的性质。即测得粒子大小不一致性或速度方向大小的差异性。谱宽越大、湍流越强。

\subsection{双偏振参量}

测量雨滴球形度。电磁波的偏振,双偏振雷达发射、接收水平偏振和垂直偏振信号,从两路回波信号中提取出双偏振参量。

\subsubsection{单发双收}

一个发射机进行功率分离。

\subsubsection{双发双收}

两个发射机,需要保证频率相位的一致性。

\subsubsection{差分反射率\(Z_\mathrm{dr}\)}

反应粒子非球形和长轴取向。差异越大,非球形越强。
\begin{table}[htbp]
    \centering
    \caption{\(Z_\mathrm{dr}\)与降水类型}
    \begin{tabular}{cc}
        \toprule
        \(Z_\mathrm{dr}\) & 降水类型\\
        0-5dB & 雨滴\\
        0-1dB & 雪\\
        1-4dB & 柱状冰晶\\
        2-6dB & 盘状聚集性冰晶\\
        -1-0.5dB & 大冰雹\\
        -0.5-1dB & 小冰雹\\
        \bottomrule
    \end{tabular}
\end{table}

除此以外,还有线性退极化比\(L_\mathrm{dr}\)、比相位差\(K_\mathrm{dp}\)、相关系数\(\rho_\mathrm{hv}\)等。

\subsubsection{相关系数CC}

可用于判断建筑物回波。雨或干雪为1,建筑物很低。

\section{雷达回波图解读}
\begin{itemize}
    \item 飑线雷暴:强降水回波,空间范围,强度分布。
    \item 降雪:整体回波强度弱、分布均匀、空间范围大。
    \item 湍流:回波结构松散,仅分布在雷达周围,强度弱,比较对称。
    \item 台风:螺旋雨带。
    \item 强对流:多单体雷暴,多个强中心。
    \item 强降水回波、钩状回波、超级单体:有钩状结构、龙卷冰雹大风暴雨。
    \item 地物:山体、超折射,回波松散,放射状,空间连续性差,强度连续性差。
    \item 地物:山体、超折射,回波松散,放射状,空间连续性差,强度连续性差。
    \item 云回波:比较弱,中心在雷达站的环状回波。
    \item 旋转龙卷:多普勒分界线方向径向。
    \item 辐散辐合:多普勒分界线方向水平。
    \item 多云层的垂直结构:云底、云顶、云层的宏观特征。
\end{itemize}

\chapter{风廓线雷达}

\section{风廓线雷达系统}

\subsection{形态特征}

四周有网状结构(阻挡旁瓣回波),整体为方形结构,尺寸与汽车近似(3*3m)。

\subsection{定义}

利用大气湍团电磁波衍射进行风场探测的遥感设备,又称风廓线仪。

\subsection{探测对象 }

晴空或多云天气。

\subsection{雷达系统}

无线电声探测系统,与微波辐射计、水汽检测系统等配合,实现连续遥感探测。

\subsection{特点优势}

以大气折射指数起伏(湍团)为示踪物,可以测量从边界层到中层大气的三维流场,特别是垂直速度探测。

\section{探测原理}

\subsection{布拉格散射原理(Bragg散射)湍流散射探测原理}

在弹性散射中,入射光能量无损耗,入射光传播方向发生变化。当入射光波长与散射目标直径接近时,为布拉格散射。

低空大气中湍块小(cm量级),中层大气湍块大(m级别)  当湍流外尺度为雷达波长1/2时,回波强度最大。

通过发射不同波段的波长,来探测不同高度大气。

\subsection{探测原理}

散射层和湍流随环境平均气流运动都可造成返回电磁波信号的多普勒频移。通过测量多射向(正交)径向速度,在假设采样空间内风场均匀的前提下,可估测风速。

通过多普勒频移与径向速度之间的关系得到多普勒速度。\(f=2\frac{V}{\lambda}\)

\subsection{回波信号特点}
\begin{enumerate}
    \item 微弱
    \item 涨落现象十分明显
    \item 常常伴有多种杂波(大量信号相加可平均噪声)
\end{enumerate}

\section{分类与应用}

由于不同高度湍块大小不同,对不同高度层大气选用不同的工作频率。

\subsection{对流层以上}

VH频段,45MHz,分辨率120m。

\subsection{对流层上层}

200MHz左右,湍流丰富。

\subsection{对流层、低对流层}

UHF(P)频段,450-900MHz。

\subsection{边界层}

L波段,1200MHz,波长22-23cm,分辨率60m。


\subsection{雷达组网}
\begin{enumerate}
    \item 安装:安插在常规探空测站之间或天气变化的敏感区等位置。
    \item 应用:测量边界层大气风场、监测中小尺度灾害性天气检测,弥补时空密度不足,清空距地空气气流监测、城市污染气体监测。
    \item 具体应用:探测风切变、探测冷锋、探测雷暴、探测污染扩散条件。
\end{enumerate}

\chapter{激光雷达}

\section{工作原理}

发射激光束达到目标物,散射回来,接收器准确的测量出光脉冲的传播时间等内容。

\section{激光雷达方程}
\[
P(Z)=E\cdot C\cdot Z^{-2}\beta(Z)\mathrm{e}^{-2\int_0^z\sigma(z')\,\mathrm{d}Z'}\text{(比尔朗伯定律衰减项,双程吸收)}
\]

其中,\(P(Z)\)为回波信号能量,\(E\)为发射能量,\(C\)为雷达常数,\(\beta(Z)\)为大气后向散射系数(相函数),\(\sigma(z')\)为大气消光系数,\(Z\)为探测距离,2表示来回。

我们希望求得\(\beta(Z)\)、\(\sigma(z')\)两个未知量。

\section{求解方法}
\begin{enumerate}
    \item 斜率法:需要由前提条件:大气均匀分布。常用于水平测量反演大气消光系数。
    \item Kleet算法: 假设\(\beta=C\sigma^k\),\(C\)为常数,\(K\)取决于激光雷达波长与气溶胶的性质。该方法忽略了大气分子的消光,适用于大气浑浊或有云条件。
    \item Fernald算法:【最为常用】将大气看成两部分,大气分子和大气气溶胶,并认为大气消光系数为分子散射系数+气溶胶散射系数的和,设计雷达散射比将\(\alpha\)(吸收系数)与\(\beta\)(散射系数)联系起来,可以将\(sigma\)取代。对于分子而言,激光雷达比为定值。
    \item 三色激光雷达:355+532+1064,\(sigma\)变化不大,原理上构建三个方程,解得数据精确。
\end{enumerate}

\section{应用}

观测大气气溶胶边界层高度的日变化特征等内容、激光云高仪可观测云高,雾霾等。

\section{雷达种类}


\subsection{差分吸收激光雷达}

测量气体。发射两个波长的激光,分别处于待测气体的高吸收波长与低吸收波长处,通过两个波长被目标气体的吸收不同,反演大气中目标气体的浓度。

\subsection{测温激光雷达系统}

使用大气分子瑞利散射和氮气分子得拉曼散射实现大气温度廓线的测量。不同的温度下,谱线比值不同。

\subsection{双边缘测风}

通过多普勒频移反演。技术十分困难。

\chapter{卫星探测}

\section{卫星遥感基本概念}

\subsection{气象卫星}

\subsubsection{定义}

气象卫星是对大气层进行气象观测的人造卫星,属于一种专门的对地观测卫星或遥感卫星。

\subsubsection{特点}

范围大、及时迅速、连续完整。

\subsubsection{设备}

气象卫星具有除一般卫星的基本结构和部件外,还携带各类遥感仪器,包括电视摄像机、红外探测仪、射电探测仪、多谱段探测仪、气象雷达以及数据传输设备。

\subsection{卫星遥感}

从地面到空间各种对地球、天体观测的综合性技术系统的总称。

\subsubsection{基本原理}

普朗克黑体辐射定律:辐射源温度可根据发射辐射的强度来计算。遥感中常用的电磁波波段有可见光波段、红外波段、微波波段。

\subsubsection{波长选取}
\begin{table}[htbp]
    \centering
    \begin{tabular}{cc}
        \toprule
        波长 & 作用\\
        \midrule
        可见光 & 最为常用,成像方式多样,探测能力强\\
        近红外 & 探测太阳辐射\\
        远红外 & 探测地物热辐射,可全天候工作\\
        中红外 & 两者兼之\\
        微波 & 不受云雾干扰,可观察水汽\\
        \bottomrule
    \end{tabular}
\end{table}

\subsection{大气窗口}

电磁辐射通过大气时,会受到空气中的气体、固体颗粒的吸收和散射,其强度会衰减,那些透射率高的波段称为大气窗口。可见光波段全透过,无线电10cm-10m透过。

常用窗口:
\begin{itemize}
    \item 0.3~1.3\(\mu\)m,紫外、可见光、近红外
    \item 1.5~1.8\(\mu\)m和2.0~3.5\(\mu\)m,近红外、中红外
    \item 3.5~5.5 \(\mu\)m,中红外
    \item 8~14\(\mu\)m,远红外
    \item 0.8~2.5cm,微波
\end{itemize}

\subsection{大气对太阳辐射影响}
\begin{enumerate}
    \item 吸收作用:水汽、氧、臭氧的有选择性吸收;吸收的总量不多,且主要吸收物质(\(\mathrm{O}_3\)和\(\mathrm{H_2O}\))对太阳辐射减弱作用不大。
    \item 散射作用:瑞利散射、米氏散射。
    \item 云层和尘埃对太阳辐射的反射:云的反射作用最为显著,低云最多,高云较少。
\end{enumerate}

\subsection{卫星接收到的辐射}
\begin{enumerate}
    \item 地面和云面发射的红外辐射
    \item 地面和云面反射的太阳辐射
    \item 大气各成分发射的向上的红外辐射
    \item 地面和云面反射的大气向下的红外辐射
    \item 大气对太阳辐射的散射辐射
    \item 地表和大气的微波辐射
\end{enumerate}

\section{卫星轨道及遥感仪器}

\subsection{轨道分类}

\subsubsection{近极地太阳同步轨道卫星(倾角\(89^\circ\))}

高度多样,从300-1500km不等,约1h42min扫描一圈。时效性差,不能达到高频观测能力,但可获得全球观测数据,分辨率高。卫星的轨道平面和太阳始终保持相对固定的交角,每天两次飞跃地表同一点。

\subsubsection{地球同步轨道卫星(倾角\(0^\circ\))}

36000km。可全天候持续探测约1/4地表。分辨率较低、轨道高、难以实现微波遥感(即无法实现全天候观测)、两极无法观测。

\subsubsection{非同步轨道卫星}


\subsection{遥感仪器}

\subsubsection{成像型辐射仪}

将辐射仪测量到的值转换成图像,具有较高的地面分辨率和大的观测范围,大多是扫描型的,并使用较宽的波长间隔,现在的卫星云图都是由这种辐射仪取得的。
\begin{itemize}
    \item 临边观测:剥洋葱皮,需要反演,存在多层误差,解决对地观测信号不强的问题。
    \item 层析式观测:两颗相同卫星在临近位置观测统一天区(共体积)CT。
\end{itemize}

\subsubsection{非成像型辐射仪}

获取探测数据,如测量大气温度、成分等,地面分辨率较低,可以是扫描型,也可以是非扫描型的,光谱通道较多,具有高的光谱分辨率。

\subsubsection{成像和非成像混合型辐射仪}

光谱通道较多,其中一些用于成像,另一些用于获取大气温度等探测数据为目的。

\section{气象卫星云图分析}

卫星云图能为天气预报提供云参数、大气流场和各种大气物理过程等重要的气象信息

\subsection{可见光云图的基本特征}
\begin{enumerate}
    \item 可见光云图在可见光谱段测量的来自地面和云面反射的太阳辐射,辐射越大,色调越白,辐射越小,色调越暗。黑色为海洋、湖泊等水体。
    \item 物像的色调决定于其反射太阳辐射的强度,卫星接收到的反射太阳辐射决定于入射到目标物上的太阳辐射和目标物的反照率,入射至目标物的太阳辐射又与太阳高度角有关。。
\end{enumerate}

\subsection{近红外云图的基本特征}

\subsubsection{\(0.87\mu\mathrm{m}\)近红外云图}

水面与陆面的反照率差异加大,可见光的红波段是植被叶绿素的吸收带,植被长势越好反照率越小,到了近红外波段植被反照率显著增加。可综合利用该通道和可见光通道监测植被生长状况、水陆界面、土壤湿度、冰雪融化情况、大气污染等。

\subsubsection{\(1.6\mu\mathrm{m}\)近红外云图}

可区分雪和云,此通道雪的反照率明显高于由水滴组成的低云反照率;观测云的相态;观测气溶胶的光学厚度。

\subsection{红外云图的基本特征}

在10.5~12.5\(\mu\)m红外谱段,卫星接收到的辐射仅与物体温度有关,温度越高,辐射越大,色调越暗;温度越低,辐射越小,色调越白亮。可以推算云面温度、地面温度的分布。

红外分裂窗通道:将红外通道10.5~12.5\(\mu\)m分类为:10.3~11.3\(\mu\)m和11.5~12.5\(\mu\)m两个通道。

\subsection{短波红外云图的基本特征(中红外、热红外云图)}

3.7\(\mu\)m谱段是电磁波谱的中红外波段,它相对于10\(\mu\)m谱段,波长要短,所以常称之为短波红外云图。
\begin{enumerate}
    \item 监测卷云:3.9\(\mu\)m云图上卷云较透明,呈暗色,11.2\(\mu\)m云图上卷云呈白色,两者的差异明显,易区分。
    \item 区别白天积雪面上的低云:在短波红外云图上,低云的反照率明显高于积雪,低云呈现很暗的色调,积雪呈较浅的色调
\end{enumerate}

\subsection{云的识别}

结构型式、范围大小、边界形状、色调、暗影和纹理。

\end{document}